\chapter{Airspace}

\section{Objective}

To familiarize the student with the types and purpose of the various airspace
designations in the National Airspace System (NAS).

\section{Elements}

\begin{itemize}
  \item Class A,B,C,D,E,G
  \item Special-Use Airspace
  \item Other airspace
  \item Special VFR
\end{itemize}

\section{Equipment}

Sectional and/or Terminal Charts.

\section{Schedule}

Discussion (40 minutes).

\section{Instructor Actions}

%% TODO(sjr): include the airspace diagram from the IFH
Discuss each of the controlled airspaces A-E, and G, and their visibility and
cloud requirements, entry requirements, etc. (Refer to FAA-H-8083-15A
Instrument Flying Handbook Fig. 8-1).

Airspace speed limits (14 CFR 91.117)
\begin{itemize}
  \item Class B—surface to 10,000' MSL, max speed 250 kts (200 kts underlying)
  \item Class C—surface to 4000' AGL, max speed 200 kts within 4nm, 2500 AGL
  \item Class D—surface to 2500' AGL, max speed 200 kts
  \item Otherwise, no civil aircraft may operate greater than Mach 1 (14 CFR 91.817)
\end{itemize}

Special Use Airspace (AIM 3-4-1)
\begin{itemize}
  \item Prohibited areas--flight not permitted
  \item Restricted areas--flight subject to restrictions (artillery, aerial
    gunnery, etc.), permission required
  \item Warning areas--3nm beyond coast, similar to restricted, but no
    permission required
  \item MOA--military training activities, a military version of ``warning
    areas''
  \item Alert areas--ok to enter, high volume of activity (pilot training,
    unusual aerial activity)
  \item Controlled firing areas—not charted, activities are suspended if
    aircraft approach, no permission required
\end{itemize}

Other Airspace Areas (AIM 3-5-1)
\begin{itemize}
  \item MTR--military training route; military aircraft $>$ 250 kts; 4 digits $<$
    1500AGL
  \item TFR--temporary flight restriction
  \item Parachute jump operations
  \item Published VFR routes
  \item VFR flyways--back of TAC chart
  \item VFR corridors--(e.g. Los Angeles Special Flight Rules)
  \item Class B VFR transition routes--(e.g. San Diego, mini route, Hollywood
    Park route)
  \item TRSA--(e.g. Palm Springs); participation is voluntary but recommended
    (AIM 4-1-17)
  \item National Security Areas--requested not to fly through; flight may be
    prohibited at certain times, check NOTAMs
\end{itemize}

Special VFR (14 CFR 91.157, AIM 4-4-6)
\begin{itemize}
  \item 1sm visibility, clear of clouds
  \item At night, must have an instrument rating and IFR-equipped aircraft
  \item Must ask for it, ATC cannot offer
  \item Not allowed at some airports--see sectional or terminal chart
\end{itemize}

Marginal VFR (not a clearance)
\begin{itemize}
  \item Visibility between 3-5 statute miles and/or ceiling between 1000-3000' AGL
\end{itemize}

\section{Evaluation}

Lesson is complete when student can demonstrate understanding of the various
airspaces in the National Airspace System.

\section{References}

FAA-H-8083-25 Pilot's Handbook of Aeronautical Knowledge Chapter 13,
FAA-H-8083-15A Instrument Flying Handbook Chapter 8, Aeronautical Information
Manual (AIM) Chapter 3.

