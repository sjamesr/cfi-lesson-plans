\section{Short-Field Takeoff}

\subsection{Description}

Maximum performance take-off where the takeoff area is short or restricted by
obstructions.

\subsection{Objective}

To teach techniques necessary for a short field take-off to avoid obstructions
or obtain maximum performance.

\subsection{Elements}

\begin{itemize}
  \item Clear the area
  \item Choose forced landing area
  \item Configure aircraft: flaps as specified (C182T: 10$^\circ$), cowl flaps
    open, propeller to full
  \item Select outside references: vanishing point on runway
  \item Taxi onto runway centerline
  \item Apply full brakes
  \item Smoothly apply full power, upon engine peak, release brakes
  \item Anticipate need for right rudder pressure
  \item Check engine instruments (in green)
  \item Attitude slightly tail-low
  \item Do NOT rotate at $V_R$ for short field take-off
  \item At $V_X$ (C182T: 65 KIAS), apply back pressure to lift nose
    wheel
  \item Adjust pitch to maintain steep climb out at $V_X$
  \item Clear the real or simulated obstacle (if applicable)
  \item Adjust to normal climb $V_Y$ (C182T: 80 KIAS)
  \item After 200' AGL, ``positive rate, flaps up''
  \item Maintain ball centered 
  \item Look for traffic
\end{itemize}

Note $V_X$ and $V_Y$: 5 knots off can have a big difference. Emphasize using
the very beginning of the runway. On lift-off, rotate firmly. Ensure flaps
remain extended until clear of obstacle.

\subsection{References}

FAA-H-8083-3A Airplane Flying Handbook p. 5-10.

