\section{Descents and Descending Turns}

\subsection{Description}

A fundamental maneuver whereby the airplane changes attitude from level to a
descent attitude.

\subsection{Objective}

To develop the fundamental techniques required for decreasing altitude.

\subsection{Elements}

\begin{itemize}
  \item Clear the area
  \item Choose forced landing area (always be aware of options) 
  \item Select outside references
  \item Configure aircraft: reduce throttle then adjust pitch based on the type
    of descent:
    \begin{itemize}
      \item Normal descent: reduce throttle to normal descent power (C182T:
        16'' Hg, 2300 RPM), allow the nose to approx. pitch 5$^\circ$ below
        horizon then adjust as necessary to maintain either a standard rate
        (i.e.  500fpm) or constant airspeed.
      \item Descent at minimum safe airspeed: reduce throttle to normal descent
        power (C182T: 16'' Hg, 2300 RPM), adjust pitch to achieve and maintain
        1.3 $V_{S_0}$ or the short field approach speed (C182T: 60 KIAS)
      \item Glide: flaps retracted, reduce throttle to idle and pitch for and
        maintain best glide speed (C182T: Propeller 2400 RPM, 76 KIAS at MGW)
      \item For turning climbs, bank at approx. 30$^\circ$ for new heading
    \end{itemize}
  \item Maintain ball centered
  \item Trim to maintain descent rate or airspeed desired
  \item Use outside references for descent
  \item Upon reaching new altitude, anticipate altitude (approx. 100-150'),
    pitch for level flight, adjust power and trim as necessary
    \begin{itemize}
      \item For turning descents, anticipate heading (approx. 50%% of bank
        angle)
    \end{itemize}
  \item Look for traffic
\end{itemize}

\subsection{Common Errors}

\begin{itemize}
  \item Failure to adequately clear the area
  \item Inadequate back-elevator control during glide entry resulting in too
    steep a glide
  \item Failure to slow the airplane to approximate glide speed prior to
    lowering pitch attitude
  \item Attempting to establish/maintain a normal glide solely by reference to
    flight instruments
  \item Inability to sense changes in airspeed through sound and feel
  \item Inability to stabilize the glide (chasing the airspeed indicator)
  \item Attempting to ``stretch'' the glide by applying back-elevator pressure
  \item Skidding or slipping during gliding turns due to inadequate
    appreciation of the difference in rudder action as opposed to turns with
    power
  \item Failure to lower pitch attitude during gliding turn entry resulting in
    a decrease in airspeed
  \item Excessive rudder pressure during recovery from gliding turns
  \item Inadequate pitch control during recovery from straight glides
  \item ``Ground shyness'' resulting in cross-controlling during gliding turns
    near the ground
  \item Failure to maintain constant bank angle during gliding turns
\end{itemize}

\subsection{References}

FAA-H-8083-3B Airplane Flying Handbook p. 3-19.

