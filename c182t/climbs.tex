\section{Climbs and Climbing Turns}

\subsection{Description}

A fundamental maneuver whereby the airplane changes attitude from level to a
climb attitude.

\subsection{Objective}

To develop the fundamental techniques required for increasing altitude.

\subsection{Elements}

\begin{itemize}
  \item Clear the area
  \item Choose forced landing area (always be aware of options) 
  \item Select outside references
  \item Configure aircraft: pitch 5-10$^\circ$ above horizon then power as necessary
    to maintain achieve either a normal or cruise climb (C182T: 23'' Hg, 2400
    RPM, 90 KIAS), best angle of climb ($V_X$) (C182T: 23'' Hg, 2400 RPM, 65
    KIAS) or best rate of climb ($V_Y$) (C182T: 23'' Hg, 2400 RPM, 80 KIAS). 
    \begin{itemize}
      \item For turning climbs, bank at approximately 30$^\circ$ for new
        heading
    \end{itemize}
  \item Anticipate left-turning tendencies with sufficient rudder pressure
  \item Use outside references to maintain climb
  \item Trim to maintain climb
  \item Anticipate altitude (approximately 10%% of climb rate), pitch for level
    flight, adjust power and trim as necessary
    \begin{itemize}
      \item For turning climbs, anticipate heading (approximately 50%% of bank
        angle)
    \end{itemize}
  \item Maintain ball centered
  \item Look for traffic
\end{itemize}

\subsection{Common Errors}

\begin{itemize}
  \item Attempting to establish climb pitch attitude by referencing the
    airspeed indicator, resulting in ``chasing'' the airspeed
  \item Applying elevator pressure too aggressively, resulting in an excessive
    climb angle
  \item Applying elevator pressure too aggressively during level-off resulting
    in negative ``G'' forces
  \item Inadequate or inappropriate rudder pressure during climbing turns
  \item Allowing the airplane to yaw in straight climbs, usually due to
    inadequate right rudder pressure
  \item Fixation on the nose during straight climbs, resulting in climbing with
    one wing low
  \item Failure to initiate a climbing turn properly with use of rudder and
    elevators, resulting in little turn, but rather a climb with one wing low
  \item Improper coordination resulting in a slip which counteracts the effect
    of the climb, resulting in little or no altitude gain
  \item Inability to keep pitch and bank attitude constant during climbing
    turns
  \item Attempting to exceed the airplane's climb capability
\end{itemize}

\subsection{References}

FAA-H-8083-3B Airplane Flying Handbook p. 3-16.

