\section{Level Turns}

\subsection{Description}

A fundamental maneuver whereby the airplane maintains a constant altitude but
turns to a new heading.

\subsection{Objective}

To develop the fundamental techniques required for changing heading.

\subsection{Elements}

\begin{itemize}
  \item Clear the area
  \item Choose forced landing area (always be aware of options) 
  \item Configure aircraft for cruise (C182T: 23'' Hg, 2300 RPM)
  \item Select outside references (point on the horizon corresponding to desired heading)
  \item Periodically insure the nose is fixed below the horizon
  \item Gently bank the airplane to no more than 20$^\circ$ for shallow turns
    when learning turning (20$^\circ$ to 45$^\circ$ for medium turns when more
    proficient) and maintain this bank until approaching desired heading
  \item Apply rudder in the direction of the bank to keep the ball centered
    while applying elevator back-pressure to maintain level flight (constant
    altitude)
  \item Use wingtips as reference of banking angle
  \item Anticipate rollout to new heading by leading with half the bank angle
    (10$^\circ$ for a 20$^\circ$ bank)
  \item Trim as necessary to maintain altitude 
  \item Look for traffic
\end{itemize}

\subsection{Common Errors}

\begin{itemize}
  \item Failure to adequately clear the area before beginning the turn
  \item Attempting to execute the turn solely by instrument reference
  \item Attempting to sit up straight, in relation to the ground, during a
    turn, rather than riding with the airplane
  \item Insufficient feel for the airplane as evidenced by the inability to
    detect slips/skids without reference to flight instruments
  \item Attempting to maintain a constant bank angle by referencing the
    ``cant'' of the airplane's nose
  \item Fixating on the nose reference while excluding wingtip reference
  \item ``Ground shyness'' -- making ``flat turns'' (skidding) while operating
    at low altitudes in a conscious or subconscious effort to avoid banking
    close to the ground
  \item Holding rudder in the turn
  \item Gaining proficiency in turns in only one direction (usually the left)
  \item Failure to coordinate the use of throttle with other controls
  \item Altitude gain/loss during the turn
\end{itemize}

\subsection{References}

FAA-H-8083-3B Airplane Flying Handbook p. 3-10.

