\section{Power-On Stalls}

\subsection{Description}

A rapid degeneration of lift as a result of excessive angle of attack, entered
from takeoff and climb configuration.

\subsection{Objective}

To teach recognition and recovery from a full stall under take-off conditions
and required recovery action.

\subsection{Setup}

\begin{itemize}
  \item Clear the area
  \item Choose forced landing area
  \item Configure aircraft for just before take-off: flaps 10$^\circ$, slow to
    normal take-off speed with max propeller RPM (C182T: 14.5'' Hg, 2400 RPM, 70
    KIAS), altitude so recovery is $\geq 1500'$ AGL
  \item Select outside references
  \item Throttle to full while simultaneously applying back-elevator pressure
    to smoothly raise nose to a high pitch attitude
  \item Maintain back-elevator pressure as necessary until airspeed falls
    below $V_{S_1}$ (C182T: 49 KIAS) and stalls
  \item Maintain coordination (ball centered) and neutral ailerons
\end{itemize}

\subsection{Recovery}

\begin{itemize}
  \item Reduce the angle of attack by releasing back-elevator pressure
  \item Simultaneously increasing throttle to full (if not already)
  \item Anticipate left-turning tendencies with right rudder pressure
  \item Return nose to straight-and-level coordinated flight 
  \item Maintain ball centered
  \item Upon positive rate of climb, retract flaps as necessary 
  \item Look for traffic
\end{itemize}

Practice both straight-and-level and turning stalls (up to 20$^\circ$).
Indicators: speed, buffeting and stall horn.

\subsection{Common Errors}

\begin{itemize}
  \item Failure to adequately clear the area
  \item Inability to recognize an approaching stall condition through feel for
    the airplane
  \item Premature recovery
  \item Over-reliance on the airspeed indicator while excluding other cues
  \item Inadequate scanning resulting in an unintentional wing-low condition
    during entry
  \item Excessive back-elevator pressure resulting in an exaggerated nose-up
    attitude during entry
  \item Inadequate rudder control
  \item Inadvertent secondary stall during recovery
  \item Failure to maintain a constant bank angle during turning stalls
  \item Excessive forward-elevator pressure during recovery resulting in
    negative load on the wings
  \item Excessive airspeed buildup during recovery
  \item Failure to take timely action to prevent a full stall during the
    conduct of imminent stalls
\end{itemize}

\subsection{References}

FAA-H-8083-3A Airplane Flying Handbook p. 4-9.

