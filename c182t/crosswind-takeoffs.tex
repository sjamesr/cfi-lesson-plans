\section{Crosswind Takeoffs}

\subsection{Description}

Takeoff roll, lift-off and initial climb with some wind acting perpendicular to
the runway.

\subsection{Objective}

To teach techniques necessary for a takeoff when the wind is not aligned with
the runway.

\subsection{Elements}

\begin{itemize}
  \item Clear the area
  \item Choose forced landing area
  \item Configure aircraft: flaps 10$^\circ$, cowl flaps open, propeller to
    full
  \item Select outside references: vanishing point on runway
  \item Taxi onto runway centerline, using wind correction during taxi
  \item Position full ailerons into the direction of the wind (turn into for
    headwinds, dive away from tailwinds)
  \item Smoothly apply full power
  \item Anticipate need for right rudder pressure and to maintain centerline
  \item Check engine instruments (in green)
  \item As controls become effective, gradually reduce aileron / rudder
    pressures
  \item At $V_R$ (C182T: 59 KIAS), gradually apply back pressure to lift nose
    wheel
  \item Pitch for normal climb attitude, climb at $V_Y$ (C182T: 80 KIAS)
  \item Crab into wind as necessary to maintain extended runway centerline
  \item At around 200ft AGL and at climb airspeed, call ``positive rate, flaps
    up''
  \item Maintain ball centered 
  \item Look for traffic
\end{itemize}

Discuss maximum demonstrated crosswind component. Discuss taxi wind correction
(turn into for headwinds, dive away from tailwinds).

\subsection{Common Errors}

\begin{itemize}
  \item Failure to adequately clear the area prior to taxiing onto the active
    runway
  \item Using less than full aileron pressure into the wind initially on the
    takeoff roll
  \item Mechanical use of aileron control rather than sensing the need for
    varying aileron control input through feel for the airplane
  \item Premature lift-off resulting in side-skipping
  \item Excessive aileron input in the latter stage of the takeoff roll
    resulting in a steep bank into the wind at lift-off
  \item Inadequate drift correction after lift-off
\end{itemize}

\subsection{References}

FAA-H-8083-3B Airplane Flying Handbook p. 5-6.

