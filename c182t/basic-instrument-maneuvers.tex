\section{Basic Instrument Maneuvers}

\subsection{Description}

Use of instruments solely as a mean to navigate,
necessary for inadvertent entry into IMC.

\subsection{Objective}

To teach basic flight maneuvers solely by reference to
instruments.

\subsection{Elements}

\begin{itemize}
  \item Clear the area
  \item Choose forced landing area
  \item Configure aircraft for cruise or as necessary (C182T: 23'' Hg, 2300
    RPM)
  \item Student: wear a view-limiting device
  \item Straight-and-level flight -- maintain heading and altitude using heading
indicator and attitude indicator; pitch only one-half bar width on the attitude
indicator for less than 100 feet of altitude change
  \item Standard rate turns -- bank using attitude indicator (approx.
    15$^\circ$) maintain airspeed control while using turn coordinator to judge
    turn, use ball to judge quality of turn, use heading indicator to roll out
    on new heading at 50%% of bank angle prior to heading
  \item Climbs and descents -- perform climbs and descents using airspeed,
    vertical speed (rate) or both
    \begin{itemize}
      \item Climbs: Pitch, Power, Trim; Descents: Power, Pitch, Trim
      \item Leveling out: for climbs: 10%% of climb rate prior to altitude; for
        descents: 100-150' prior to altitude
      \item With turns, roll out on new heading at 50%% of bank angle prior to
        heading
    \end{itemize}
  \item Maintain ball centered
  \item Instructor looks for traffic while student performs under a hood
\end{itemize}

\subsection{Common Errors}

\begin{itemize}
  \item Failure to properly interpret instruments
  \item Failure to cross-reference instruments
  \item Fixation on one instrument
  \item Omission of instruments from cross-check
\end{itemize}

\subsection{References}

FAA-H-8083-15A Instrument Flying Handbook Chapter 4 and 5.
