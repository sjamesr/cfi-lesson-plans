\section{Normal Takeoff}

\subsection{Description}

Takeoff roll, lift-off and initial climb with headwind or light wind, hard
surface of sufficient length and no obstructions.

\subsection{Objective}

To teach techniques necessary for a normal takeoff.

\subsection{Elements}

\begin{itemize}
  \item Clear the area
  \item Choose forced landing area
  \item Configure aircraft: flaps 10$^\circ$, cowl flaps open, propeller to
    full
  \item Select outside references: vanishing point on runway
  \item Taxi onto runway centerline
  \item Smoothly apply full power
  \item Anticipate need for right rudder pressure
  \item Check engine instruments (in green)
  \item At $V_R$ (C182T: 59 KIAS), apply slight back pressure to lift nose
    wheel (objective is not to take-off at $V_R$, but to position aircraft for
    take-off; as airspeed builds, aircraft will take-off)
  \item Pitch for normal climb attitude, climb at $V_Y$ (C182T: 80 KIAS)
  \item At around 200ft, best climb speed established, call out ``positive
    rate, flaps up''
  \item Maintain ball centered 
  \item Look for traffic
\end{itemize}

\subsection{Common Errors}

\begin{itemize}
  \item Failure to adequately clear the area prior to taxiing into position on
    the active runway
  \item Abrupt use of the throttle
  \item Failure to check engine instruments for signs of malfunction after
    applying takeoff power
  \item Failure to anticipate the airplane's left turning tendency on initial
    acceleration
  \item Overcorrecting for left turning tendency
  \item Relying solely on the airspeed indicator rather than developed feel for
    indications of speed and airplane controllability during acceleration and
    lift-off
  \item Failure to attain proper lift-off attitude
  \item Inadequate compensation for torque/P-factor during initial climb
    resulting in a sideslip
  \item Over-control of elevators during initial climbout
  \item Limiting scan to areas directly ahead of the airplane (pitch attitude
    and direction), resulting in allowing a wing (usually the left) to drop
    immediately after lift-off
  \item Failure to attain/maintain best rate-of-climb airspeed ($V_Y$)
  \item Failure to employ the principles of attitude flying during climb-out,
    resulting in ``chasing'' the airspeed indicator
\end{itemize}

\subsection{References}

FAA-H-8083-3B Airplane Flying Handbook p. 5-3.
