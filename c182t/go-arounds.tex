\section{Go-Arounds}

\subsection{Description}

Whenever landing conditions are not satisfactory, a rejected landing is
warranted. The need for a go-around is not a consequence of a poor approach and
is considered as a normal maneuver.

\subsection{Objective}

To teach the procedures of a go-around and to emphasize the need to be prepared
for unexpected situations during landing.

\subsection{Elements}

\begin{itemize}
  \item Clear the area
  \item Choose forced landing area (runway)
  \item Configure aircraft: landing checklist, begin the final approach to land
    (C182T: 14.5'' Hg, 2400 RPM)
  \item Select outside references (field)
  \item Commit to go-around (don't change mind!) 
  \item Smoothly apply throttle to full while simultaneously anticipating
    left-turning tendencies with right rudder pressure 
  \item Adjust pitch for $V_X$ (C182T: 65 KIAS)
  \item Establish positive rate of climb
  \item Smoothly retract first notch of flaps; stabilize attitude (C182T:
    Retract to 20$^\circ$)
  \item Rough trim as needed
  \item Maneuver airplane to the traffic-pattern side of the runway and fly
    parallel to it
  \item Smoothly retract the remaining flaps incrementally (allow aircraft to
    stabilize)
  \item Adjust pitch for $V_Y$ (C182T: 80 KIAS)
  \item Open cowl flaps 
  \item Maintain ball centered
  \item Look for traffic
\end{itemize}

\subsection{Common Errors}

\begin{itemize}
  \item Failing to commit to a go-around when situation warrants
  \item Failing to anticipate and adjust for pitch to maintain safe airspeed
  \item Failing to anticipate and adjust for left turning tendencies 
  \item Failing to apply full go-around power
  \item Not selecting flaps $^\circ$ after setting power and establishing proper pitch
  \item Retracting flaps before reaching a safe airspeed
\end{itemize}

\subsection{References}

FAA-H-8083-3B Airplane Flying Handbook p. 8-12.
