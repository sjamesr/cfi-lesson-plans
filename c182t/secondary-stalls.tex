\section{Secondary Stalls}

\subsection{Description}

A stall resulting from improper stall recovery technique of a power-on or
power-off stall.

\subsection{Objective}

To demonstrate secondary stalls to show how to recognize the characteristics of
the stall and the correct methods of recovery.

\subsection{Setup}

\begin{itemize}
  \item Clear the area
  \item Choose forced landing area
  \item Configure aircraft for a normal power-on or power-off stall, with
    altitude as necessary to recover $\geq 1500'$ AGL
  \item Select outside references
  \item During recovery from initial stall, quickly increase the pitch attitude
    again while maintaining directional control with aileron and rudder
    pressure, allowing the airplane to enter a secondary stall
\end{itemize}

\subsection{Recovery}

\begin{itemize}
  \item Reduce the angle of attack by releasing back-elevator pressure,
    simultaneously increasing throttle to full (if not already)
  \item Anticipate left-turning tendencies with right rudder pressure
  \item Return nose to straight-and-level, coordinated flight 
  \item Maintain ball centered
  \item Upon positive rate of climb, retract flaps as necessary 
  \item Look for traffic
\end{itemize}

\subsection{References}

FAA-H-8083-3A Airplane Flying Handbook p. 4-10.

