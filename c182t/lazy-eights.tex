\section{Lazy Eights}

\subsection{Description}

A maneuver that is basically two 180$^\circ$ turns in opposite directions, with
each turn including a climb and descent.

\subsection{Objective}

To develop the smoothness, coordination, orientation, planning, division of
attention, and ability to maintain precise aircraft control.

\subsection{Elements}

\begin{itemize}
  \item Clear the area
  \item Choose forced landing area
  \item Configure aircraft: straight-and-level, airspeed less than $V_A$
    (C182T: 18'' Hg, 2300 RPM, 110 KIAS at MGW), altitude $\geq$ 1600' AGL 
  \item Select outside references (45$^\circ$, 90$^\circ$, 135$^\circ$ points)
  \item Gradual climbing turn toward 45$^\circ$ point, increasing pitch and
    bank so that at 45$^\circ$ point you are highest pitch and 15$^\circ$ of
    bank (tip: adjust pitch before bank, e.g. ``pitch pitch pitch bank'') 
  \item From 45$^\circ$ to 90$^\circ$ decrease pitch to horizon while
    increasing your bank and maintain coordinated flight so that at 90$^\circ$
    you are level with maximum altitude, approx. 30$^\circ$ bank and 5-10 KIAS
    above stall, all the while gradually reducing elevator back pressure
  \item From 90$^\circ$ to 135$^\circ$ slowly roll out bank while gradually
    lowering the nose so that upon reaching 135$^\circ$ the nose is at the
    lowest pitch approx. equal but opposite of the highest pitch achieved
    earlier, with bank angle of 15$^\circ$, all the while gradually relaxing
    rudder and aileron pressure
  \item Arrive at 180$^\circ$ point straight-and-level at the original heading,
    altitude and airspeed
  \item Maintain ball centered 
  \item Look for traffic
  \item Continue the maneuver in the opposite direction
\end{itemize}

\subsection{Common Errors}

\begin{itemize}

  \item Failure to clear area
  \item Using the nose or engine cowling instead of the true longitudinal axis,
    resulting in unsymmetrical loops
  \item Watching the airplane instead of the reference points
  \item Inadequate planning, resulting in peaks of the loops both above and
    below the horizon not coming the proper place
  \item Control roughness, usually caused by attempts to counteract poor
    planning
  \item Persistent gain or loss of altitude with the completion of each eight
  \item Attempting to perform the maneuver rhythmically, resulting in poor
    pattern symmetry.
  \item Allowing the airplane to ``fall'' out of the tops of the loops rather
    than flying the airplane through the maneuver
  \item Slipping and/or skidding
  \item Failure to scan for traffic during the maneuver

\end{itemize}

\subsection{References}

FAA-H-8083-3B Airplane Flying Handbook p. 9-6.

