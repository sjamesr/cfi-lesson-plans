\section{Fuel System (C182T)}

\subsection{Objective}

To teach the components and operating procedures of the fuel system.

\subsection{Elements}

\begin{itemize}
  \item Components
  \item Pre-flight
  \item Normal operation
  \item Emergency operation
\end{itemize}

\subsection{Schedule}

Discussion (30 minutes).

\subsection{Equipment}

Aircraft, Pilot Operating Handbook (POH) or FAA-approved Airplane Flight Manual (AFM).

\subsection{Instructor Actions}

Discuss the following components:

\begin{itemize}
  \item Two vented integral fuel tanks--fuel flows by gravity from the tanks
    \begin{itemize}
      \item Standard tank capacity is 46 gallons (total 92 gal), and
        useable capacity is 43.5 gallons (total 87 gal)
    \end{itemize}

  \item Fuel tank vent--venting is accomplished by an interconnecting line
    between the tanks, with a check-valve-equipped overboard vents in each
    tank. The vents protrude from the bottom of the wing behind the struts.
    Both fuel caps are also vented in case the overboard vents become blocked.

  \item Fuel gauges--indicate the amount of fuel measured by a sensing unit
    in each tank and is displayed in gallons.

  \item Fuel sumps and drains--allow for checks at preflight to be made in
    the fuel tanks, selector, and strains, of visible moisture and/or
    sediments, as well as check for the proper grade of fuel

  \item Four-position selector valve--the selector can be set to OFF, BOTH,
    LEFT, and RIGHT; when the selector is not set to OFF, fuel is able to
    flow through to the rest of the system

  \item Fuel strainer--removes any impurities, including moisture and other
    sediments that might be present in the fuel

  \item Fuel pressure gauge--shows the fuel flow in PSI and can be used to
    indicate a failure in of the fuel pump

  \item Engine-driven fuel pump--driven by the engine to pump fuel to the
    carburetor

  \item Electric auxiliary fuel pump--electrically drives fuel to the injector
    for priming the engine, as well as for vapor suppression in hot weather

  \item Return fuel system--for improved operation during extended idle
    operations in hot weather, designed to return fuel/vapor back to the tanks
    at a rate of approximately 7 gallons per hour
\end{itemize}

Discuss fuel grades: Aviation gasoline (AVGAS) is identified by an octane or
performance number (grade). The higher the grade of gasoline, the more pressure
the fuel can withstand without detonating. If the proper grade of fuel is not
available, use the next higher grade as a substitute (but not JET A). Never use
a lower grade. This can cause the cylinder head temperature and engine oil
temperature to exceed their normal operating range, which may result in
detonation. Available AVGAS is 80 (dyed red), 100 (dyed green), and 100LL (dyed
blue). The C182T used 100LL.

Discuss pre-ignition and detonation.

Discuss refueling, including grounding, use of a ladder, etc. Note that if
refueling before flight, should redo sumping after the fuel has settled (at
least 10 minutes).

Discuss preflight of the fuel system per POH.

\subsection{Evaluation}

Lesson is complete when student can demonstrate and discuss proper use of fuel
system.

\subsection{References}

FAA-H-8083-25B Pilot's Handbook of Aeronautical Knowledge p. 7-25, POH / AFM p.
7-38.
