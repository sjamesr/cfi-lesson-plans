\section{Power-Off 180$^\circ$ Accuracy Approach and Landing}

\subsection{Description}

Approach and landing made by gliding with the engine idling from downwind to a
touchdown beyond and within 200 feet of a designated line or mark on the
runway.

\subsection{Objective}

To teach judgment in and procedures necessary for accurately flying the
airplane, without power, to a safe landing.

\subsection{Elements}

\begin{itemize}
  \item Clear the area
  \item Choose forced landing area (runway)
  \item Configure aircraft: landing checklist (GUMPFS, no flaps), approx.
    1000' AGL downwind for landing (C182T: 16.5'' Hg, 2400 RPM)
  \item Select outside references (e.g. aim point markers)
  \item Throttle to idle abeam touchdown
  \item Establish and maintain best glide speed or 1.4 $V_{S_0}$ (C182T: 76
    KIAS 0$^\circ$ Flaps)
  \item Select aim point slightly before touchdown point (compensate for ground
    effect)
  \item Position and adjust base leg as necessary and compensate for wind
  \item Flaps or forward slip as necessary for landing on touchdown point
  \item Land at touchdown point beyond and within 200' (How to measure? Runway
    edge lights are spaced 50', centerline strips usually 200' from beginning
    of one to beginning of next)
\end{itemize}

\subsection{Common Errors}

\begin{itemize}
  \item Downwind leg too far from runway/landing area
  \item Overextension of downwind leg resulting from tailwind
  \item Inadequate compensation for wind drift on base leg
  \item Skidding turns in an effort to increase gliding distance
  \item Attempting to ``stretch'' the glide during undershoot
  \item Premature flap extension
  \item Use of throttle to increase the glide instead of merely clearing the
    engine
  \item Forcing the airplane onto the runway to avoid overshooting the
    touchdown point
\end{itemize}

\subsection{References}

FAA-H-8083-3B Airplane Flying Handbook p. 8-23.
