\section{Crosswind Landings}

\subsection{Description}

Approach and landing with some wind acting perpendicular to the runway.

\subsection{Objective}

To teach techniques necessary for a landing when the wind is not aligned with
the runway.

\subsection{Elements}

\begin{itemize}
  \item Clear the area
  \item Choose forced landing area (should be runway)
  \item Configure aircraft for normal approach and begin descent as normal
  \item Select outside references (e.g. runway numbers)
  \item Clear area, then turn to final ($\leq$ 30$^\circ$ bank)
  \item On final: remaining flaps (C182T: 30$^\circ$ when runway is assured,
    GUMPFS check
  \item Select aim point (e.g. before runway numbers)
  \item Maintain centerline: Turn the aircraft into the wind in order to
    counteract drift
  \item Adjust pitch and power to maintain normal approach speed and descent
    angle (C182T: 70 KIAS)
  \item Trim to relieve control pressures
  \item At round-out: rudder as required to align the nose of the aircraft with
    the centerline, bank as required to oppose aircraft drift
  \item Make sure feet are not on brakes
  \item 10-20' off ground: reduce throttle to idle
  \item Gradually apply back pressure to pitch for landing attitude, attempting
    to fly just above runway (fly in ground effect) straight-and-level until
    passing aim point, then continue adjusting pitch for climb attitude just
    above horizon
  \item Maintain crosswind control inputs to the surface, as airspeed
    decreases, greater crosswind control inputs are required 
  \item Touchdown on upwind main gear first, just above stalling speed
  \item Maintain pitch attitude for aerodynamic braking
  \item Gradually relax back pressure to lower nose wheel while applying
    maximum aileron in upwind direction
  \item Gentle braking as required
  \item Crosswind control inputs for taxi
\end{itemize}

Discuss maximum demonstrated crosswind component. Discuss taxi wind correction
(turn into for headwinds, dive away from tailwinds). Discuss turbulent air
approaches (i.e. consider a no-flap landing or less flaps, allowing faster
approach; or use the half gust factor plus normal approach speed for the
approach). Keep one hand on throttle.

\subsection{Common Errors}

\begin{itemize}
  \item Attempting to land in crosswinds that exceed the airplane's maximum
    demonstrated crosswind component or pilot's ability
  \item Inadequate compensation for wind drift on the turn from base leg to final
    approach, resulting in undershooting or overshooting
  \item Inadequate compensation for wind drift on final approach
  \item Unstabilized approach
  \item Failure to compensate for increased drag during sideslip resulting in
    excessive sink rate and/or too low an airspeed
  \item Touchdown while drifting
  \item Excessive airspeed on touchdown
  \item Failure to apply appropriate flight control inputs during rollout
  \item Failure to maintain direction control on rollout
  \item Excessive braking
\end{itemize}

\subsection{References}

FAA-H-8083-3A Airplane Flying Handbook p. 8-14.

