\section{Short-Field Landing}

\subsection{Description}

Maximum performance landing where the landing area is short or restricted by
obstructions.

\subsection{Objective}

To teach techniques necessary for a short field landing to avoid obstructions
or minimize ground roll.

\subsection{Elements}

\begin{itemize}
  \item Clear the area
  \item Choose forced landing area (should be runway)
  \item Configure aircraft for normal approach and begin descent as normal
  \item Select outside references (e.g. runway numbers)
  \item Clear area, then turn to final ($\leq 30^\circ$ bank)
  \item On final: remaining flaps (C172RG: 30$^\circ$) when runway is assured, CCGUMPS check
  \item Select aim point (e.g. before runway numbers)
  \item Adjust pitch and power to maintain short field approach speed and
    relatively steep descent angle of 1.3 $V_{S_0}$ or as specified (C172RG: 63
    KIAS)
  \item Trim to relieve control pressures
  \item Make sure feet are not on brakes
  \item Insure stabilized approach from 500' AGL
  \item 10-20' off ground: reduce throttle to idle
  \item Gradually apply back pressure to pitch for landing attitude: (don't fly
    into ground effect, cut through it) when passing aim point, adjusting
    pitch for climb attitude just above horizon
  \item Touchdown on main gear at minimum controllable airspeed with little or
    no float, just above a power-off stall, touchdown at selected point beyond
    and within 100'
  \item Maintain pitch attitude for aerodynamic braking
  \item Smoothly relax back pressure to quickly lower nose wheel
  \item Flaps up (simulate, necessary to put more weight on gear)
  \item Heavy braking as required (simulate)
\end{itemize}

Include a discussion on performance charts and other landing scenarios. Keep
one hand on throttle. Discuss steeper than normal approach. Emphasize brakes
only after touchdown, but simulated for practice landings. A wider than normal
pattern can be used to give time to configure the airplane.

\subsection{Common Errors}

\begin{itemize}
  \item Failure to allow enough room on final to set up the approach,
    necessitating an overly steep approach and high sink rate
  \item Unstabilized approach
  \item Undue delay in initiating glidepath corrections
  \item Too low an airspeed on final resulting in inability to flare properly
    and landing hard
  \item Too high an airspeed resulting in floating on roundout
  \item Prematurely reducing power to idle on roundout resulting in hard
    landing
  \item Touchdown with excessive airspeed
  \item Excessive and/or unnecessary braking after touchdown
  \item Failure to maintain directional control
\end{itemize}

\subsection{References}

FAA-H-8083-3A Airplane Flying Handbook p. 8-17.
