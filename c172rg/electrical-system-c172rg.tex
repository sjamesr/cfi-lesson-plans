\chapter{Electrical System (C172RG)}

\section{Objective}

To teach the components and operating procedures of the electrical system.

\section{Elements}

\begin{itemize}
  \item Components
  \item Pre-flight
  \item Emergency operation
\end{itemize}

\section{Schedule}

Discussion (30 minutes).

\section{Equipment}

Aircraft, Pilot Operating Handbook (POH) or FAA-approved Airplane Flight Manual (AFM).

\section{Instructor Actions}

Discuss the following components:
\begin{itemize}
  \item 28V DC System

  \item Battery -- 24V Located aft of the rear cabin wall

  \item Alternator -- 60A Belt-driven

  \item Buses: Primary and Avionics (interconnected with primary via avionics
    power switch/breaker)

  \item Master Switch -- split switch: battery and alternator

  \item Avionics Power Switch -- power from primary to avionics bus; is also a
    circuit breaker

  \item Ammeter -- indicates battery charging rate when alternator is on and
    working, or rate of battery discharge when alternator is off or
    malfunctioning

  \item Alternator Control Unit (ACU) and Low Voltage Warning -- combo alternator
    regulator and high-low voltage control unit; mounted on engine side of
    firewall

  \item ``LOW VOLTAGE'' light on instrument panel

  \item Circuit Breakers and Fuses -- Most are ``push to reset'' except
    ``Alternator Output'' and ``Landing Gear'' which are ``pull-off'' type and
    the ``AVN PWR'' which is a rocker switch; cigarette lighter and control
    wheel map light uses fuses as well as breakers

  \item Ground Service Plug Receptacle (Optional) - for use with external power
    during cold weather starting or lengthy maintenance work

  \item Lighting System: 3 navigation, taxi, landing, rotating beacon, strobes,
    courtesy, interior, flood, post lights (outside instruments), integral
    (inside instruments)

  \item Electrical instruments (turn coordinator, clock)

  \item Radio and navigation devices
\end{itemize}

Discuss pre-flight checklist in POH/AFM as it pertains to electrical system
items. In particular, note the ammeter during run-up with and without an
electrical load. Discuss the electrical fire checklist in the POH/AFM. Discuss
if the ammeter shows excessive rate of charge. Discuss if the low voltage light
illuminates during low RPM operations on the ground and goes off as RPM in
increased, this is not a problem. Otherwise, follow checklist procedures.

\section{Evaluation}

Lesson is complete when student can demonstrate and discuss proper use of
electrical system.

\section{References}

FAA-H-8083-25 Pilot's Handbook of Aeronautical Knowledge p. 5-19, POH / AFM pp.
3-6, 3-10, 4-21, 7-27, 7-29 thru 7-34, POH Supplement: Ground Service
Receptacle 1 thru 4, electrical diagram on p. 7-30.
