\section{Eights on Pylons}

\subsection{Description}

An advanced training maneuver in which the pilot's attention is directed at
maintaining a pivotal position on a selected pylon, with a minimum of attention
within the cockpit.

\subsection{Objective}

To develop the ability to control the airplane accurately while dividing
attention between the flightpath and the selected points on the ground. 

\subsection{Elements}

\begin{itemize}
  \item Clear the area
  \item Choose forced landing area
  \item Configure aircraft: flaps and gear up, maneuvering power (C172RG: 18''
    Hg, 2300 RPM, approx. 100 KIAS), adjust altitude to pivotal altitude, pitch
    and trim to maintain pivotal altitude
    \begin{itemize}
      \item Pivotal altitude in AGL = $GS^2 / 11.3$ (assuming ground speed in
        knots)
      \item Pivotal altitude at 100 KIAS = 885' AGL, if 10 knots of winds,
        lowest groundspeed is 90 KIAS = 717' AGL, highest groundspeed is 110
        KIAS = 1071' AGL
    \end{itemize}
  \item Select pylon -- Fly with the winds at the 45$^\circ$ quartering
    tailwind position, maintaining pivotal altitude until an ideal pylon (point
    as a ground reference) is selected, which should be within the 45$^\circ$
    off the left side of the flight path and close enough for a 30$^\circ$ to
    40$^\circ$ bank
  \item Bank onto pylon (30$^\circ$ to 40$^\circ$) -- Maintain
    straight-and-level flight at pivotal altitude until abeam the first pylon
    (it should be off the left wingtip), then roll 30$^\circ$ to 40$^\circ$
    angle of bank
  \item Adjust pivotal altitude -- Gradually decrease pivotal altitude and
    slightly reduce bank angle as you turn about the pylon, turning directly
    into the wind
  \item Keep wingtip reference on pylon -- Throughout, maintain ball centered
    and pivotal altitude necessary to hold wingtips (i.e. a row of rivets along
    the wingtip) on pylon, adjusting with pivotal altitude and bank changes
    only, not with power or rudder; no need for constant radius
  \item Wings level (3-5 seconds) -- Begin the rollout to straight-and-level
    flight as the first turn is completed, then maintain straight-and-level
    flight and crab into the wind as necessary, flying straight-and-level for 3
    to 5 seconds, climbing if necessary to return to pivotal altitude
  \item Select second pylon -- Select a second pylon, within 45$^\circ$ off the
    right side of the flight path and close enough for a 30$^\circ$ to
    40$^\circ$ bank
  \item Bank onto pylon (30$^\circ$ to 40$^\circ$) -- Maintain
    straight-and-level flight at pivotal altitude until abeam the first pylon
    (it should be off the right wingtip), then roll 30$^\circ$ to 40$^\circ$
    angle of bank, noting that your line of reference may look different in the
    opposite turn
  \item Adjust pivotal altitude -- Gradually decrease pivotal altitude and
    slightly reduce bank angle as you turn about the pylon, turning directly
    into the wind
  \item Keep wingtip reference on pylon -- Throughout, maintain ball centered and
    pivotal altitude necessary to hold wingtips (i.e. a row of rivets along the
    wingtip) on pylons
  \item Avoid slips and skids
  \item Repeat the maneuver as necessary or recover by exiting from the
    straight-leg between turns
  \item Look for traffic throughout
\end{itemize}

Factors not important in the maneuver: bank and radius. Draw the perspective
from the pilot: if the pylon moves back, pull back; if the pylon moves forward,
push forward.

An alternative way to select pylons: fly with the wind perpendicular to the
flightpath, select your first pylon, 7 seconds later your midpoint, 7 seconds
later your second pylon. Then circle back to enter maneuver.

\subsection{Common Errors}

\begin{itemize}
  \item Failure to adequately clear the area
  \item Skidding or slipping in turns
  \item Attempting to maintain the pylon with yaw
  \item Excessive gain or loss of altitude
  \item Over concentration on the pylon and failure to observe traffic
  \item Poor choice of pylons
  \item Not entering the pylon turns into the wind
  \item Failure to assume a heading when flying between pylons that will
    compensate sufficiently for drift
  \item Failure to time the bank so that the turn entry is completed with the
    pylon in position
  \item Abrupt control usage
  \item Inability to select pivotal altitude
\end{itemize}

\subsection{References}

FAA-H-8083-3A Airplane Flying Handbook p. 6-13.
