\section{Soft-Field Landings}

\subsection{Description}

Minimum descent rate landing to a soft field, designed to touchdown as softly
as possible to eliminate risk of rough landing caused by tall grass, soft sand,
mud or snow.

\subsection{Objective}

To teach techniques necessary for landing when it is necessary to touchdown as
softly as possible by slowly transferring weight from wings to landing gear.

\subsection{Elements}

\begin{itemize}
  \item Clear the area
  \item Choose forced landing area (should be runway)
  \item Configure aircraft for normal approach and begin descent as normal
  \item Select outside references (e.g. runway numbers)
  \item Clear area, then turn to final ($\leq 30^\circ$ bank)
  \item On final: remaining flaps (C172RG: $30^\circ$) when runway is assured,
    CCGUMPS check
  \item Select aim point (e.g. before runway numbers)
  \item Adjust pitch and power to maintain normal approach speed and descent
    angle (C172RG: 65 KIAS)
  \item Trim to relieve control pressures
  \item Make sure feet are not on brakes
  \item 10-20' off ground: power to maintain minimum descent rate (e.g.
    throttle to idle then re-add some power to slow descent rate) (C172RG:
    maintain descent power to the ground)
  \item Gradually apply back pressure to pitch for straight-and-level attitude,
    attempting to fly just above runway (fly in ground effect)
    straight-and-level until passing aim point, then continue adjusting pitch
    for climb attitude just above horizon
  \item Hold aircraft in ground effect, 1-2' above surface, as long as possible
  \item Upon touchdown on main gear, smoothly apply elevator pressure to hold
    nose wheel off surface as long as possible (C172RG: instrument glare shield
    on horizon)
  \item Gently lower the nose wheel to the surface
  \item Slight addition of throttle upon touchdown can assist softly lowering
    the nose wheel
  \item Do not use brakes and maintain elevator back pressure for taxiing
\end{itemize}

Emphasize flying in ground effect as long as possible; holding back pressure on
elevator throughout taxi. Keep one hand on throttle. A wider than normal
pattern can be used to give time to configure the airplane.

\subsection{Common Errors}

\begin{itemize}
  \item Excessive descent rate on final approach
  \item Excessive airspeed on final approach
  \item Unstabilized approach
  \item Roundout too high above the runway surface
  \item Poor power management during roundout and touchdown
  \item Hard touchdown
  \item Inadequate control of the airplane weight transfer from wings to wheels
    after touchdown
  \item Allowing the nose wheel to ``fall'' to the runway after touchdown
    rather than controlling its descent
\end{itemize}

\subsection{References}

FAA-H-8083-3A Airplane Flying Handbook p. 8-19.

