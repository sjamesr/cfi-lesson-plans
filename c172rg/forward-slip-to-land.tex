\section{Forward Slip to a Landing}

\subsection{Description}

An intentionally uncoordinated maneuver when the bank angle of an airplane is
too steep for the existing rate of turn but the ground track is kept parallel
to the runway. Useful in forced landings and steep approaches for landing in
confined areas.

\subsection{Objective}

To teach judgment and procedures necessary to descend rapidly while maintaining
control and ground track to allow for a safe landing.

\subsection{Elements}

\begin{itemize}
  \item Clear the area
  \item Choose forced landing area (runway)
  \item Configure aircraft: landing checklist (no flaps), extend gear, final
    approach (C172RG: 15'' Hg, 2700 RPM, 70 KIAS)
  \item Select outside references (touchdown point)
  \item Throttle to idle, bank in the direction of any crosswind with
    sufficient bank to give a steep descent based on the altitude necessary to
    lose while simultaneously applying sufficient opposite rudder to maintain
    the original flightpath
  \item Maintain attitude (airspeed indicator not accurate)
  \item Recover when a normal landing can be made: level the wings while
    releasing rudder
  \item Pitch to normal glide attitude
    % TODO(sjr): update guide for C182T, which has no limitations regarding
    % slips with flaps
  \item Land at touchdown point (without use of flaps) beyond and within 400’
    at approximate stalling speed with no side drift
\end{itemize}

\subsection{Common Errors}

\begin{itemize}
  \item Recovering by abruptly releasing rudder pressure, causing the nose to
    swing to quickly and causing excessive airspeed
  \item Failing to control pitch resulting in excessive or insufficient airspeed
  \item Side slipping, resulting in drifting away from the ground track to the
    centerline of the runway
\end{itemize}

\subsection{References}

FAA-H-8083-3A Airplane Flying Handbook p. 8-10.

