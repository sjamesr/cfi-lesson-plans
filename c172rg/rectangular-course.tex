\section{Rectangular Course}

\subsection{Description}

A training maneuver in which the ground track of the airplane is equidistant
from all sides of a selected rectangular area on the ground.

\subsection{Objective}

To teach the conditions encountered in an airport traffic pattern.

\subsection{Elements}

\begin{itemize}
  \item Clear the area
  \item Choose forced landing area (ideally within rectangular pattern) \item
    Configure aircraft for maneuvering: flaps and gear up, traffic pattern
    power and speed (C172RG: 18'' Hg, 2500 RPM, 90 KIAS), approximately traffic
    pattern altitude of 600 - 1000' AGL
  \item Select outside references (ideally a large rectangular field or parking
    lot), deciding on either a left or right turns course 
  \item Enter the pattern on the downwind, usually at 45$^\circ$ to the
    direction of the downwind, maintaining airspeed and ball centered
    \begin{itemize}
      \item It can be entered on any leg of the course, but ideally enter on
        downwind
    \end{itemize}
  \item At the first corner, turn to base leg, which is more than 90$^\circ$
    due to a necessary crab angle on the base leg; the bank will be steeper
    than normal due to the tailwind at start of the turn
  \item Crab as necessary to maintain a straight base leg
  \item At the second corner, turn to upwind leg, which is less than 90$^\circ$
    due to the crab or wind correction on base; the bank will be shallower than
    normal due to headwind at end of turn
  \item Fly the upwind, which should require no wind correction
  \item At the third corner, turn to crosswind leg, which is less than
    90$^\circ$ due to a necessary crab angle on the crosswind leg; the bank
    will be shallower than normal due to the headwind at start of the turn
  \item Crab as necessary to maintain a straight crosswind leg
  \item At the fourth corner, turn to downwind leg, which is more than
    90$^\circ$ due to the crab or wind correction on crosswind; the bank will
    be steeper than normal due to tailwind at end of turn
  \item Complete additional circuits or exit from downwind by turning
    45$^\circ$ from the downwind leg
  \item Maintain ball centered
  \item Look for traffic
\end{itemize}

\subsection{Common Errors}

\begin{itemize}
  \item Failure to adequately clear the area
  \item Failure to establish proper altitude prior to entry (typically entering
    the maneuver while descending)
  \item Failure to establish appropriate wind correction angle resulting in
    drift
  \item Gaining or losing altitude
  \item Poor coordination (typically skidding in turns from a downwind heading
    and slipping in turns from an upwind heading)
  \item Abrupt control usage
  \item Inability to adequately divide attention between airplane control and
    maintaining ground track
  \item Improper timing in beginning and recovering from turns
  \item Inadequate visual lookout for other aircraft
\end{itemize}

\subsection{References}

FAA-H-8083-3A Airplane Flying Handbook p. 6-4.

