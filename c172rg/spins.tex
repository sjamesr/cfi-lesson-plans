\section{Spins}

\subsection{Description}

An aggravated stall that results in what is termed ``autorotation'' wherein the
airplane follows a downward corkscrew path.

\subsection{Objective}

To teach stall and spin aerodynamics, recognition, and recovery.

\subsection{Setup}

\begin{itemize}
  \item Pre-flight: weight and balance of the aircraft must be calculated
    before the maneuver to confirm that the airplane is within the utility
    category
  \item Clear the area
  \item Choose forced landing area (should be runway)
  \item Configure aircraft for landing (except gear up): (C172R: 1500 RPM),
    altitude sufficient for recovery not below 1500' AGL (ideally 5000' AGL to
    start), flaps full, carburetor heat on
  \item Select outside references for orientation
  \item Power-off stall: reduce power to idle, simultaneously raise the nose
  \item Add full rudder in direction of desired spin as airplane stalls
  \item Apply full back pressure on elevator to the limit of travel
  \item Ailerons in the neutral position
  \item Take flaps out immediately 
  \item Allow spin to develop into a steady-state (developed) spin
\end{itemize}

\subsection{Recovery}

PARE is the recovery technique:

\begin{itemize}
  \item Power -- reduce to idle
  \item Ailerons -- position to neutral
  \item Rudder -- full opposite against the rotation
  \item Elevator -- brisk elevator control full forward to brake stall
  \item After spin rotation stops, neutralize the rudder 
  \item Smoothly apply back-elevator pressure to raise the nose to level
    flight, apply full power and climb to recover altitude
\end{itemize}

Spins are an aerobatic maneuver. 14 CFR 91.303 governs where aerobatic
maneuvers may be done: Uncongested area, not in class B, C, D, or E airspace
near an airport or airway. 3+ statute mile visibility is required. A parachute
is ordinarily required, but not for spins that are done when required for a
rating (14 CFR 91.307(d)). The airplane must be in the utility category or
aerobatic category to withstand the loads imposed during a spin. Load factors
for aircraft categories: Utility +4.4g, -1.76g; Normal +3.8g, -1.52g; Aerobatic
+6.0g, -3.0g.

\subsection{Common Errors}

\begin{itemize}
  \item Failure to establish proper configuration prior to spin entry
  \item Failure to achieve and maintain a full stall during spin entry
  \item Failure to close throttle when a spin entry is achieved
  \item Failure to recognize the indications of an imminent, unintentional spin
  \item Improper use of flight controls during spin entry, rotation, or
    recovery
  \item Disorientation during a spin
  \item Failure to distinguish between a high-speed spiral and a spin
  \item Excessive speed or accelerated stall during recovery
  \item Failure to recover with minimum loss of altitude
  \item Attempting to spin an airplane not approved for spins
\end{itemize}

\subsection{References}

FAA-H-8083-3A Airplane Flying Handbook p. 4-12.
FAA-S-8081-6CS Flight Instructor for Airplane Single-Engine Land and Sea PTS p.
1-56.

