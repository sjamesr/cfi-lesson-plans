\section{Power-Off Stalls}

\subsection{Description}

A rapid degeneration of lift as a result of excessive angle of attack, entered
from the landing configuration.

\subsection{Objective}

To teach recognition and recovery from a full stall under landing conditions
and required recovery action.

\subsection{Setup}

\begin{itemize}
  \item Clear the area
  \item Choose forced landing area
  \item Configure aircraft: gear down, full flaps, carburetor heat on, power
    set for final approach to land (C172RG: 18'' Hg, 2700 RPM), altitude so
    recovery is $\geq 1500$' AGL
  \item Select outside references
  \item Start a descent (as if on final approach) using power and pitch
    (C172RG: 65 KIAS) 
  \item Power to idle
  \item Smoothly raise nose until a stall is induced, maintaining constant
    pitch with the elevator
  \item Maintain coordination (ball centered) and neutral ailerons
\end{itemize}

\subsection{Recovery}

\begin{itemize}
  \item Reduce the angle of attack by releasing back-elevator pressure
  \item Advancing the throttle to maximum power
  \item Carburetor heat off, retract first notch of flaps immediately
  \item Anticipate left-turning tendencies with right rudder pressure
  \item Continue to lower nose to regain flying speed, slowly returning to
    level flight
  \item Upon positive rate of climb, retract flaps and gear are as necessary 
  \item When in level flight, reduce power to a setting for cruise flight or
    climb as necessary
  \item Maintain ball centered
  \item Look for traffic
\end{itemize}

Practice both straight-and-level and turning stalls (up to 30º). Note buffeting
and stall horn as indicators.

\subsection{Common Errors}

\begin{itemize}
  \item Failure to adequately clear the area
  \item Inability to recognize an approaching stall condition through feel for
    the airplane
  \item Premature recovery
  \item Over-reliance on the airspeed indicator while excluding other cues
  \item Inadequate scanning resulting in an unintentional wing-low condition
    during entry
  \item Excessive back-elevator pressure resulting in an exaggerated nose-up
    attitude during entry
  \item Inadequate rudder control
  \item Inadvertent secondary stall during recovery
  \item Failure to maintain a constant bank angle during turning stalls
  \item Excessive forward-elevator pressure during recovery resulting in
    negative load on the wings
  \item Excessive airspeed buildup during recovery
  \item Failure to take timely action to prevent a full stall during the
    conduct of imminent stalls
\end{itemize}

\subsection{References}

FAA-H-8083-3A Airplane Flying Handbook p. 4-7.
