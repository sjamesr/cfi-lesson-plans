\section{Chandelles}

\subsection{Description}

A maximum performance 180$^\circ$ climbing turn.

\subsection{Objective}

To teach planning, orientation, division of attention and control feel for
maximum performance flight.

\subsection{Elements}

\begin{itemize}
  \item Clear the area
  \item Choose forced landing area
  \item Configure aircraft for maneuvering: $\leq V_A$ with propeller to full
    (C172RG: 18'' Hg, 2500 RPM, 106 KIAS at MGW), gear up, flaps up, cowl flaps
    open, altitude $\geq 1500'$ AGL
  \item Select outside references (off wingtip for 90$^\circ$ point) 
  \item Establish a coordinated 30$^\circ$ bank turn
  \item Throttle to full (C172RG: 25'' Hg, 2500 RPM)
  \item Apply pitch to arrive at 90$^\circ$ reference point with max desired
    pitch
  \item Anticipate need for rudder pressure
  \item Maintain pitch attitude once at 90$^\circ$ with increasing back elevator
    pressure while reducing bank to 180$^\circ$ point
    \begin{itemize}
      \item As aircraft slows, greater back elevator is required to maintain
        pitch
    \end{itemize}
  \item At 180$^\circ$ point, roll wings level, (90$^\circ$ point is now off
    opposite wing from the start), at minimum controllable airspeed
    \begin{itemize}
      \item Rolling out of a left chandelle requires more right rudder pressure
      \item Rolling out of a right chandelle requires little rudder pressure,
        but will require right rudder pressure to maintain heading upon
        completion of rollout
    \end{itemize}
  \item Gently reduce pitch to straight-and-level but maintain altitude and
    build airspeed
  \item Adjust throttle to maintain altitude
  \item Maintain ball centered
  \item Look for traffic
\end{itemize}

Easy way to remember the maneuver's basics: ``Bank and Yank'' Can do this
maneuver at cruise (C172RG: 18'' Hg, 2300 RPM) for constant-speed propellers,
but should use maximum RPM for fixed-pitch propellers (watch to not go into red
line on tachometer).

\subsection{Common Errors}

\begin{itemize}
  \item Failure to adequately clear the area
  \item Too shallow an initial bank, resulting in a stall
  \item Too steep an initial bank, resulting in failure to gain maximum
    performance
  \item Allowing the actual bank to increase after establishing initial bank
    angle
  \item Failure to start the recovery at the 90$^\circ$ point in the turn
  \item Allowing the pitch attitude to increase as the bank is rolled out
    during the second 90$^\circ$ of turn
  \item Removing all of the bank before the 180$^\circ$ point is reached
  \item Allowing the aircraft to sink during recovery
  \item Control roughness
  \item Poor coordination (slipping or skidding)
  \item Stalling at any point during the maneuver
  \item Execution of a steep turn instead of a climbing maneuver
  \item Failure to scan for other aircraft
  \item Attempting to perform the maneuver by instrument reference rather than
    visual reference
\end{itemize}

\subsection{References}

FAA-H-8083-3A Airplane Flying Handbook p. 9-4.
