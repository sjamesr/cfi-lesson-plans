\section{Emergency Approaches and Landings}

\subsection{Description}

Simulated engine failure power-off approach and landing.

\subsection{Objective}

To teach planning, orientation, division of attention, control feel and
emergency procedures for a power-off approach and landing.

\subsection{Elements}

\begin{itemize}
  \item Clear the area
  \item Choose forced landing area (your outside reference): choose a field and
    stick with it; best choice may be behind you
  \item Configure aircraft: no flaps, gear extended, carburetor heat on,
    throttle to idle (C172RG: Propeller to 2700 RPM), establish and maintain
    best glide speed or 1.4 $V_{S_0}$ (C172RG: 75 KIAS 0$^\circ$ Flaps)
  \item Engine out checklist: carburetor heat on, fuel selector both, mixture
    rich, auxiliary fuel pump on (if pressure $<$ 0.5 psi), primer in and locked,
    cowl flaps closed
  \item Simulated: squawk 7700, mayday on 121.5
  \item Clear engine using brief applications of power
  \item Forced landing checklist (simulated): mixture cutoff, fuel valve off,
    gear down, ignition off, unlatch doors, master off when landing assured
  \item Maneuver as necessary to reach key point on base
  \item When landing assured, gear up if terrain is rough or soft else gear
    down
  \item Flaps and slip as necessary to reach aiming point
  \item Maintain ball centered 
  \item Look for traffic
\end{itemize}

\subsection{Common Errors}

\begin{itemize}
  \item Failure to maintain a safe airspeed
  \item Failure to establish best glide speed
  \item Attempting to ``stretch'' the glide
  \item Failure to identify an appropriate landing area, including wind
    considerations
  \item Failing to configure the aircraft correctly (e.g. forgetting to add
    flaps)
  \item Inappropriate energy management, inability to reach the desired landing
    point
  \item Failure to use the checklist
\end{itemize}

\subsection{References}

FAA-H-8083-3A Airplane Flying Handbook p. 8-21.

