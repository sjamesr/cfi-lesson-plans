\section{Accelerated Stalls}

\subsection{Description}

A stall resulting from a higher-than-normal airspeed in steep turns, pull-ups,
or other abrupt changes in your flight attitude, often occurring faster than
other stalls.

\subsection{Objective}

To demonstrate that a stall will not always occur at the same airspeed if
excessive maneuvering loads are imposed by steep turns, pull-ups or other
abrupt changes in flight path.

\subsection{Setup}

\begin{itemize}
  \item Clear the area
  \item Choose forced landing area
  \item Configure aircraft for a maneuvering: (C172RG: 18" Hg, 2300 RPM) at
    $\leq{}V_{A}$ (C172RG: $\leq$106 KIAS at MGW, e.g. use 90 KIAS), flaps up
    (flaps down will lead to excessive loads), gear up, carburetor heat off,
    altitude so recovery is $\geq$1500' AGL
  \item Select outside references
  \item Roll to a $45^{\circ}$ level bank while gradually increasing back
    pressure to maintain altitude
  \item Slowly increase back pressure while maintaining altitude until the
    airplane stalls
\end{itemize}

\subsection{Recovery}

\begin{itemize}
  \item Immediately release back pressure on the control and increase power
    \begin{itemize}
      \item If the turn is not coordinated, one wing may drop suddenly, causing
        the airplane to roll in that direction, if so the excessive back
        pressure must be released to break the stall before adding power
    \end{itemize}
  \item Return to straight-and-level, coordinated flight
  \item Maintain ball centered
  \item Look for traffic
\end{itemize}

It is important to recover at the first sign of the stall, as a prolonged
accelerated stall can develop into a spin. This stall tends to be more rapid or
severe than the unaccelerated stall and, because they occur at
higher-than-normal-speed, and/or may occur at lower than anticipated pitch
attitudes, they may be unexpected by an inexperienced pilot
