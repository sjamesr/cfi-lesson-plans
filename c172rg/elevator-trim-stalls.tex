\section{Elevator Trim Stalls}

\subsection{Description}

A stall resulting from application of full power during a go-around when
positive control of the airplane is not maintained.

\subsection{Objective}

To demonstrate recovery procedures for overcoming strong trim forces and how to
maintain control of the airplane by using proper and timely trim techniques.

\subsection{Setup}

\begin{itemize}
  \item Clear the area
  \item Choose forced landing area
  \item Configure aircraft for final approach for landing: CCGUMPS, approach
    power (C172RG: 15'' Hg, 2700 RPM), full flaps down, gear extended,
    carburetor heat on, altitude so recovery is $\geq$ 1500' AGL
  \item Select outside references
  \item Reduce power to idle
  \item Maintain altitude until normal glide speed is reached (C172RG: 65 KIAS)
  \item Trim nose up (full up for best effect) to simulate landing approach to
    maintain final approach speed
  \item Apply full power to simulate a go-around
    \begin{itemize}
      \item The combined forces of power, engine torque, back elevator trim
        will make the nose pitch up sharply with a left-turning tendency; as
        the pitch attitude increases to a point well above normal climb
        attitude, the potential for a stall exists
    \end{itemize}
\end{itemize}

\subsection{Recovery}

\begin{itemize}
  \item Immediately apply positive forward elevator pressure to lower nose and
    return to normal climbing attitude
  \item Trim to relieve excessive control pressure
  \item Continue normal go-around procedures and level off at the desired
    altitude 
  \item Maintain ball centered
  \item Look for traffic
\end{itemize}

It is imperative that a stall not occur during an actual go-around as there may
not be sufficient altitude to recover.

\subsection{References}
FAA-H-8083-3A Airplane Flying Handbook p. 4-11.
JS314510-001 Jeppesen Guided Flight Discovery Private Pilot Maneuvers p. 5-13.

