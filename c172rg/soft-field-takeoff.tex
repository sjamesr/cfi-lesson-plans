\section{Soft-Field Takeoff}

\subsection{Description}

Maximum performance take-off from a soft field, designed to get airborne as
quickly as possible to eliminate drag caused by tall grass, soft sand, mud or
snow.

\subsection{Objective}

To teach techniques necessary for a take-off when it is necessary to get
airborne as quickly as possible by quickly transfer weight from landing gear to
wings.

\subsection{Elements}

\begin{itemize}
  \item Clear the area
  \item Choose forced landing area
  \item Configure aircraft: flaps as specified (C172RG: 10$^\circ$), cowl flaps
    open, propeller to full
  \item Select outside references: vanishing point on runway
  \item Taxi onto runway centerline with full elevator back pressure: do not
    stop once taxiing
  \item Smoothly apply full power
  \item Anticipate need for right rudder pressure
  \item Check engine instruments (in green)
  \item As aircraft accelerates, apply enough back pressure to establish
    positive angle of attack (C172RG: pitch to put instrument glare shield on
    horizon)
  \item After lift-off, lower the nose gently with the wheels clear of the
    runway, attempting to fly in straight-and-level flight within a
    half-wingspan above the ground
  \item Accelerate in ground effect to $V_Y$ (C172RG: 84 KIAS)
  \item Gear up upon positive rate of climb, safe airspeed, no useable runway 
  \item After 500' AGL, flaps up
  \item Maintain ball centered 
  \item Look for traffic
\end{itemize}

Emphasize holding back pressure on elevator throughout taxi. On lift-off, add
gentle forward elevator pressure.

\subsection{Common Errors}

\begin{itemize}
  \item Failure to adequately clear the area
  \item Insufficient back-elevator pressure during initial takeoff roll
    resulting in inadequate angle of attack
  \item Failure to cross-check engine instruments for indications of proper
    operation after applying power
  \item Poor directional control
  \item Climbing too steeply after lift-off
  \item Abrupt and/or excessive elevator control while attempting to level off
    and accelerate after lift-off
  \item Allowing the airplane to ``mush'' or settle resulting in an inadvertent
    touchdown
  \item Attempting to climb out of ground effect area before attaining
    sufficient climb speed
  \item Failure to anticipate an increase in pitch attitude as the airplane
    climbs out of ground effect
\end{itemize}

\subsection{References}

FAA-H-8083-3A Airplane Flying Handbook p. 5-10.

