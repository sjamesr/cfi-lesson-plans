\section{Cross-control Stall}

\subsection{Description}

A stall resulting from crossed controls: aileron pressure applied in one
direction and rudder pressure in opposite direction.

\subsection{Objective}

To demonstrate the effect of improper control technique and to emphasize the
importance of using coordinated control pressures when making turns.

\subsection{Setup}

\begin{itemize}
  \item Clear the area
  \item Choose forced landing area
  \item Configure aircraft for final approach for landing: CCGUMPS, approach
    power (C172RG: 15'' Hg, 2700 RPM), flaps up (flaps down will lead to
    excessive loads), gear extended, carburetor heat on, altitude so recovery
    is $\geq$ 1500' AGL
  \item Select outside references
  \item Reduce power to idle
  \item Maintain altitude until a normal glide (C172RG: 65 KIAS) and trim to
    relieve control pressures
  \item Roll into a medium bank turn (20-30$^\circ$) once on simulated approach
  \item Apply heavy rudder pressure in the direction of the turn
  \item Apply opposite aileron pressure to maintain the bank
  \item Increase back elevator pressure to keep the nose from lowering
  \item Increase all flight control pressures until airplane stalls
\end{itemize}

\subsection{Recovery}

\begin{itemize}
  \item Immediately release all control pressures and, if necessary, allow the
    roll to continue until airplane reaches upright and level flight
  \item Increase power to full to climb and recover
  \item Maintain ball centered
  \item Look for traffic
\end{itemize}

\subsection{References}

FAA-H-8083-3A Airplane Flying Handbook p. 4-10.
JS314510-001 Jeppesen Guided Flight Discovery Private Pilot Maneuvers p. 5-12.
