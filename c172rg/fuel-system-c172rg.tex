\chapter{Fuel System (C172RG)}

\section{Objective}

To teach the components and operating procedures of the fuel system.

\section{Elements}

\begin{itemize}
  \item Components
  \item Pre-flight
  \item Normal operation
  \item Emergency operation
\end{itemize}

\section{Schedule}

Discussion (30 minutes).

\section{Equipment}

Aircraft, Pilot Operating Handbook (POH) or FAA-approved Airplane Flight Manual (AFM).

\section{Instructor Actions}

Discuss the following components:

\begin{itemize}
  \item Two vented integral fuel tanks--fuel flows by gravity from the tanks
    \begin{itemize}
      \item Standard tank capacity is 33 gallons (total 62 gal), and
        useable capacity is 24 gallons (total 44 gal)
    \end{itemize}

  \item Fuel tank vent--venting is accomplished by an interconnected line
    from the right fuel tank to the left tank, the left tank is vented
    overboard though a vent line, which protrudes from the bottom surface
    of the wing; the right fuel tank filler cap is also vented

  \item Fuel gauges--indicate the amount of fuel measured by a sensing unit
    in each tank and is displayed in gallons and pounds.

  \item Fuel sumps and drains--allow for checks at preflight to be made in
    the fuel tanks, selector, and strains, of visible moisture and/or
    sediments, as well as check for the proper grade of fuel

  \item Four-position selector valve--the selector can be set to OFF, BOTH,
    LEFT, and RIGHT; when the selector is not set to OFF, fuel is able to
    flow through to the rest of the system

  \item Fuel strainer--(inside oil compartment) removes any impurities,
    including moisture and other sediments that might be present in the
    fuel

  \item Manual primer--takes fuel directly from the strainer and vaporizes
    it directly into three of the cylinders

  \item Fuel pressure gauge--shows the fuel flow in PSI and can be used to
    indicate a failure in of the fuel pump

  \item Engine-driven fuel pump--driven by the engine to pump fuel to the
    carburetor

  \item Electric auxiliary fuel pump--electrically drives fuel to the
    carburetor and should be used when the fuel flow drops below 0.5 PSI
\end{itemize}

Discuss fuel grades: Aviation gasoline (AVGAS) is identified by an octane or
performance number (grade). The higher the grade of gasoline, the more pressure
the fuel can withstand without detonating. If the proper grade of fuel is not
available, use the next higher grade as a substitute (but not JET A). Never use
a lower grade. This can cause the cylinder head temperature and engine oil
temperature to exceed their normal operating range, which may result in
detonation. Available AVGAS is 80 (dyed red), 100 (dyed green), and 100LL (dyed
blue). The C172RG used 100LL.

Discuss pre-ignition and detonation.

Discuss refueling, including grounding, use of a ladder, etc. Note that if
refueling before flight, should redo sumping after the fuel has settled (at
least 10 minutes).

Discuss preflight of the fuel system per POH.

\section{Evaluation}

Lesson is complete when student can demonstrate and discuss proper use of fuel
system.

\section{References}

FAA-H-8083-25 Pilot's Handbook of Aeronautical Knowledge p. 5-13, POH / AFM p.
7-23.
