\section{Straight-and-Level Flight}

\subsection{Description}

The most fundamental maneuver whereby the airplane maintains a constant heading
and altitude.

\subsection{Objective}

To develop the fundamental techniques required for basic flight.

\subsection{Elements}

\begin{itemize}
  \item Clear the area
  \item Choose forced landing area (always be aware of options) 
  \item Configure aircraft for cruise (C172RG: 23'' Hg, 2300 RPM)
  \item Select outside references (point on the horizon corresponding to
    heading)
  \item Periodically insure the nose is fixed below the horizon
  \item Periodically insure the wingtips are equally above or below the horizon
    (level)
  \item Compensate for cross wind to maintain desired ground track
  \item Trim as necessary to maintain altitude
  \item Maintain ball centered
  \item Look for traffic
\end{itemize}

\subsection{Common Errors}

\begin{itemize}
  \item Attempting to use improper reference points on the airplane to
    establish attitude
  \item Forgetting the location of preselected reference points on subsequent
    flights
  \item Attempting to establish or correct airplane attitude using flight
    instruments rather than outside visual reference
  \item Attempting to maintain direction using only rudder control
  \item Habitually flying with one wing low.
  \item ``Chasing'' the flight instruments rather than adhering to the
    principles of attitude flying
  \item Too tight a grip on the flight controls resulting in over-control and
    lack of feel
  \item Pushing or pulling on the flight controls rather than exerting pressure
    against the airstream
  \item Improper scanning and/or devoting insufficient time to outside visual
    reference (head in the cockpit)
  \item Fixation on the nose (pitch attitude) reference point
  \item Unnecessary or inappropriate control inputs
  \item Failure to make timely and measured control inputs when deviations from
    straight-and-level flight are detected
  \item Inadequate attention to sensory inputs in developing feel for the
    airplane
\end{itemize}

\subsection{References}

FAA-H-8083-3A Airplane Flying Handbook p. 3-4.

