\section{Unusual Attitudes}

\subsection{Description}

Use of instruments solely as a mean to recover from steep climbing or
descending turns, necessary for inadvertent entry into IMC.

\subsection{Objective}

To teach proper techniques required to return the airplane to straight and
level flight.

\subsection{Elements}

\begin{itemize}
  \item Clear the area
  \item Instructor: choose forced landing area
  \item Configure aircraft for cruise or as necessary (C172RG: 23'' Hg, 2300
    RPM)
  \item Student: wear a view-limiting device
  \item Instructor: put aircraft into an unusual attitude
  \item Student: Recover from unusual attitude using only instruments:
    \begin{itemize}
      \item Nose low: reduce power to idle, wings level, pitch the nose to
        level flight, then readjust for cruise flight
      \item Nose high: increase power to full, apply forward elevator pressure
        to prevent stall, wings level, then readjust for cruise flight
    \end{itemize}
  \item Maintain ball centered throughout recovery
  \item Instructor looks for traffic while student performs under a hood
\end{itemize}

\subsection{Common Errors}

\begin{itemize}
  \item Slow cross-check, fixation on or omission of instruments
  \item Attempting recovery by sensory information versus instruments
  \item Failure to practice basic instrument skills
\end{itemize}

\subsection{References}

FAA-H-8083-15A Instrument Flying Handbook p. 5-26.

