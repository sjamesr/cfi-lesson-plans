\chapter{Emergency Procedures}

\section{Objective}

To familiarize the student with emergency considerations for flight.

\section{Elements}

\begin{itemize}
  \item Airplane emergency scenarios
  \item ELT
  \item Transponder
  \item 121.5 MHz
  \item Survival supplies and considerations
  \item Night flight considerations
\end{itemize}

\section{Schedule}

Discussion (30 minutes).

\section{Equipment}

Airplane’s Aircraft Flight Manual or Pilot Operating Handbook.

\section{Instructor Actions}

Discuss emergency procedures listed in the AFM/POH. Discuss various scenarios.

Discuss the ELT, its required checks (12 calendar months, 50\% of life, or 1
hour of cumulative use, per 14 CFR 91.207). Transmits on 121.5 and 243.0 MHz.
In the event of an emergency, stay with the ELT. Bring it with you if you have
to leave a crash site. The antenna is flexible on most airplanes to survive the
crash. The ELT should be removable. Flying on a flight plan dramatically
increase odds of being found quickly.

Discuss use of transponder frequencies: 7500 Hijack, 7600 Lost Communications,
7700 Emergency. Avoid other 7000-series squawk settings.

Emergency radio frequency is 121.5 MHz (and 243.0 MHz for military radios).

Consider carrying food and water and survival supplies depending on the terrain
flown, the season, the availability of services along the route, etc. Examples
include coats or blankets, flares, flotation devices if over water, etc. A
first aid kit is also a good idea. A mirror or reflector (or a CD), a knife,
and a Leatherman are also good ideas.

For night flights, always have multiple sources of light (flashlights, etc).

\section{Evaluation}


Lesson is complete when student can demonstrate and comprehend emergency
considerations.

\section{References}

Airplane’s Aircraft Flight Manual or Pilot Operating Handbook.
