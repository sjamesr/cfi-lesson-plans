\documentclass[twoside,openright]{report}

\usepackage{cclicenses}
\usepackage{emptypage}
\usepackage{fancyhdr}
\usepackage{textcomp}
\usepackage{hyperref}
\pagestyle{fancy}

\fancyfoot[C]{This document is provided for informational use only. It is for
use by authorized instructors. Consult the aircraft manual and appropriate FAA
handbooks to double-check all information. \textcopyright{} Derek W Beck 2008.
Some Rights Reserved. \url{http://www.derekbeck.com/}.
\\
\byncsa{} \href{https://creativecommons.org/licenses/by-nc-sa/3.0/us/}{Licensed
under Creative Commons Attribution-Noncommercial-Share Alike License}.
}

\fancypagestyle{plain}

\title{Certificated Flight Instructor – Airplane (CFI-A)
Private and Commercial Pilot Lesson Plans}
\author{Derek W Beck}
\date{June 2008}

\begin{document}

\makeatletter
\begin{titlepage}
  \begin{center}
    \vspace*{1cm}
    \textbf{\@title}

    \vspace*{1cm}
    \@author

    \@date

    \vspace*{2cm}
    \ccby You may copy, distribute, display this copyrighted work — and
    derivative works based upon it — but only if they give credit to Derek W
    Beck and abide by the other license requirements listed here.

    \ccnc This work and its derivatives may not be sold without permission from
    Derek W Beck.

    \ccsa You may distribute derivative works only under a license identical to
    the licenses listed here and only if these licenses are explicitly depicted
    somewhere on the derivative work.
  \end{center}
\end{titlepage}

\setcounter{tocdepth}{0}
\tableofcontents

\chapter{Private Pilot Introduction}

\section{Objective}

Familiarize the student with the privileges, obligations and responsibilities
of a private pilot.

\section{Elements}

\begin{itemize}
  \item Typical VFR flight
  \item Privileges
  \item Training requirements
  \item Currency requirements
\end{itemize}

\section{Schedule}

Discussion (30 minutes)

\section{Instructor Actions}

Explain objectives.

\section{Evaluation}

Ensure understanding of private pilot flying, its objectives, etc.

\chapter{Piloting considerations}

\section{Objective}

To familiarize the student with currency requirements, health requirements,
medical requirements, etc.

\section{Elements}

\begin{itemize}
  \item Health ("I'M SAFE")
  \item Medical requirements
  \item Aeromedical physiology
  \item Currency requirements
  \item Log books
\end{itemize}

\section{Schedule}

Discussion (30 minutes)

\section{Equipment}

14 CFR (FAR/AIM)

\section{Instructor Actions}

Discuss the following regulations and requirements:

\begin{itemize}
  \item Health considerations: "I'M SAFE": Illness (14 CFR 61.53, 91.17),
    Medication (14 CFR 91.17; the best list is at AOPA's members section or
    \url{http://www.leftseat.com/medcat1.htm}; FAA has no official list),
    Stress, Alcohol (14 CFR 91.17; none within 8 hrs, < 0.04\% BAC), Fatigue,
    Emotion.

  \item Medical requirements: (14 CFR 61.23) 3rd class medical lasts 36
    calendar months (< age 40) or 24 calendar months (> age 40). A current
    medical is required to exercise the privileges allowed by the pilot
    certificates held.

  \item Aeromedical physiology: (chapter 8 of the AIM) hypoxia, ear block,
    sinus block, decompression sickness, hyperventilation, carbon monoxide
    poisoning and disorientation.

  \item Currency requirements: (14 CFR 61.57) 3 take-offs and landings every 90
    days for daytime, 3 full-stop take-offs and landings every 90 days for
    nighttime to fly with passengers. Also, a biennial flight review (14 CFR
    61.56) consisting of 1 hour of ground that at least covers 14 CFR 91 and 1
    hour of flight maneuvers as deemed adequate by the instructor. The pilot
    certificate (other than student certificate) lasts indefinitely (14 CFR
    61.19). Moving requires an update sent to the FAA within 30 days (14 CFR
    61.60).

  \item Log book requirements: (14 CFR 61.51) Only requirement is training and
    experience required to obtain certificates, ratings, a flight review, or
    currency requirements. Expand on how to log time and what to log.

\end{itemize}

\section{Evaluation}

Ensure understanding of piloting considerations, currency requirements, etc.

\section{References}

AIM Chapter 8, 14 CFR 61 and 91.

\chapter{Pre-flight}

\section{Objective}

To teach the student with the airplane and the necessary steps for pre-flight.

\section{Elements}

\begin{itemize}
  \item Checklists
  \item Pre-flight procedures
\end{itemize}

\section{Schedule}

Instructor demonstration (45 minutes)

\section{Equipment}

Aircraft, airplane Pilot Operating Handbook or FAA-approved Airplane Flight
Manual.

\section{Instructor Actions}

\begin{itemize}
  \item Discuss the POH/AFM. Discuss the importance of checklists.

  \item Discuss required documents (AROW).

  \item Discuss the need for all the steps as outlined in the POH. Discuss the
    instrument tolerances inside the cockpit. Discuss the inspection of the
    wing and control surfaces. Discuss fuel and oil, grades, types,
    contaminants, etc. Discuss landing gear, tires, brakes, etc. Discuss the
    engine and propeller.

  \item Discuss engine run-up.

  \item Discuss other items of operation in the POH/AFM.
\end{itemize}

\section{Student Actions}

Demonstrate a pre-flight inspection.

\section{Evaluation}

Lesson is complete when student can demonstrate and explain the need for each
procedure and checklist item that is listed as part of the pre-flight of the
aircraft.

\section{References}

Pilot Operating Handbook or FAA-approved Airplane Flight Manual, FAA-H-8083-3A
Airplane Flying Handbook Chapter 2.

\end{document}
