\documentclass[twoside,openright]{report}

\usepackage[margin=1in]{geometry}
\usepackage{cclicenses}
\usepackage{emptypage}
\usepackage{fancyhdr}
\usepackage{textcomp}
\usepackage{hyperref}
\pagestyle{fancy}

\fancyfoot[C]{This document is provided for informational use only. It is for
use by authorized instructors. Consult the aircraft manual and appropriate FAA
handbooks to double-check all information. \textcopyright{} Derek W Beck 2008.
Some Rights Reserved. \url{http://www.derekbeck.com/}.
\\
\byncsa{} \href{https://creativecommons.org/licenses/by-nc-sa/3.0/us/}{Licensed
under Creative Commons Attribution-Noncommercial-Share Alike License}.
}

\fancypagestyle{plain}

\title{Certificated Flight Instructor – Airplane (CFI-A)
Private and Commercial Pilot Lesson Plans}
\author{Derek W Beck}
\date{June 2008}

\begin{document}

\makeatletter
\begin{titlepage}
  \begin{center}
    \vspace*{1cm}
    \textbf{\@title}

    \vspace*{1cm}
    \@author

    \@date

    \vspace*{2cm}
    \ccby You may copy, distribute, display this copyrighted work -- and
    derivative works based upon it -- but only if they give credit to Derek W
    Beck and abide by the other license requirements listed here.

    \ccnc This work and its derivatives may not be sold without permission from
    Derek W Beck.

    \ccsa You may distribute derivative works only under a license identical to
    the licenses listed here and only if these licenses are explicitly depicted
    somewhere on the derivative work.
  \end{center}
\end{titlepage}

\setcounter{tocdepth}{0}
\tableofcontents

\chapter{Private Pilot Introduction}

\section{Objective}

Familiarize the student with the privileges, obligations and responsibilities
of a private pilot.

\section{Elements}

\begin{itemize}
  \item Typical VFR flight
  \item Privileges
  \item Training requirements
  \item Currency requirements
\end{itemize}

\section{Schedule}

Discussion (30 minutes).

\section{Instructor Actions}

Explain objectives.

\section{Evaluation}

Ensure understanding of private pilot flying, its objectives, etc.

\chapter{Piloting considerations}

\section{Objective}

To familiarize the student with currency requirements, health requirements,
medical requirements, etc.

\section{Elements}

\begin{itemize}
  \item Health (``I'M SAFE'')
  \item Medical requirements
  \item Aeromedical physiology
  \item Currency requirements
  \item Log books
\end{itemize}

\section{Schedule}

Discussion (30 minutes).

\section{Equipment}

14 CFR (FAR/AIM).

\section{Instructor Actions}

Discuss the following regulations and requirements:

\begin{itemize}
  \item Health considerations: ``I'M SAFE'': Illness (14 CFR 61.53, 91.17),
    Medication (14 CFR 91.17; the best list is at AOPA's members section or
    \url{http://www.leftseat.com/medcat1.htm}; FAA has no official list),
    Stress, Alcohol (14 CFR 91.17; none within 8 hrs, < 0.04\% BAC), Fatigue,
    Emotion.

  \item Medical requirements: (14 CFR 61.23) 3rd class medical lasts 36
    calendar months (< age 40) or 24 calendar months (> age 40). A current
    medical is required to exercise the privileges allowed by the pilot
    certificates held.

  \item Aeromedical physiology: (chapter 8 of the AIM) hypoxia, ear block,
    sinus block, decompression sickness, hyperventilation, carbon monoxide
    poisoning and disorientation.

  \item Currency requirements: (14 CFR 61.57) 3 take-offs and landings every 90
    days for daytime, 3 full-stop take-offs and landings every 90 days for
    nighttime to fly with passengers. Also, a biennial flight review (14 CFR
    61.56) consisting of 1 hour of ground that at least covers 14 CFR 91 and 1
    hour of flight maneuvers as deemed adequate by the instructor. The pilot
    certificate (other than student certificate) lasts indefinitely (14 CFR
    61.19). Moving requires an update sent to the FAA within 30 days (14 CFR
    61.60).

  \item Log book requirements: (14 CFR 61.51) Only requirement is training and
    experience required to obtain certificates, ratings, a flight review, or
    currency requirements. Expand on how to log time and what to log.

\end{itemize}

\section{Evaluation}

Ensure understanding of piloting considerations, currency requirements, etc.

\section{References}

AIM Chapter 8, 14 CFR 61 and 91.

\chapter{Pre-flight}

\section{Objective}

To teach the student with the airplane and the necessary steps for pre-flight.

\section{Elements}

\begin{itemize}
  \item Checklists
  \item Pre-flight procedures
\end{itemize}

\section{Schedule}

Instructor demonstration (45 minutes)

\section{Equipment}

Aircraft, airplane Pilot Operating Handbook or FAA-approved Airplane Flight
Manual.

\section{Instructor Actions}

\begin{itemize}
  \item Discuss the POH/AFM. Discuss the importance of checklists.

  \item Discuss required documents (AROW).

  \item Discuss the need for all the steps as outlined in the POH. Discuss the
    instrument tolerances inside the cockpit. Discuss the inspection of the
    wing and control surfaces. Discuss fuel and oil, grades, types,
    contaminants, etc. Discuss landing gear, tires, brakes, etc. Discuss the
    engine and propeller.

  \item Discuss engine run-up.

  \item Discuss other items of operation in the POH/AFM.
\end{itemize}

\section{Student Actions}

Demonstrate a pre-flight inspection.

\section{Evaluation}

Lesson is complete when student can demonstrate and explain the need for each
procedure and checklist item that is listed as part of the pre-flight of the
aircraft.

\section{References}

Pilot Operating Handbook or FAA-approved Airplane Flight Manual, FAA-H-8083-3A
Airplane Flying Handbook Chapter 2.

\chapter{Aerodynamics}

\section{Objective}

To teach basic aerodynamic principles.

\section{Elements}

\begin{itemize}
  \item Four forces
  \item Airfoils
  \item Drag
  \item Stability and controllability
  \item Turning tendencies
  \item Climbs, descents, and turns
  \item Load factors
  \item Ground effect
  \item Adverse yaw
  \item Wingtip vortices
\end{itemize}

\section{Schedule}

Discussion (1 hour).

\section{Equipment}

Model aircraft.

\section{Instructor Actions}

Discuss the following:

\begin{itemize}
  \item Aerodynamics is the branch of physics that deals with the motion of a
    solid body through fluids.

  \item Four forces: lift, weight, thrust, drag

  \item Newton's laws of motion: first: an object in motion stays in motion,
    second: force equals mass times acceleration, third: for every action there
    is an equal and opposite reaction.

  \item In steady flight, the sum of these opposing forces is equal to zero.
    (Newton's first law) Lift = weight, thrust = drag.

  \item Bernoulli's principle: an increase in the speed of the fluid occurs
    simultaneously with a decrease in pressure.

  \item An airfoil is a structure designed to obtain reaction upon its surface
    from the air through which it moves or that moves past it. (Bernoulli)
\end{itemize}

Discuss the wing definitions: leading edge, trailing edge, camber, chord line.

\begin{itemize}
  \item Aspect ratio--wingspan to mean chord line
    \begin{itemize}
      \item High aspect ratio = more lift, less drag.
    \end{itemize}

  \item Angle of incidence--angle between the chord and the longitudinal axis

  \item Angle of attack (AoA)

  \item Wing always stalls at the critical angle of attack
\end{itemize}

Drag:

\begin{itemize}
  \item Parasite: increases as the square of the airspeed
    \begin{itemize}
      \item Form drag: shape of the aircraft, i.e. streamlined object

      \item Skin friction: surface finish

      \item Interference: if two objects are placed adjacent to one another, the
        resulting turbulence produced may be 50 to 200\% greater than the parts
        tested separately
    \end{itemize}
  \item Induced-byproduct of lift (horizontal component of lift), varies
    inversely as the square of the airspeed, also caused by downwash from
    wingtip vortices
\end{itemize}

Discuss the $L/D_{max}$ drag graph. (See figure 3-5 in PHAK)

Discuss stability (see PHAK 3-10) (Use marble and bowl analogy):

\begin{itemize}
  \item Stability is the inherent quality of an airplane to correct for
    conditions that may disturb its equilibrium, and to return or to continue
    on the original flightpath
  \item Static:
    \begin{itemize}
      \item Static stability is the initial tendency of an object to return to
        its original position after being disturbed

      \item Positive, neutral, negative

      \item Push controls: positive returns to original state; neutral: remains
        at new state; negative: keeps moving beyond
    \end{itemize}
  \item Dynamic:
    \begin{itemize}
      \item Tendency after equilibrium is disturbed (stability in motion)

      \item Positive, neutral, negative

      \item Positive: dampened, pattern of movements become smaller; neutral:
        pattern continues unchanged; negative: pattern diverges
    \end{itemize}
\end{itemize}

Discuss axes of rotation: longitudinal (roll/ailerons), lateral
(pitch/elevator), vertical (yaw/rudder).

Longitudinal stability (pitch):

\begin{itemize}
  \item Most affected by pilot, especially aircraft loading

  \item Draw CL, CG, horizontal stabilizer down force

  \item Horizontal stabilizer is an upside-down wing, provides longitudinal
    stability with a down force

  \item Nose moves up (tail down), horizontal stabilizer's AoA decreases,
    reduces lift and the nose moves back down

  \item Nose moves down (tail up), horizontal stabilizer's AoA increases,
    produces more lift and the pushes the nose moves back up

  \item Unstable if CG moves too close to or behind the center of lift
\end{itemize}

Lateral stability (roll)

\begin{itemize}
  \item Most common design factor for positive stability is wing dihedral (see
    PHAK Fig. 3-17)

  \item Positive dihedral is when the wing tips are higher than the wing root

  \item Wing position--high wing aircraft are more laterally stable than low
    wings (everything held constant)

  \item High wing aircraft require less dihedral than low wing aircraft
\end{itemize}

Vertical stability (yaw)

\begin{itemize}
  \item Vertical stabilizer, and the fuselage behind the CG to a lesser extent,
    provide directional stability

  \item Vertical stabilizer acts like the tail feathers on an arrow

  \item Swept back wings, and double taper wings to a lesser extent, also
    provide directional stability

  \item Increased induced drag on the wing that is moved forward pulls it back,
    causes Dutch roll tendency
\end{itemize}

Controllability versus Maneuverability (PHAK 3-10)

\begin{itemize}
  \item Controllability--capability of an aircraft to respond to the pilot’s
    control inputs, especially with regard to flightpath and attitude

  \item Maneuverability--quality of an airplane that allows it to be easily
    controlled and maneuvered and withstand stresses imposed by maneuvers

  \item An F-16 sacrifices stability for controllability, maneuverability
\end{itemize}

Turning tendencies (PHAK 3-23)

\begin{itemize}
  \item Torque--opposite reaction to the engine (Newton’s third law): leads to a
    left roll

  \item Spiraling slipstream--pushes on port side of vertical stabilizer,
    causing left yaw

  \item Gyroscopic action (precession)--a pitch down will cause a left yaw 90º
    from the top of the propeller, a pitch up will cause a right yaw 90º from
    the bottom of the propeller disk

  \item P-factor--downward blade has a higher angle of attack during a high
    pitch attitude, causing greater thrust on the right side of the propeller
    disk, creating a left yaw (and vice versa)
\end{itemize}

Climbs: Initially lift is greater, then it stabilizes in steady-state.

Descents: Initially lift is reduced, then it stabilizes in steady-state.

Turns: Show a vector diagram with horizontal component of lift.

Load factor: Any force applied to an airplane to deflect its flight from a
straight line. Ratio of total load acting on the airplane to the gross weight.
Discuss maneuvering speed $V_{a}$.

Limit load factors: Normal 3.8 to -1.52, Utility 4.4 to -1.76, Aerobatic 6.0 to
-3.0. 1.5 factor of safety built in. See load factor chart (PHAK Fig. 3-36),
noting $60^{\circ}$ of bank equals $2g$. Discuss $V_{g}$ diagram (PHAK Fig.
3-38).

Ground effect (PHAK 3-7):

\begin{itemize}
  \item Within one wingspan of earth’s surface, wingtip vortices are reduced

  \item Provides decrease in induced drag

  \item Wing will require a lower angle of attack in ground effect to produce
    the same lift
\end{itemize}

Adverse yaw (PHAK 4-2):

\begin{itemize}
  \item Purpose of the rudder is to counteract adverse yaw and control movement
    around the vertical axis

  \item Raised wing in a turn has a higher angle of attack, more lift, more
    induced drag

  \item Differential ailerons, Frise-type ailerons, and coupled ailerons and
    rudder reduce adverse yaw
\end{itemize}


Wingtip vortices (PHAK 3-6):

\begin{itemize}
    \item Spanwise movement of air along wing due to pressure differential

    \item Air from bottom of wing moves outward from fuselage and ``spills''
      over the wingtips, creating a vortex

    \item Air from the top of the wing flows in toward the fuselage and spills
      off the trailing edge--vortex is insignificant because fuselage limits the
      flow

    \item Vortices increase drag because of energy spent in producing turbulence

    \item High angle of attack = more violent vortices

    \item Heavier and slower aircraft = more violent vortices
\end{itemize}

Wake turbulence on take-off/landing:

\begin{itemize}
  \item Stay above glidepath

  \item Land beyond the point of landing of the preceding heavier aircraft--look
    for puffs of tire smoke

  \item Liftoff prior to the point a larger aircraft took off

  \item Light quartering tailwind keeps vortices on the runway the longest (the
    most dangerous)
\end{itemize}

\section{Evaluation}

Lesson is complete when student can demonstrate fundamental understanding of
aerodynamics.

\section{References}

FAA-H-8083-25 Pilot's Handbook of Aeronautical Knowledge (PHAK) Chapter 3,
FAA-H-8083-3A Airplane Flying Handbook Chapter 3, Aeronautical Information
Manual (AIM) 7-3-1.

\chapter{Aircraft Systems}

\section{Objective}

To teach the student the basics of aircraft systems.

\section{Elements}

\begin{itemize}
  \item Primary flight controls and trim
  \item Flaps, leading edge devices, spoilers
  \item Power plant
  \item Oil
  \item Avionics
\end{itemize}

\section{Schedule}

Discussion (45 minutes).

\section{Equipment}

Aircraft, Pilot Operating Handbook (POH) or FAA-approved Airplane Flight Manual
(AFM).

\section{Instructor Actions}

Discuss the components:

\begin{itemize}
  \item Primary flight controls--elevator, rudder, and ailerons
    \begin{itemize}
      \item Movement of the control surfaces changes the airflow and pressure
        distribution over and around the airfoil (relate to CG):
\begin{table}[h]
\centering
\begin{tabular}{l|l|l|l}
Primary control surface & Airplane movement & Axis of rotation & Type of stability \\\hline
aileron                 & roll              & longitudinal     & lateral           \\
elevator                & pitch             & lateral          & longitudinal      \\
rudder                  & yaw               & vertical         & directional
\end{tabular}
\end{table}
    \end{itemize}
  \item Ailerons:
    \begin{itemize}
      \item Control roll about longitudinal axis

      \item Most light airplanes have two ailerons, one on the trailing edge of
        each wing

      \item Connected to control wheel through cables and pulleys

      \item Move in opposite directions

      \item Discuss how ailerons change angle of attack and cause roll

      \item Adverse yaw can be counter acted with rudder use, or special
        aileron designs: differential ailerons, Frise-type ailerons, coupled
        ailerons and rudder

      \item Aileron trim: not common on light airplanes
    \end{itemize}
  \item Elevator
    \begin{itemize}
      \item Controls pitch about lateral axis

      \item Main purpose is to change the wing's angle of attack

      \item Most light airplanes have one elevator, located on the trailing
        edge of the horizontal stabilizer

      \item Some aircraft (e.g. Pipers) use a stabilator, or movable horizontal
        stabilizer

      \item Control wheel connected to the elevator by bell cranks, cables and
        pulleys

      \item Horizontal stabilizer has a negative angle of attack to provide
        downward force

      \item Elevator moves up to increase this downward push and move the nose
        up, and therefore increase the wing's angle of attack, and vice versa

      \item Discuss how elevator movement affects pitch attitude

      \item Elevator trim: almost all light aircraft are equipped with some
        form of elevator trim; moves in the opposite direction of the control
        surface, deflecting the control surface to relieve control pressure and
        maintain a constant pitch attitude
    \end{itemize}

  \item{Rudder}
    \begin{itemize}
      \item Controls the airplane about its vertical axis -- yaw

      \item Most light airplanes have one rudder, located on the trailing edge
        of the vertical stabilizer

      \item Controlled through the use of foot pedals, connected to the rudder
        by bell cranks, cables and pulleys

      \item Rudder does not turn the airplane, only yaws it

      \item Used in conjunction with the ailerons for properly turning the airplane

      \item Rudder trim: most light aircraft are equipped with some form of
        rudder trim; trim tabs move in the opposite direction of the control
        surface, deflecting the control surface to relieve control pressure and
        maintain a constant yaw attitude; some may be a ``manual rudder trim''
        or a piece of metal that is manually adjusted before flight
    \end{itemize}

  \item{Flaps}
    \begin{itemize}
      \item Increase lift and drag

      \item Flaps have three main functions: permit a slower landing speed,
        allow for a steep angle on descent without an increase in airspeed,
        shorten takeoff distance and allow for a steeper climb

      \item Plain--simplest, changes camber, increases lift, greatly increases
        drag

      \item Split--greater increase in lift vs. plain, more drag

      \item Slotted--(most common) increases lift coefficient significantly
        more than plain or split (high-energy air is ducted to the flap’s upper
        surface, delaying airflow separation)

      \item Fowler flap--a variety of slotted flap; changes camber and
        increases wing area

      \item Most flaps are located on the trailing edge of the wing in-between
        the fuselage and aileron

      \item In light aircraft they are controlled manually or electrically

      \item Extending the flaps will increase lift, cause a pitch up and loss
        of airspeed

      \item Retracting the flaps will decrease lift cause a pitch down and
        increase in airspeed
    \end{itemize}

  \item Leading edge devices:
    \begin{itemize}
      \item Fixed slots--direct airflow to upper wing surface and delay airflow
        separation; stall is delayed to greater angle of attack

      \item Moveable slats--leading edge segments on tracks; may be automatic or pilot-operated

      \item Leading edge flaps--increase coefficient of lift and camber
    \end{itemize}

  \item Spoilers:
    \begin{itemize}
      \item High-drag device; reduces lift, increases drag

      \item Used for roll control on some aircraft by eliminating adverse yaw

      \item Can shorten ground roll
    \end{itemize}

  \item Power plant:
    \begin{itemize}
        \item Reciprocating engines classified by cylinder arrangement (radial,
          inline, v-type, opposed), method of cooling (liquid or air), method
          of intake (carburetor, fuel-injection, turbo-charged), etc

        \item Main components: cylinders (contain intake/exhaust valves, spark
          plugs, pistons); crankcase (contains crankshaft, connecting rods);
          accessory housing (contains magnetos)

        \item Four-stroke operating cycle: intake, compression, power, exhaust
    \end{itemize}

  \item Oil:
    \begin{itemize}
      \item Lubricates, reduces friction, cools, provides a seal, and carries
        away contaminants

      \item Wet-sump system—sump is an integral part of the engine (in a dry
        system, it's a separate tank)

      \item Filter, cooler, filler cap/dipstick, quick-drain valve (bottom of
        sump)

      \item Pressure and temperature gauges (required instruments)
    \end{itemize}

  \item Avionics:
    \begin{itemize}
      \item Communication and navigation radios

      \item VOR, ADF, GPS

      \item Transponder

      \item Autopilot (if available)

      \item Avionics cooling fan--cools and eliminates moisture (if available)

      \item Microphone/headset intercom

      \item Static dischargers (wicks)
    \end{itemize}
\end{itemize}

\section{Evaluation}

Lesson is complete when student can demonstrate and discuss aircraft control
surfaces, power plant and other major systems.

\section{References}

FAA-H-8083-25 Pilot's Handbook of Aeronautical Knowledge Chapter 4-5, POH / AFM
Chapter 7.

\chapter{Fuel System (C172RG)}

\section{Objective}

To teach the components and operating procedures of the fuel system.

\section{Elements}

\begin{itemize}
  \item Components
  \item Pre-flight
  \item Normal operation
  \item Emergency operation
\end{itemize}

\section{Schedule}

Discussion (30 minutes).

\section{Equipment}

Aircraft, Pilot Operating Handbook (POH) or FAA-approved Airplane Flight Manual (AFM).

\section{Instructor Actions}

Discuss the following components:

\begin{itemize}
  \item Two vented integral fuel tanks--fuel flows by gravity from the tanks
    \begin{itemize}
      \item Standard tank capacity is 33 gallons (total 62 gal), and
        useable capacity is 24 gallons (total 44 gal)
    \end{itemize}

  \item Fuel tank vent--venting is accomplished by an interconnected line
    from the right fuel tank to the left tank, the left tank is vented
    overboard though a vent line, which protrudes from the bottom surface
    of the wing; the right fuel tank filler cap is also vented

  \item Fuel gauges--indicate the amount of fuel measured by a sensing unit
    in each tank and is displayed in gallons and pounds.

  \item Fuel sumps and drains--allow for checks at preflight to be made in
    the fuel tanks, selector, and strains, of visible moisture and/or
    sediments, as well as check for the proper grade of fuel

  \item Four-position selector valve--the selector can be set to OFF, BOTH,
    LEFT, and RIGHT; when the selector is not set to OFF, fuel is able to
    flow through to the rest of the system

  \item Fuel strainer--(inside oil compartment) removes any impurities,
    including moisture and other sediments that might be present in the
    fuel

  \item Manual primer--takes fuel directly from the strainer and vaporizes
    it directly into three of the cylinders

  \item Fuel pressure gauge--shows the fuel flow in PSI and can be used to
    indicate a failure in of the fuel pump

  \item Engine-driven fuel pump--driven by the engine to pump fuel to the
    carburetor

  \item Electric auxiliary fuel pump--electrically drives fuel to the
    carburetor and should be used when the fuel flow drops below 0.5 PSI
\end{itemize}

Discuss fuel grades: Aviation gasoline (AVGAS) is identified by an octane or
performance number (grade). The higher the grade of gasoline, the more pressure
the fuel can withstand without detonating. If the proper grade of fuel is not
available, use the next higher grade as a substitute (but not JET A). Never use
a lower grade. This can cause the cylinder head temperature and engine oil
temperature to exceed their normal operating range, which may result in
detonation. Available AVGAS is 80 (dyed red), 100 (dyed green), and 100LL (dyed
blue). The C172RG used 100LL.

Discuss pre-ignition and detonation.

Discuss refueling, including grounding, use of a ladder, etc. Note that if
refueling before flight, should redo sumping after the fuel has settled (at
least 10 minutes).

Discuss preflight of the fuel system per POH.

\section{Evaluation}

Lesson is complete when student can demonstrate and discuss proper use of fuel
system.

\section{References}

FAA-H-8083-25 Pilot's Handbook of Aeronautical Knowledge p. 5-13, POH / AFM p.
7-23.

\chapter{Electrical System (C172RG)}

\section{Objective}

To teach the components and operating procedures of the electrical system.

\section{Elements}

\begin{itemize}
  \item Components
  \item Pre-flight
  \item Emergency operation
\end{itemize}

\section{Schedule}

Discussion (30 minutes).

\section{Equipment}

Aircraft, Pilot Operating Handbook (POH) or FAA-approved Airplane Flight Manual (AFM).

\section{Instructor Actions}

Discuss the following components:
\begin{itemize}
  \item 28V DC System

  \item Battery -- 24V Located aft of the rear cabin wall

  \item Alternator -- 60A Belt-driven

  \item Buses: Primary and Avionics (interconnected with primary via avionics
    power switch/breaker)

  \item Master Switch -- split switch: battery and alternator

  \item Avionics Power Switch -- power from primary to avionics bus; is also a
    circuit breaker

  \item Ammeter -- indicates battery charging rate when alternator is on and
    working, or rate of battery discharge when alternator is off or
    malfunctioning

  \item Alternator Control Unit (ACU) and Low Voltage Warning -- combo alternator
    regulator and high-low voltage control unit; mounted on engine side of
    firewall

  \item ``LOW VOLTAGE'' light on instrument panel

  \item Circuit Breakers and Fuses -- Most are ``push to reset'' except
    ``Alternator Output'' and ``Landing Gear'' which are ``pull-off'' type and
    the ``AVN PWR'' which is a rocker switch; cigarette lighter and control
    wheel map light uses fuses as well as breakers

  \item Ground Service Plug Receptacle (Optional) - for use with external power
    during cold weather starting or lengthy maintenance work

  \item Lighting System: 3 navigation, taxi, landing, rotating beacon, strobes,
    courtesy, interior, flood, post lights (outside instruments), integral
    (inside instruments)

  \item Electrical instruments (turn coordinator, clock)

  \item Radio and navigation devices
\end{itemize}

Discuss pre-flight checklist in POH/AFM as it pertains to electrical system
items. In particular, note the ammeter during run-up with and without an
electrical load. Discuss the electrical fire checklist in the POH/AFM. Discuss
if the ammeter shows excessive rate of charge. Discuss if the low voltage light
illuminates during low RPM operations on the ground and goes off as RPM in
increased, this is not a problem. Otherwise, follow checklist procedures.

\section{Evaluation}

Lesson is complete when student can demonstrate and discuss proper use of
electrical system.

\section{References}

FAA-H-8083-25 Pilot's Handbook of Aeronautical Knowledge p. 5-19, POH / AFM pp.
3-6, 3-10, 4-21, 7-27, 7-29 thru 7-34, POH Supplement: Ground Service
Receptacle 1 thru 4, electrical diagram on p. 7-30.

\chapter{Landing Gear (C172RG)}

\section{Objective}

To teach the components and operating procedures of the landing gear system.

\section{Elements}

\begin{itemize}
  \item Components
  \item Pre-flight
  \item Normal operation
  \item Emergency operation
\end{itemize}

\section{Schedule}

Discussion (45 minutes), in-flight (20 minutes).

\section{Equipment}

Aircraft, Pilot Operating Handbook (POH) or FAA-approved Airplane Flight Manual (AFM).

\section{Instructor Actions}

Discuss the components (show hydraulic schematic, POH 7-28):
\begin{itemize}
  \item Nose gear--nitrogen/oil nose gear shock strut, positive mechanical down
    lock

  \item Nose gear doors--mechanically opened and closed by nose gear

  \item Main gear--tubular spring steel struts, positive mechanical down locks

  \item Hydraulic power pack--electrically driven, located aft of firewall
    between pilot and copilot's rudder pedals
    \begin{itemize}
      \item Pressurized between 1000-1500 psi
      \item Pressure switch causes electric pump to turn on
      \item If the pump stays on, there is a problem
      \item MIL-H-5606, red color, hydraulic fluid
    \end{itemize}

  \item Hydraulic actuators--one for each gear

  \item Landing gear lever--directs pressure

  \item Landing gear position indicator lights--required for flight

    \begin{itemize}
      \item Amber = up, green = down (some models, red gear unsafe light and
        green down light for other models)

      \item Lights are interchangeable

      \item Up and down switches for each gear, in series
    \end{itemize}

  \item Nose gear safety squat switch--open on the ground, prevents inadvertent
    gear retraction

  \item Gear-up warning system--intermittent tone through the speaker if
    manifold pressure $<$12'' Hg or flaps $\geq20^{\circ}$
    \begin{itemize}
      \item Push green light to turn off the tone
    \end{itemize}
  \item Emergency extension hand pump--double action hydraulic pump
    \begin{itemize}
      \item Can't retract the gear with pump
    \end{itemize}

  \item Circuit breakers -- ``pull off'' for gear pump, separate breaker for
    position lights
\end{itemize}

Discuss pre-flight of landing gear:

\begin{itemize}
  \item Cockpit--push to test gear indicator lights
  \item Check that gear handle is down
  \item Check reservoir at 25 hour intervals
  \item Outside--check for leaks
  \item Clear the wheel wells
  \item Make sure squat switch is open
\end{itemize}

Discuss normal operation:

\begin{itemize}
  \item $V_{LO}$ 140 (just below top of green arc), $V_{LE}$ 164 (redline)

  \item 5-7 seconds to extend or retract

  \item Keep hand on lever until operation is complete

  \item Tap brakes before retraction—tires expand due to centrifugal force and
    heat

  \item Mains swing down 2' during retraction

  \item Extend gear before entering traffic pattern

  \item Leave gear extended for continuous traffic pattern operations
\end{itemize}

Discuss manual gear extension:

\begin{itemize}
  \item Not an emergency
  \item Follow the checklist:
    \begin{itemize}
      \item Master ON
      \item Landing gear lever DOWN
      \item Breakers IN
      \item Hand pump--pump about 35 times until gear down light indicates
    \end{itemize}
\end{itemize}

\section{Student Actions}

Demonstrate and explain adequate gear pre-flight. Demonstrate proper use of
landing gear during flight. Conduct a manual gear extension in-flight.

\section{Evaluation}

Lesson is complete when student can demonstrate and discuss proper use of
landing gear in all flight scenarios.

\section{References}

FAA-H-8083-25 Pilot's Handbook of Aeronautical Knowledge p. 5-22,
FAA-H-8083-15A Airplane Flying Handbook p. 11-9, POH / AFM pp. 3-8, 4-18, 4-21,
7-11, 7-27.

\end{document}
