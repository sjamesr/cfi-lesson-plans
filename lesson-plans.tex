\documentclass[twoside,openright]{report}

\usepackage[margin=1in]{geometry}
\usepackage{cclicenses}
\usepackage{emptypage}
\usepackage{fancyhdr}
\usepackage{textcomp}
\usepackage{hyperref}
\pagestyle{fancy}

\fancyfoot[C]{This document is provided for informational use only. It is for
use by authorized instructors. Consult the aircraft manual and appropriate FAA
handbooks to double-check all information. \textcopyright{} Derek W Beck 2008.
Some Rights Reserved. \url{http://www.derekbeck.com/}.
\\
\byncsa{} \href{https://creativecommons.org/licenses/by-nc-sa/3.0/us/}{Licensed
under Creative Commons Attribution-Noncommercial-Share Alike License}.
}

\fancypagestyle{plain}

\title{Certificated Flight Instructor – Airplane (CFI-A)
Private and Commercial Pilot Lesson Plans}
\author{Derek W Beck}
\date{June 2008}

\begin{document}

\makeatletter
\begin{titlepage}
  \begin{center}
    \vspace*{1cm}
    \textbf{\@title}

    \vspace*{1cm}
    \@author

    \@date

    \vspace*{2cm}
    \ccby You may copy, distribute, display this copyrighted work -- and
    derivative works based upon it -- but only if they give credit to Derek W
    Beck and abide by the other license requirements listed here.

    \ccnc This work and its derivatives may not be sold without permission from
    Derek W Beck.

    \ccsa You may distribute derivative works only under a license identical to
    the licenses listed here and only if these licenses are explicitly depicted
    somewhere on the derivative work.
  \end{center}
\end{titlepage}

\setcounter{tocdepth}{0}
\tableofcontents

\chapter{Private Pilot Introduction}

\section{Objective}

Familiarize the student with the privileges, obligations and responsibilities
of a private pilot.

\section{Elements}

\begin{itemize}
  \item Typical VFR flight
  \item Privileges
  \item Training requirements
  \item Currency requirements
\end{itemize}

\section{Schedule}

Discussion (30 minutes).

\section{Instructor Actions}

Explain objectives.

\section{Evaluation}

Ensure understanding of private pilot flying, its objectives, etc.

\chapter{Piloting considerations}

\section{Objective}

To familiarize the student with currency requirements, health requirements,
medical requirements, etc.

\section{Elements}

\begin{itemize}
  \item Health (``I'M SAFE'')
  \item Medical requirements
  \item Aeromedical physiology
  \item Currency requirements
  \item Log books
\end{itemize}

\section{Schedule}

Discussion (30 minutes).

\section{Equipment}

14 CFR (FAR/AIM).

\section{Instructor Actions}

Discuss the following regulations and requirements:

\begin{itemize}
  \item Health considerations: ``I'M SAFE'': Illness (14 CFR 61.53, 91.17),
    Medication (14 CFR 91.17; the best list is at AOPA's members section or
    \url{http://www.leftseat.com/medcat1.htm}; FAA has no official list),
    Stress, Alcohol (14 CFR 91.17; none within 8 hrs, $<$ 0.04\% BAC), Fatigue,
    Emotion.

  \item Medical requirements: (14 CFR 61.23) 3rd class medical lasts 36
    calendar months ($<$ age 40) or 24 calendar months ($>$ age 40). A current
    medical is required to exercise the privileges allowed by the pilot
    certificates held.

  \item Aeromedical physiology: (chapter 8 of the AIM) hypoxia, ear block,
    sinus block, decompression sickness, hyperventilation, carbon monoxide
    poisoning and disorientation.

  \item Currency requirements: (14 CFR 61.57) 3 take-offs and landings every 90
    days for daytime, 3 full-stop take-offs and landings every 90 days for
    nighttime to fly with passengers. Also, a biennial flight review (14 CFR
    61.56) consisting of 1 hour of ground that at least covers 14 CFR 91 and 1
    hour of flight maneuvers as deemed adequate by the instructor. The pilot
    certificate (other than student certificate) lasts indefinitely (14 CFR
    61.19). Moving requires an update sent to the FAA within 30 days (14 CFR
    61.60).

  \item Log book requirements: (14 CFR 61.51) Only requirement is training and
    experience required to obtain certificates, ratings, a flight review, or
    currency requirements. Expand on how to log time and what to log.

\end{itemize}

\section{Evaluation}

Ensure understanding of piloting considerations, currency requirements, etc.

\section{References}

AIM Chapter 8, 14 CFR 61 and 91.

\chapter{Pre-flight}

\section{Objective}

To teach the student with the airplane and the necessary steps for pre-flight.

\section{Elements}

\begin{itemize}
  \item Checklists
  \item Pre-flight procedures
\end{itemize}

\section{Schedule}

Instructor demonstration (45 minutes)

\section{Equipment}

Aircraft, airplane Pilot Operating Handbook or FAA-approved Airplane Flight
Manual.

\section{Instructor Actions}

\begin{itemize}
  \item Discuss the POH/AFM. Discuss the importance of checklists.

  \item Discuss required documents (AROW).

  \item Discuss the need for all the steps as outlined in the POH. Discuss the
    instrument tolerances inside the cockpit. Discuss the inspection of the
    wing and control surfaces. Discuss fuel and oil, grades, types,
    contaminants, etc. Discuss landing gear, tires, brakes, etc. Discuss the
    engine and propeller.

  \item Discuss engine run-up.

  \item Discuss other items of operation in the POH/AFM.
\end{itemize}

\section{Student Actions}

Demonstrate a pre-flight inspection.

\section{Evaluation}

Lesson is complete when student can demonstrate and explain the need for each
procedure and checklist item that is listed as part of the pre-flight of the
aircraft.

\section{References}

Pilot Operating Handbook or FAA-approved Airplane Flight Manual, FAA-H-8083-3A
Airplane Flying Handbook Chapter 2.

\chapter{Aerodynamics}

\section{Objective}

To teach basic aerodynamic principles.

\section{Elements}

\begin{itemize}
  \item Four forces
  \item Airfoils
  \item Drag
  \item Stability and controllability
  \item Turning tendencies
  \item Climbs, descents, and turns
  \item Load factors
  \item Ground effect
  \item Adverse yaw
  \item Wingtip vortices
\end{itemize}

\section{Schedule}

Discussion (1 hour).

\section{Equipment}

Model aircraft.

\section{Instructor Actions}

Discuss the following:

\begin{itemize}
  \item Aerodynamics is the branch of physics that deals with the motion of a
    solid body through fluids.

  \item Four forces: lift, weight, thrust, drag

  \item Newton's laws of motion: first: an object in motion stays in motion,
    second: force equals mass times acceleration, third: for every action there
    is an equal and opposite reaction.

  \item In steady flight, the sum of these opposing forces is equal to zero.
    (Newton's first law) Lift = weight, thrust = drag.

  \item Bernoulli's principle: an increase in the speed of the fluid occurs
    simultaneously with a decrease in pressure.

  \item An airfoil is a structure designed to obtain reaction upon its surface
    from the air through which it moves or that moves past it. (Bernoulli)
\end{itemize}

Discuss the wing definitions: leading edge, trailing edge, camber, chord line.

\begin{itemize}
  \item Aspect ratio--wingspan to mean chord line
    \begin{itemize}
      \item High aspect ratio = more lift, less drag.
    \end{itemize}

  \item Angle of incidence--angle between the chord and the longitudinal axis

  \item Angle of attack (AoA)

  \item Wing always stalls at the critical angle of attack
\end{itemize}

Drag:

\begin{itemize}
  \item Parasite: increases as the square of the airspeed
    \begin{itemize}
      \item Form drag: shape of the aircraft, i.e. streamlined object

      \item Skin friction: surface finish

      \item Interference: if two objects are placed adjacent to one another, the
        resulting turbulence produced may be 50 to 200\% greater than the parts
        tested separately
    \end{itemize}
  \item Induced-byproduct of lift (horizontal component of lift), varies
    inversely as the square of the airspeed, also caused by downwash from
    wingtip vortices
\end{itemize}

Discuss the $L/D_{max}$ drag graph. (See figure 3-5 in PHAK)

Discuss stability (see PHAK 3-10) (Use marble and bowl analogy):

\begin{itemize}
  \item Stability is the inherent quality of an airplane to correct for
    conditions that may disturb its equilibrium, and to return or to continue
    on the original flightpath
  \item Static:
    \begin{itemize}
      \item Static stability is the initial tendency of an object to return to
        its original position after being disturbed

      \item Positive, neutral, negative

      \item Push controls: positive returns to original state; neutral: remains
        at new state; negative: keeps moving beyond
    \end{itemize}
  \item Dynamic:
    \begin{itemize}
      \item Tendency after equilibrium is disturbed (stability in motion)

      \item Positive, neutral, negative

      \item Positive: dampened, pattern of movements become smaller; neutral:
        pattern continues unchanged; negative: pattern diverges
    \end{itemize}
\end{itemize}

Discuss axes of rotation: longitudinal (roll/ailerons), lateral
(pitch/elevator), vertical (yaw/rudder).

Longitudinal stability (pitch):

\begin{itemize}
  \item Most affected by pilot, especially aircraft loading

  \item Draw CL, CG, horizontal stabilizer down force

  \item Horizontal stabilizer is an upside-down wing, provides longitudinal
    stability with a down force

  \item Nose moves up (tail down), horizontal stabilizer's AoA decreases,
    reduces lift and the nose moves back down

  \item Nose moves down (tail up), horizontal stabilizer's AoA increases,
    produces more lift and the pushes the nose moves back up

  \item Unstable if CG moves too close to or behind the center of lift
\end{itemize}

Lateral stability (roll)

\begin{itemize}
  \item Most common design factor for positive stability is wing dihedral (see
    PHAK Fig. 3-17)

  \item Positive dihedral is when the wing tips are higher than the wing root

  \item Wing position--high wing aircraft are more laterally stable than low
    wings (everything held constant)

  \item High wing aircraft require less dihedral than low wing aircraft
\end{itemize}

Vertical stability (yaw)

\begin{itemize}
  \item Vertical stabilizer, and the fuselage behind the CG to a lesser extent,
    provide directional stability

  \item Vertical stabilizer acts like the tail feathers on an arrow

  \item Swept back wings, and double taper wings to a lesser extent, also
    provide directional stability

  \item Increased induced drag on the wing that is moved forward pulls it back,
    causes Dutch roll tendency
\end{itemize}

Controllability versus Maneuverability (PHAK 3-10)

\begin{itemize}
  \item Controllability--capability of an aircraft to respond to the pilot’s
    control inputs, especially with regard to flightpath and attitude

  \item Maneuverability--quality of an airplane that allows it to be easily
    controlled and maneuvered and withstand stresses imposed by maneuvers

  \item An F-16 sacrifices stability for controllability, maneuverability
\end{itemize}

Turning tendencies (PHAK 3-23)

\begin{itemize}
  \item Torque--opposite reaction to the engine (Newton’s third law): leads to a
    left roll

  \item Spiraling slipstream--pushes on port side of vertical stabilizer,
    causing left yaw

  \item Gyroscopic action (precession)--a pitch down will cause a left yaw 90º
    from the top of the propeller, a pitch up will cause a right yaw 90º from
    the bottom of the propeller disk

  \item P-factor--downward blade has a higher angle of attack during a high
    pitch attitude, causing greater thrust on the right side of the propeller
    disk, creating a left yaw (and vice versa)
\end{itemize}

Climbs: Initially lift is greater, then it stabilizes in steady-state.

Descents: Initially lift is reduced, then it stabilizes in steady-state.

Turns: Show a vector diagram with horizontal component of lift.

Load factor: Any force applied to an airplane to deflect its flight from a
straight line. Ratio of total load acting on the airplane to the gross weight.
Discuss maneuvering speed $V_{a}$.

Limit load factors: Normal 3.8 to -1.52, Utility 4.4 to -1.76, Aerobatic 6.0 to
-3.0. 1.5 factor of safety built in. See load factor chart (PHAK Fig. 3-36),
noting $60^{\circ}$ of bank equals $2g$. Discuss $V_{g}$ diagram (PHAK Fig.
3-38).

Ground effect (PHAK 3-7):

\begin{itemize}
  \item Within one wingspan of earth’s surface, wingtip vortices are reduced

  \item Provides decrease in induced drag

  \item Wing will require a lower angle of attack in ground effect to produce
    the same lift
\end{itemize}

Adverse yaw (PHAK 4-2):

\begin{itemize}
  \item Purpose of the rudder is to counteract adverse yaw and control movement
    around the vertical axis

  \item Raised wing in a turn has a higher angle of attack, more lift, more
    induced drag

  \item Differential ailerons, Frise-type ailerons, and coupled ailerons and
    rudder reduce adverse yaw
\end{itemize}


Wingtip vortices (PHAK 3-6):

\begin{itemize}
    \item Spanwise movement of air along wing due to pressure differential

    \item Air from bottom of wing moves outward from fuselage and ``spills''
      over the wingtips, creating a vortex

    \item Air from the top of the wing flows in toward the fuselage and spills
      off the trailing edge--vortex is insignificant because fuselage limits the
      flow

    \item Vortices increase drag because of energy spent in producing turbulence

    \item High angle of attack = more violent vortices

    \item Heavier and slower aircraft = more violent vortices
\end{itemize}

Wake turbulence on take-off/landing:

\begin{itemize}
  \item Stay above glidepath

  \item Land beyond the point of landing of the preceding heavier aircraft--look
    for puffs of tire smoke

  \item Liftoff prior to the point a larger aircraft took off

  \item Light quartering tailwind keeps vortices on the runway the longest (the
    most dangerous)
\end{itemize}

\section{Evaluation}

Lesson is complete when student can demonstrate fundamental understanding of
aerodynamics.

\section{References}

FAA-H-8083-25 Pilot's Handbook of Aeronautical Knowledge (PHAK) Chapter 3,
FAA-H-8083-3A Airplane Flying Handbook Chapter 3, Aeronautical Information
Manual (AIM) 7-3-1.

\chapter{Aircraft Systems}

\section{Objective}

To teach the student the basics of aircraft systems.

\section{Elements}

\begin{itemize}
  \item Primary flight controls and trim
  \item Flaps, leading edge devices, spoilers
  \item Power plant
  \item Oil
  \item Avionics
\end{itemize}

\section{Schedule}

Discussion (45 minutes).

\section{Equipment}

Aircraft, Pilot Operating Handbook (POH) or FAA-approved Airplane Flight Manual
(AFM).

\section{Instructor Actions}

Discuss the components:

\begin{itemize}
  \item Primary flight controls--elevator, rudder, and ailerons
    \begin{itemize}
      \item Movement of the control surfaces changes the airflow and pressure
        distribution over and around the airfoil (relate to CG):
\begin{table}[h]
\centering
\begin{tabular}{l|l|l|l}
Primary control surface & Airplane movement & Axis of rotation & Type of stability \\\hline
aileron                 & roll              & longitudinal     & lateral           \\
elevator                & pitch             & lateral          & longitudinal      \\
rudder                  & yaw               & vertical         & directional
\end{tabular}
\end{table}
    \end{itemize}
  \item Ailerons:
    \begin{itemize}
      \item Control roll about longitudinal axis

      \item Most light airplanes have two ailerons, one on the trailing edge of
        each wing

      \item Connected to control wheel through cables and pulleys

      \item Move in opposite directions

      \item Discuss how ailerons change angle of attack and cause roll

      \item Adverse yaw can be counter acted with rudder use, or special
        aileron designs: differential ailerons, Frise-type ailerons, coupled
        ailerons and rudder

      \item Aileron trim: not common on light airplanes
    \end{itemize}
  \item Elevator
    \begin{itemize}
      \item Controls pitch about lateral axis

      \item Main purpose is to change the wing's angle of attack

      \item Most light airplanes have one elevator, located on the trailing
        edge of the horizontal stabilizer

      \item Some aircraft (e.g. Pipers) use a stabilator, or movable horizontal
        stabilizer

      \item Control wheel connected to the elevator by bell cranks, cables and
        pulleys

      \item Horizontal stabilizer has a negative angle of attack to provide
        downward force

      \item Elevator moves up to increase this downward push and move the nose
        up, and therefore increase the wing's angle of attack, and vice versa

      \item Discuss how elevator movement affects pitch attitude

      \item Elevator trim: almost all light aircraft are equipped with some
        form of elevator trim; moves in the opposite direction of the control
        surface, deflecting the control surface to relieve control pressure and
        maintain a constant pitch attitude
    \end{itemize}

  \item{Rudder}
    \begin{itemize}
      \item Controls the airplane about its vertical axis -- yaw

      \item Most light airplanes have one rudder, located on the trailing edge
        of the vertical stabilizer

      \item Controlled through the use of foot pedals, connected to the rudder
        by bell cranks, cables and pulleys

      \item Rudder does not turn the airplane, only yaws it

      \item Used in conjunction with the ailerons for properly turning the airplane

      \item Rudder trim: most light aircraft are equipped with some form of
        rudder trim; trim tabs move in the opposite direction of the control
        surface, deflecting the control surface to relieve control pressure and
        maintain a constant yaw attitude; some may be a ``manual rudder trim''
        or a piece of metal that is manually adjusted before flight
    \end{itemize}

  \item{Flaps}
    \begin{itemize}
      \item Increase lift and drag

      \item Flaps have three main functions: permit a slower landing speed,
        allow for a steep angle on descent without an increase in airspeed,
        shorten takeoff distance and allow for a steeper climb

      \item Plain--simplest, changes camber, increases lift, greatly increases
        drag

      \item Split--greater increase in lift vs. plain, more drag

      \item Slotted--(most common) increases lift coefficient significantly
        more than plain or split (high-energy air is ducted to the flap’s upper
        surface, delaying airflow separation)

      \item Fowler flap--a variety of slotted flap; changes camber and
        increases wing area

      \item Most flaps are located on the trailing edge of the wing in-between
        the fuselage and aileron

      \item In light aircraft they are controlled manually or electrically

      \item Extending the flaps will increase lift, cause a pitch up and loss
        of airspeed

      \item Retracting the flaps will decrease lift cause a pitch down and
        increase in airspeed
    \end{itemize}

  \item Leading edge devices:
    \begin{itemize}
      \item Fixed slots--direct airflow to upper wing surface and delay airflow
        separation; stall is delayed to greater angle of attack

      \item Moveable slats--leading edge segments on tracks; may be automatic or pilot-operated

      \item Leading edge flaps--increase coefficient of lift and camber
    \end{itemize}

  \item Spoilers:
    \begin{itemize}
      \item High-drag device; reduces lift, increases drag

      \item Used for roll control on some aircraft by eliminating adverse yaw

      \item Can shorten ground roll
    \end{itemize}

  \item Power plant:
    \begin{itemize}
        \item Reciprocating engines classified by cylinder arrangement (radial,
          inline, v-type, opposed), method of cooling (liquid or air), method
          of intake (carburetor, fuel-injection, turbo-charged), etc

        \item Main components: cylinders (contain intake/exhaust valves, spark
          plugs, pistons); crankcase (contains crankshaft, connecting rods);
          accessory housing (contains magnetos)

        \item Four-stroke operating cycle: intake, compression, power, exhaust
    \end{itemize}

  \item Oil:
    \begin{itemize}
      \item Lubricates, reduces friction, cools, provides a seal, and carries
        away contaminants

      \item Wet-sump system—sump is an integral part of the engine (in a dry
        system, it's a separate tank)

      \item Filter, cooler, filler cap/dipstick, quick-drain valve (bottom of
        sump)

      \item Pressure and temperature gauges (required instruments)
    \end{itemize}

  \item Avionics:
    \begin{itemize}
      \item Communication and navigation radios

      \item VOR, ADF, GPS

      \item Transponder

      \item Autopilot (if available)

      \item Avionics cooling fan--cools and eliminates moisture (if available)

      \item Microphone/headset intercom

      \item Static dischargers (wicks)
    \end{itemize}
\end{itemize}

\section{Evaluation}

Lesson is complete when student can demonstrate and discuss aircraft control
surfaces, power plant and other major systems.

\section{References}

FAA-H-8083-25 Pilot's Handbook of Aeronautical Knowledge Chapter 4-5, POH / AFM
Chapter 7.

\chapter{Fuel System (C172RG)}

\section{Objective}

To teach the components and operating procedures of the fuel system.

\section{Elements}

\begin{itemize}
  \item Components
  \item Pre-flight
  \item Normal operation
  \item Emergency operation
\end{itemize}

\section{Schedule}

Discussion (30 minutes).

\section{Equipment}

Aircraft, Pilot Operating Handbook (POH) or FAA-approved Airplane Flight Manual (AFM).

\section{Instructor Actions}

Discuss the following components:

\begin{itemize}
  \item Two vented integral fuel tanks--fuel flows by gravity from the tanks
    \begin{itemize}
      \item Standard tank capacity is 33 gallons (total 62 gal), and
        useable capacity is 24 gallons (total 44 gal)
    \end{itemize}

  \item Fuel tank vent--venting is accomplished by an interconnected line
    from the right fuel tank to the left tank, the left tank is vented
    overboard though a vent line, which protrudes from the bottom surface
    of the wing; the right fuel tank filler cap is also vented

  \item Fuel gauges--indicate the amount of fuel measured by a sensing unit
    in each tank and is displayed in gallons and pounds.

  \item Fuel sumps and drains--allow for checks at preflight to be made in
    the fuel tanks, selector, and strains, of visible moisture and/or
    sediments, as well as check for the proper grade of fuel

  \item Four-position selector valve--the selector can be set to OFF, BOTH,
    LEFT, and RIGHT; when the selector is not set to OFF, fuel is able to
    flow through to the rest of the system

  \item Fuel strainer--(inside oil compartment) removes any impurities,
    including moisture and other sediments that might be present in the
    fuel

  \item Manual primer--takes fuel directly from the strainer and vaporizes
    it directly into three of the cylinders

  \item Fuel pressure gauge--shows the fuel flow in PSI and can be used to
    indicate a failure in of the fuel pump

  \item Engine-driven fuel pump--driven by the engine to pump fuel to the
    carburetor

  \item Electric auxiliary fuel pump--electrically drives fuel to the
    carburetor and should be used when the fuel flow drops below 0.5 PSI
\end{itemize}

Discuss fuel grades: Aviation gasoline (AVGAS) is identified by an octane or
performance number (grade). The higher the grade of gasoline, the more pressure
the fuel can withstand without detonating. If the proper grade of fuel is not
available, use the next higher grade as a substitute (but not JET A). Never use
a lower grade. This can cause the cylinder head temperature and engine oil
temperature to exceed their normal operating range, which may result in
detonation. Available AVGAS is 80 (dyed red), 100 (dyed green), and 100LL (dyed
blue). The C172RG used 100LL.

Discuss pre-ignition and detonation.

Discuss refueling, including grounding, use of a ladder, etc. Note that if
refueling before flight, should redo sumping after the fuel has settled (at
least 10 minutes).

Discuss preflight of the fuel system per POH.

\section{Evaluation}

Lesson is complete when student can demonstrate and discuss proper use of fuel
system.

\section{References}

FAA-H-8083-25 Pilot's Handbook of Aeronautical Knowledge p. 5-13, POH / AFM p.
7-23.

\chapter{Electrical System (C172RG)}

\section{Objective}

To teach the components and operating procedures of the electrical system.

\section{Elements}

\begin{itemize}
  \item Components
  \item Pre-flight
  \item Emergency operation
\end{itemize}

\section{Schedule}

Discussion (30 minutes).

\section{Equipment}

Aircraft, Pilot Operating Handbook (POH) or FAA-approved Airplane Flight Manual (AFM).

\section{Instructor Actions}

Discuss the following components:
\begin{itemize}
  \item 28V DC System

  \item Battery -- 24V Located aft of the rear cabin wall

  \item Alternator -- 60A Belt-driven

  \item Buses: Primary and Avionics (interconnected with primary via avionics
    power switch/breaker)

  \item Master Switch -- split switch: battery and alternator

  \item Avionics Power Switch -- power from primary to avionics bus; is also a
    circuit breaker

  \item Ammeter -- indicates battery charging rate when alternator is on and
    working, or rate of battery discharge when alternator is off or
    malfunctioning

  \item Alternator Control Unit (ACU) and Low Voltage Warning -- combo alternator
    regulator and high-low voltage control unit; mounted on engine side of
    firewall

  \item ``LOW VOLTAGE'' light on instrument panel

  \item Circuit Breakers and Fuses -- Most are ``push to reset'' except
    ``Alternator Output'' and ``Landing Gear'' which are ``pull-off'' type and
    the ``AVN PWR'' which is a rocker switch; cigarette lighter and control
    wheel map light uses fuses as well as breakers

  \item Ground Service Plug Receptacle (Optional) - for use with external power
    during cold weather starting or lengthy maintenance work

  \item Lighting System: 3 navigation, taxi, landing, rotating beacon, strobes,
    courtesy, interior, flood, post lights (outside instruments), integral
    (inside instruments)

  \item Electrical instruments (turn coordinator, clock)

  \item Radio and navigation devices
\end{itemize}

Discuss pre-flight checklist in POH/AFM as it pertains to electrical system
items. In particular, note the ammeter during run-up with and without an
electrical load. Discuss the electrical fire checklist in the POH/AFM. Discuss
if the ammeter shows excessive rate of charge. Discuss if the low voltage light
illuminates during low RPM operations on the ground and goes off as RPM in
increased, this is not a problem. Otherwise, follow checklist procedures.

\section{Evaluation}

Lesson is complete when student can demonstrate and discuss proper use of
electrical system.

\section{References}

FAA-H-8083-25 Pilot's Handbook of Aeronautical Knowledge p. 5-19, POH / AFM pp.
3-6, 3-10, 4-21, 7-27, 7-29 thru 7-34, POH Supplement: Ground Service
Receptacle 1 thru 4, electrical diagram on p. 7-30.

\chapter{Landing Gear (C172RG)}

\section{Objective}

To teach the components and operating procedures of the landing gear system.

\section{Elements}

\begin{itemize}
  \item Components
  \item Pre-flight
  \item Normal operation
  \item Emergency operation
\end{itemize}

\section{Schedule}

Discussion (45 minutes), in-flight (20 minutes).

\section{Equipment}

Aircraft, Pilot Operating Handbook (POH) or FAA-approved Airplane Flight Manual (AFM).

\section{Instructor Actions}

Discuss the components (show hydraulic schematic, POH 7-28):
\begin{itemize}
  \item Nose gear--nitrogen/oil nose gear shock strut, positive mechanical down
    lock

  \item Nose gear doors--mechanically opened and closed by nose gear

  \item Main gear--tubular spring steel struts, positive mechanical down locks

  \item Hydraulic power pack--electrically driven, located aft of firewall
    between pilot and copilot's rudder pedals
    \begin{itemize}
      \item Pressurized between 1000-1500 psi
      \item Pressure switch causes electric pump to turn on
      \item If the pump stays on, there is a problem
      \item MIL-H-5606, red color, hydraulic fluid
    \end{itemize}

  \item Hydraulic actuators--one for each gear

  \item Landing gear lever--directs pressure

  \item Landing gear position indicator lights--required for flight

    \begin{itemize}
      \item Amber = up, green = down (some models, red gear unsafe light and
        green down light for other models)

      \item Lights are interchangeable

      \item Up and down switches for each gear, in series
    \end{itemize}

  \item Nose gear safety squat switch--open on the ground, prevents inadvertent
    gear retraction

  \item Gear-up warning system--intermittent tone through the speaker if
    manifold pressure $<$12'' Hg or flaps $\geq20^{\circ}$
    \begin{itemize}
      \item Push green light to turn off the tone
    \end{itemize}
  \item Emergency extension hand pump--double action hydraulic pump
    \begin{itemize}
      \item Can't retract the gear with pump
    \end{itemize}

  \item Circuit breakers -- ``pull off'' for gear pump, separate breaker for
    position lights
\end{itemize}

Discuss pre-flight of landing gear:

\begin{itemize}
  \item Cockpit--push to test gear indicator lights
  \item Check that gear handle is down
  \item Check reservoir at 25 hour intervals
  \item Outside--check for leaks
  \item Clear the wheel wells
  \item Make sure squat switch is open
\end{itemize}

Discuss normal operation:

\begin{itemize}
  \item $V_{LO}$ 140 (just below top of green arc), $V_{LE}$ 164 (redline)

  \item 5-7 seconds to extend or retract

  \item Keep hand on lever until operation is complete

  \item Tap brakes before retraction—tires expand due to centrifugal force and
    heat

  \item Mains swing down 2' during retraction

  \item Extend gear before entering traffic pattern

  \item Leave gear extended for continuous traffic pattern operations
\end{itemize}

Discuss manual gear extension:

\begin{itemize}
  \item Not an emergency
  \item Follow the checklist:
    \begin{itemize}
      \item Master ON
      \item Landing gear lever DOWN
      \item Breakers IN
      \item Hand pump--pump about 35 times until gear down light indicates
    \end{itemize}
\end{itemize}

\section{Student Actions}

Demonstrate and explain adequate gear pre-flight. Demonstrate proper use of
landing gear during flight. Conduct a manual gear extension in-flight.

\section{Evaluation}

Lesson is complete when student can demonstrate and discuss proper use of
landing gear in all flight scenarios.

\section{References}

FAA-H-8083-25 Pilot's Handbook of Aeronautical Knowledge p. 5-22,
FAA-H-8083-15A Airplane Flying Handbook p. 11-9, POH / AFM pp. 3-8, 4-18, 4-21,
7-11, 7-27.

\chapter{Propeller System}

\section{Objective}

To familiarize the student with common propeller systems.

\section{Elements}

\begin{itemize}
  \item Propeller basics
  \item Pre-flight
  \item Constant-speed propellers
\end{itemize}

\section{Schedule}

Discussion (45 minutes).

\section{Equipment}

Aircraft, Pilot Operating Handbook (POH) or FAA-approved Airplane Flight Manual
(AFM).

\section{Instructor Actions}

Discuss the two main types of propellers: fixed pitch and constant speed.
Others include full-feathering, reversing, and ground-adjustable.

Discuss basics of propellers: materials (typically wood, composite or aluminum
alloy).

The hub is the center of the propeller, and the spinner mounts above and covers
the hub. Propeller blades are airfoils, and have camber like any airfoil.

Blade twist: the hub has less pitch than the tips. That is, the pitch on each
blade changes the further you go from the hub. This is necessary to give the
relative same thrust across the blade despite the increased speed as you move
out towards the tips.

Discuss pre-flight of propellers: if a nick exists from one side to the other
side at the propeller's edge, or if any is, in your judgment, too significant
(recommended: greater than a quarter inch or greater than two tenths an inch
deep), it must be looked at by a mechanic. Nicks in the last third of the
propeller are extremely worrisome. This is because the outside of the propeller
travels faster than the inside, leading to higher stresses in the metal, which
can cause fatigue at the nick. At 2500 RPM, a typical single-engine
propeller’s tips travel at nearly 650 knots. Nicks are shaved off, which is
why blades have a tolerance for how long they can be. If one blade is shaved
down, the other(s) must be as well to remain balanced. Nicks that are not
dealt with can grow to bigger ones. The worse-case scenario is propeller
separation during flight, shearing the blade and causing such imbalance that
the engine is then sheared from the airplane. Other pre-flight considerations:
corrosion (painted propellers prevents corrosion), leaking oil at the hub, etc.
Also try to twist constant speed propellers (hold them at their midpoint, not
their tips as this can cause damage), they should resist rotation and not
rotate more than a quarter inch. The blades should also not move forward or aft
more than a quarter inch.

Propellers are subject to periodic inspections based on tachometer and on
calendar months. Consult the logbooks to see the propeller inspections.

Discuss the mechanics of the constant speed propeller and the governor system.
Discuss the flyweight system, speeder spring, and how these mechanisms control
high-pressure oil to the propeller hub. Discuss how the propeller hub's
internal piston moves the blades' pitch against the propeller spring. Discuss
loss of engine oil or governor, and the resulting safe condition of high-RPM
low-pitch. Discuss which settings are ideal for take-off and landing, and
consult the AFM/POH for more discussion.

\section{Evaluation}

Lesson is complete when student can discuss a constant speed propeller system.

\section{References}

%% TODO(sjr): these links are broken
\begin{itemize}
  \item \url{http://www.mccauley.textron.com/prop/prop-tech/pg00intro.html}

  \item \url{http://flash.aopa.org/asf/engine_prop/}
\end{itemize}

\chapter{VFR Instruments}

\section{Objective}

To teach the student about VFR instruments, their function, and the
requirements.

\section{Elements}

\begin{itemize}
  \item VFR required instruments
  \item Vacuum-driven gyroscopic instruments
  \item Electric-driven gyroscopic instruments
  \item Pitot-static system
  \item Compass errors
\end{itemize}

\section{Schedule}

Discussion (1h30m).

\section{Equipment}

Aircraft.

\section{Instructor Actions}

Explain required instruments (``A-GOOSEACAT'': Anti-collision lights, Gas
gauges, Oil pressure gauge, Oil temperature gauge, Seatbelts with should
harnesses, ELT, Airspeed indicator, Compass, Altimeter, Tachometer for each
engine), plus night requirements (``APES'': Anti-collision lights, Position
lights, Electrical source, Spare fuses as required). For complex aircraft:
landing gear position indicators. For commercial flights: landing light. Other
requirements: manifold pressure gauge for each altitude engine and temperature
gauge for each liquid-cooled engine. Source: 14 CFR 91.205.

Explain Vacuum-Driven Gyroscopic Instruments:

Attitude Indicator: Explain construction, demonstrate and explain behavior and
indications.

Directional Gryo (Heading Indicator): Explain construction, demonstrate and
explain behavior, indications, tick marks (i.e. $45^{\circ}$ tick marks),
precession (check with magnetic compass every 15 minutes). Note: Heading
indicator is always the primary instrument for bank.

Explain Electric Gyroscopic Instruments:

Turn Coordinator (with Inclinometer): Explain needle and ball construction.
Demonstrate and explain needle and ball behavior under all conditions. Discuss
and explain needle and ball indications.

Explain Pitot-Static System:

Airspeed Indicator: Explain construction, demonstrate and explain behavior,
indications (V-Speeds), types of airspeed, errors (pitot tube at high pitch
attitude). In straight and level flight the airspeed indicator is the primary
power instrument. In climbs and descents at a specific airspeed, the airspeed
indicator is the primary pitch instrument.

Altimeter: Explain construction, demonstrate and explain behavior, types of
altitudes, errors (``High to Low or Hot to Cold, Look Out Below!'') In straight
and level flight the altimeter is the primary pitch instrument. Altimeter
should be within 75' of field elevation.

Vertical Speed Indicator: Explain construction, demonstrate and explain
behavior, rate information (vertical speed) versus trend information (changes
of vertical speed), 6-9 second lag. During constant rate climbs and descents,
the vertical speed indicator is the primary pitch instrument.

Explain Pitot-Static System Blockages:
\begin{itemize}
  \item Complete blockage (Pitot tube and drain, static ports): airspeed and
    altimeter will stay constant and VSI will indicate zero

  \item Pitot tube complete blockage (static port open): altimeter and VSI will
    indicate correctly but airspeed will react like an altimeter

  \item Pitot tube blocked, drain clear (static port open): altimeter and VSI
    will indicate correctly but airspeed will decrease to zero

  \item Static port blocked (only): airspeed continues to operate but will be
    erroneous; at higher altitude than when the blockage occurred, airspeed
    will show slower, and vice versa
\end{itemize}

Discuss magnetic compass errors. Explain construction, demonstrate and explain
behavior, variation (magnetic versus true north), deviation, magnetic dip
errors, Northerly Turning Error (``Lag from the North, Lead from the South'';
When determining to lag or lead, remember OSUN: ``Overshoot when turning to
South, undershoot when turning to North''), Acceleration Errors (On an east or
west heading, ANDS: Accelerate turns to the North, Decelerate turns to the
South).

\section{Evaluation}

Lesson is complete when student has a thorough knowledge of required
instruments and compass errors.

\section{References}

FAA-H-8083-15A Instrument Flying Handbook Chapter 3.

\chapter{Aircraft Documents and Maintenance}

\section{Objectives}

To familiarize the student about the required documents needed to legally
operate an aircraft as well as the maintenance required to keep the aircraft in
a legal status.

\section{Elements}

\begin{itemize}
  \item Certification and documentation required
  \item Maintenance and inspections
  \item Preventative maintenance
\end{itemize}

\section{Schedule}

Discussion (45 minutes).

\section{Equipment}

Aircraft, Aircraft maintenance records.

\section{Instructor Actions}

The pilot in command ultimately decides if the aircraft is airworthy (14 CFR
91.7).

Discuss required documents ``AROW'': \textbf{A}irworthiness certificate (14 CFR
91.203), \textbf{R}egistration certificate (14 CFR 91.203), \textbf{O}perating
manual and placards (FAA-approved Airplane Flight Manual, serial number
specific) (14 CFR 91.9), \textbf{W}eight and balance, current and specific for
the exact aircraft serial number (part of the AFM, covered under 14 CFR 91.9).

\begin{table}[h]
\centering
\begin{tabular}{ll}
A & \textbf{A}nnual Inspection (12 calendar months) (14 CFR 91.409)        \\
V & \textbf{V}OR (30 days for IFR, annual for VFR) (14 CFR 91.171)         \\
1 & Inspection at \textbf{1}00 Hours off of tachometer (14 CFR 91.409)     \\
A & \textbf{A}ltimeter/Pitot Static System (24 calendar months) 91.411)    \\
T & \textbf{T}ransponder/Mode C (24 calendar months) (14 CFR 91.413)       \\
E & \textbf{E}LT (50\% battery life/1 hr cum. use, annual) (14 CFR 91.207) \\
A & \textbf{A}irworthiness Directives (AD’s, aka Recalls) (14 CFR 91.403),
\end{tabular}
\end{table}

ADs may be divided into two categories:
\begin{itemize}
  \item those of an emergency nature requiring immediate compliance prior to
    further flight, and

  \item those of a less urgent nature requiring compliance within a specified
    period of time
\end{itemize}

Preventative maintenance can be accomplished by any certificated pilot (14 CFR
43 Appendix A). Repairs must be completed by FAA-certified mechanics. If they
are major or minor repairs, as defined by 14 CFR 43 Appendix A, defines what
level of oversight and qualification is required of the mechanic or repair
facility.

A special flight permit is a Special Airworthiness Certificate issued
authorizing operation of an aircraft that does not currently meet applicable
airworthiness requirements but is safe for a specific flight (e.g. to a repair
facility).

\section{Evaluation}

Lesson is complete when student can demonstrate and comprehend emergency
considerations.

\section{References}

FAA-H-8083-25 Pilot's Handbook of Aeronautical Knowledge Chapter 7.

\chapter{Weight and Balance}

\section{Objective}

To teach the student aircraft weight and balance considerations.

\section{Elements}

\begin{itemize}
  \item Weight and balance definitions
  \item Effects of greater weight
  \item Effects of CG location
  \item Discuss how to calculate weight and balance
\end{itemize}

\section{Schedule}

Discussion (1 hour).

\section{Equipment}

Aircraft, airplane Pilot Operating Handbook or FAA-approved Airplane Flight
Manual.

\section{Instructor Actions}

Discuss weight and balance limitations (max weights determined by structural
strength and performance). Discuss CG and the CG envelope. Define center of
pressure (lift) and its relation to angle of attack (center of lift moves
forward with higher AoA). CG is always ahead of center of pressure (otherwise
an airplane would tumble). Compare lift on the wing, CG, and lift generated
from the horizontal stabilizer.

Define weight definitions (empty weight, payload, zero fuel weight, fuel load,
useful load). Define standard weights: Fuel (Avgas) 1 gallon = 6 lbs, Oil 1
gallon = 7.5 lbs (approx. 2 lbs per quart). Define moment (tendency to rotate),
arm (distance at which a force is applied), station (arm, measured in reference
to datum on aircraft).

Discuss the effects of greater weight (higher take-off speeds, longer take-off
run, reduced rate and angle of climb, lower maximum altitude, shorter range,
reduced cruising speed, reduced maneuverability, higher stalling speed, higher
approach and landing speed, longer landing roll, excessive weight on
nose/tailwheel).

Discuss effects of CG location. Discuss how this can shift due to weight shift
during flight, fuel burn, etc. Forward CG: nose heavy, increased take-off and
landing speed, higher stall speed, good stall recovery, higher angle of attack,
less range and endurance, increased stability. Aft CG: tail heavy, decreased
take-off and landing speed, lower stall speed, poor stall recovery, smaller
angle of attack, greater range and endurance, decreased stability.

Discuss calculations using examples from the POH. Discuss the ``WAM'' equation:
$W \cdot A = M$. Weight times Arm equals Moment. Discuss how this simple setup
can be used for weight shift, weight change, as well as standard CG
calculations.  Discuss how lateral CG is not computed but can cause wing
heaviness.

\section{Student Actions}

Calculate weight and balance under a variety of scenarios or as directed.

\section{Evaluation}

Lesson is complete when student can demonstrate and calculate weight and balance.

\section{References}

FAA-H-8083-25 Pilot’s Handbook of Aeronautical Knowledge Chapter 8.

\chapter{Weather Theory}

\section{Objective}

To teach the student about basic weather theory and how to anticipate possible
weather conditions.

\section{Elements}

\begin{itemize}
  \item Atmosphere
  \item Wind
  \item Moisture and stability
  \item Clouds
  \item Fronts
\end{itemize}

\section{Schedule}

Discussion (2 hours).

\section{Equipment}

FAA-H-8083-25 Pilot’s Handbook of Aeronautical Knowledge, AC 00-6A Aviation
Weather.

\section{Instructor Actions}

Discuss the following:
\begin{itemize}
  \item Nature of the atmosphere (78\% nitrogen, 21\% oxygen, 1\% other)

  \item Troposphere, the first layer of the atmosphere, contains most of the
    weather, and goes up from 20,000' to 48,000' MSL at the poles

  \item Pressure

  \item The cause of weather: uneven heating by the sun

  \item Convection

  \item How air flows from high (clockwise) to low (counterclockwise, i.e.
    cyclonic) – wind

  \item How oceans and mountains affect wind

  \item Turbulence due to mountains and man-made objects

  \item Moisture, humidity and relative humidity

  \item Stability, inversions

  \item Temperature and dew point

  \item Fog

  \item Cloud formation (vapor in the air, refer to condensation nuclei), types
    of clouds (cumulus, stratus, cirrus, nimbus, etc)

  \item Relate stability to clouds (unstable = cumuliform, stable = stratiform)

  \item Thunderstorms

  \item Cloud cover (few, scattered) and ceilings (broken, overcast)

  \item Precipitation

  \item Air masses

  \item Fronts (boundaries between air masses); flying across a front will lead
    to wind shift and likely some form of weather (always know where the fronts
    are in a long-distance flight)

  \item Warm fronts:
    \begin{itemize}
      \item Prior to passage: cirriform or stratiform clouds, fog, etc, plus
        cumulonimbus in summer; light to moderate precipitation; poor
        visibility; winds from south-southeast

      \item At passage: pressure falling; stratiform clouds; drizzle; poor
        visibility but improving; rising temperature

      \item After passage: stratocumulus clouds; rain showers possible;
        visibility improving but hazy; wind from the south-southwest; slight
        rise in pressure
    \end{itemize}

  \item Cold fronts:
    \begin{itemize}
      \item Prior to passage: cirriform or towering cumulus clouds,
        cumulonimbus also possible; rain showers and haze; wind from
        south-southwest; high dew point; falling pressure

      \item At passage: towering cumulus or cumulonimbus; heavy rain, hail,
        thunderstorms possible; more severe cold fronts produce tornadoes; poor
        visibility; winds variable and gusting; temperature and dew point
        falling rapidly; pressure falling rapidly

      \item After passage: clouds dissipate with corresponding decrease in
        precipitation; good visibility; winds from the west-northwest;
        temperatures remain cooler; pressure begins to rise
    \end{itemize}

  \item Stationary fronts: two air masses holding position for days; the
    weather at the front is usually a mix of warm and cold front weather

  \item Occluded fronts (when a fast-moving cold front catches up to a
    slow-moving warm front):
    \begin{itemize}
      \item Temperatures of the colliding fronts play a large part in the
        weather of occluded fronts

      \item Conditions vary depending on the air mass ahead of the warm front
        being overtaken

      \item Cold front occlusion: the fast-moving cold front is colder than the
        cold mass ahead of the warm front; the weather at this front is usually
        a mix of warm followed by cold front weather and is relatively stable

      \item Warm front occlusion: the fast-moving cold front is less cool than
        the cold mass ahead of the warm front, resulting in the most severe
        weather including embedded thunderstorms, rain, fog, etc
    \end{itemize}
\end{itemize}

\section{Evaluation}

Lesson is complete when student can discuss various aspects of weather phenomena.

\section{References}

FAA-H-8083-25 Pilot's Handbook of Aeronautical Knowledge, AC 00-6A Aviation
Weather.

\chapter{Weather Services}

\section{Objective}

To teach sources of weather information.

\section{Elements}

\begin{itemize}
  \item Observations
  \item Forecasts
  \item Weather products
  \item Briefings
\end{itemize}

\section{Schedule}

Discussion (2 hours).

\section{Equipment}

AC 00-6A Aviation Weather, Sectional or Terminal chart.

\section{Instructor Actions}

Discuss the following, and in the case of each weather source, go through real examples.

Preflight actions: 14 CFR 91.103 requires that the PIC shall become familiar
with all available information concerning that flight including weather
information when flying beyond the vicinity of an airport.

To contact FSS: 1-800-WX-BRIEF, Duat, Duats, Visit in person, 122.2 or discrete
frequency (see sectional chart). A FSS standard briefing gives complete and
customized description of all conditions that may affect the proposed flight,
based on route and altitude to by flown. Includes adverse conditions, synopsis,
current and forecast conditions, winds and temps aloft, NOTAM’s, anything else
requested.

Briefing formats:

\begin{table}[h]
\centering
\begin{tabular}{llll}
Time                                    & Visualize        & Compare          & Briefing     \\\hline
6+ hours before departure               & big picture      & N/A              & outlook      \\
1-4 hours before departure              & detailed picture & big picture      & standard     \\
just before departure and during flight & updated picture  & detailed picture & abbrieviated
\end{tabular}
\end{table}

Current weather products (observations):

\begin{itemize}
  \item Satellite Weather picture:
    \begin{itemize}
      \item graphically display cloud position and approx. thickness and height
      \item issued every 30 minutes and as needed
      \item valid at time of report
    \end{itemize}
  \item Radar Summary Chart:
    \begin{itemize}
      \item graphically display areas of precipitation—not clouds
      \item issued hourly and as needed
      \item valid at time of report
      \item contours indicate intensity of precipitation
    \end{itemize}
  \item Weather Depiction Chart:
    \begin{itemize}
      \item same as significant weather chart
      \item issued every 3 hours
      \item valid at time of report
    \end{itemize}
  \item Freezing Level Chart
  \item Aviation Routine Weather Report (METAR) (AIM 7-1-30):
    \begin{itemize}
      \item aviation routine weather report at the surface, issued hourly
      \item SPECI issued when certain significant changes occur
    \end{itemize}
  \item Pilot Weather Report PIREP's (UA/UUA):
    \begin{itemize}
      \item provide information on actual flight conditions as experienced by
        pilots
      \item issued upon receipt
      \item valid at time of report
      \item UA = normal, UUA = urgent
    \end{itemize}
\end{itemize}

Forecast weather products:

\begin{itemize}
  \item AIRMET (WA):
    \begin{itemize}
      \item Airman's meteorological information
      \item Moderate icing (zulu AIRMET), turbulence (tango AIRMET), IFR
        (sierra AIRMET), mountain obscuration, sustained surface winds $>$30
        knots
    \end{itemize}
  \item SIGMET (WS):
    \begin{itemize}
      \item significant meteorological information
      \item severe turbulence, icing, widespread duststorms, sandstorms,
        volcanic ash lowering visibility to $<$3 SM
      \item valid 4 hours (except 6 hours for hurricanes)
    \end{itemize}
  \item Convective SIGMET (WST):
    \begin{itemize}
      \item severe thunderstorms with surface winds greater than 50 knots, hail
        at the surface greater than or equal to 3/4 inch in diameter,
        tornadoes, embedded thunderstorms, lines of thunderstorms, or
        thunderstorms with heavy or greater precipitation
      \item valid 2 hours
    \end{itemize}
  \item Prognostic Charts:
    \begin{itemize}
      \item 24-, 36-, and 72-hour formats
      \item graphically display general weather conditions for contiguous U.S.
      \item 4 times daily
      \item left panel valid 12 hours, right panel valid 24 hours
      \item types and positions of fronts and pressure systems
      \item pressures in millibars.
      \item significant weather chart shows areas of VFR, MVFR, IFR, freezing
        level
    \end{itemize}
  \item Convective Outlook Chart
    %% TODO(sjr): replace this with Graphical Area Forecast, FA's are gone
  \item Area Forecast (FA):
    \begin{itemize}
      \item describes forecast weather conditions for several states
      \item issued 3 times daily
      \item synopsis valid 18 hours, weather and clouds 12 hours, +6 hr outlook
    \end{itemize}
  \item Terminal Aerodrome Forecast (TAF):
    \begin{itemize}
      \item describes forecast weather for an area within 5sm of airport
      \item issued 4 times daily
      \item valid 24 hours
    \end{itemize}
  \item Winds and Temperatures Aloft Forecast (FD):
    \begin{itemize}
      \item provide estimated wind direction, speed, and temperatures at
        selected stations and altitudes
      \item issued 2 times daily
      \item valid as stated—6, 12, or 24 hours
      \item winds greater than 100, subtract 50 from wind direction
    \end{itemize}
\end{itemize}

In-flight weather services:

\begin{itemize}
  \item Automatic Terminal Information System ATIS:
    \begin{itemize}
      \item airport name, time (UTC), wind direction and speed, visibility and
        obstructions, cloud coverage, temp and dew point, altimeter, remarks.
      \item different from AWOS/ASOS in that ATIS is usually only issued hourly
        and includes NOTAM's
    \end{itemize}
  \item AWOS, ASOS:
    \begin{itemize}
      \item AWOS—automated weather observation system (AIM 4-3-26, 7-1-12)
      \item ASOS—automated surface observation system
    \end{itemize}
  \item EFAS
    \begin{itemize}
      \item en-route flight advisory service (aka ``Flight Watch'')
      \item 122.0 MHz above 5000' AGL
      \item operated by the local center, e.g. ``Los Angeles Flight Watch''
    \end{itemize}
  \item TWEB:
    \begin{itemize}
      \item continuous transcribed weather broadcast.
      \item available over selected NAVAID's (T on chart)
    \end{itemize}
  \item HIWAS
    \begin{itemize}
      \item Hazardous In-flight Weather Advisories Services HIWAS
      \item recorded severe weather advisories
      \item ``H'' on NAVAID on charts
    \end{itemize}
\end{itemize}

NOTAMS (Notices to Airmen):

\begin{itemize}
  \item Notification of unforeseen changes in the national airspace system, not
    known in sufficiently in advance to publicize by other means, that may
    affect the pilot’s decision to make a flight

  \item May be divided into three categories: local (e.g. taxi way closures),
    distant (e.g. VOR out of service), FDC (issued by the Flight Data Center,
    regulatory in nature such as temporary flight restrictions, instrument
    approach procedure changes, etc)

  \item Obtain the same way a briefing can be obtained (FSS, Duats, Duat, etc)
\end{itemize}

\section{Evaluation}

\begin{itemize}
  \item Exhibit knowledge of the elements related to weather information by
    analyzing weather reports, charts, and forecasts from various sources with
    emphasis on:
    \begin{itemize}
        %% TODO(sjr): remove references to FA.
      \item METAR, TAF, and FA.
      \item surface analysis chart.
      \item radar summary chart.
      \item winds and temperature aloft chart.
      \item significant weather prognostic charts.
      \item convective outlook chart.
      \item AWOS, ASOS, and ATIS reports.
    \end{itemize}
  \item Make a competent ``go/no-go'' decision based on available weather information.
\end{itemize}

\section{References}

AC 00-6A Aviation Weather, FAA-H-8083-25 Pilot's Handbook of Aeronautical
Knowledge Chapter 12.

\chapter{Airspace}

\section{Objective}

To familiarize the student with the types and purpose of the various airspace
designations in the National Airspace System (NAS).

\section{Elements}

\begin{itemize}
  \item Class A,B,C,D,E,G
  \item Special-Use Airspace
  \item Other airspace
  \item Special VFR
\end{itemize}

\section{Equipment}

Sectional and/or Terminal Charts.

\section{Schedule}

Discussion (40 minutes).

\section{Instructor Actions}

%% TODO(sjr): include the airspace diagram from the IFH
Discuss each of the controlled airspaces A-E, and G, and their visibility and
cloud requirements, entry requirements, etc. (Refer to FAA-H-8083-15A
Instrument Flying Handbook Fig. 8-1).

Airspace speed limits (14 CFR 91.117)
\begin{itemize}
  \item Class B—surface to 10,000' MSL, max speed 250 kts (200 kts underlying)
  \item Class C—surface to 4000' AGL, max speed 200 kts within 4nm, 2500 AGL
  \item Class D—surface to 2500' AGL, max speed 200 kts
  \item Otherwise, no civil aircraft may operate greater than Mach 1 (14 CFR 91.817)
\end{itemize}

Special Use Airspace (AIM 3-4-1)
\begin{itemize}
  \item Prohibited areas--flight not permitted
  \item Restricted areas--flight subject to restrictions (artillery, aerial
    gunnery, etc.), permission required
  \item Warning areas--3nm beyond coast, similar to restricted, but no
    permission required
  \item MOA--military training activities, a military version of ``warning
    areas''
  \item Alert areas--ok to enter, high volume of activity (pilot training,
    unusual aerial activity)
  \item Controlled firing areas—not charted, activities are suspended if
    aircraft approach, no permission required
\end{itemize}

Other Airspace Areas (AIM 3-5-1)
\begin{itemize}
  \item MTR--military training route; military aircraft $>$ 250 kts; 4 digits $<$
    1500AGL
  \item TFR--temporary flight restriction
  \item Parachute jump operations
  \item Published VFR routes
  \item VFR flyways--back of TAC chart
  \item VFR corridors--(e.g. Los Angeles Special Flight Rules)
  \item Class B VFR transition routes--(e.g. San Diego, mini route, Hollywood
    Park route)
  \item TRSA--(e.g. Palm Springs); participation is voluntary but recommended
    (AIM 4-1-17)
  \item National Security Areas--requested not to fly through; flight may be
    prohibited at certain times, check NOTAMs
\end{itemize}

Special VFR (14 CFR 91.157, AIM 4-4-6)
\begin{itemize}
  \item 1sm visibility, clear of clouds
  \item At night, must have an instrument rating and IFR-equipped aircraft
  \item Must ask for it, ATC cannot offer
  \item Not allowed at some airports--see sectional or terminal chart
\end{itemize}

Marginal VFR (not a clearance)
\begin{itemize}
  \item Visibility between 3-5 statute miles and/or ceiling between 1000-3000' AGL
\end{itemize}

\section{Evaluation}

Lesson is complete when student can demonstrate understanding of the various
airspaces in the National Airspace System.

\section{References}

FAA-H-8083-25 Pilot's Handbook of Aeronautical Knowledge Chapter 13,
FAA-H-8083-15A Instrument Flying Handbook Chapter 8, Aeronautical Information
Manual (AIM) Chapter 3.

\end{document}
