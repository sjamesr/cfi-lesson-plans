\section{Aircraft Systems}

\subsection{Objective}

To teach the student the basics of aircraft systems.

\subsection{Elements}

\begin{itemize}
  \item Primary flight controls and trim
  \item Flaps, leading edge devices, spoilers
  \item Power plant
  \item Oil
  \item Avionics
\end{itemize}

\subsection{Schedule}

Discussion (45 minutes).

\subsection{Equipment}

Aircraft, Pilot Operating Handbook (POH) or FAA-approved Airplane Flight Manual
(AFM).

\subsection{Instructor Actions}

Discuss the components:

\begin{itemize}
  \item Primary flight controls--elevator, rudder, and ailerons
    \begin{itemize}
      \item Movement of the control surfaces changes the airflow and pressure
        distribution over and around the airfoil (relate to CG):
\begin{table}[h]
\centering
\begin{tabular}{l|l|l|l}
Primary control surface & Airplane movement & Axis of rotation & Type of stability \\\hline
aileron                 & roll              & longitudinal     & lateral           \\
elevator                & pitch             & lateral          & longitudinal      \\
rudder                  & yaw               & vertical         & directional
\end{tabular}
\end{table}
    \end{itemize}
  \item Ailerons:
    \begin{itemize}
      \item Control roll about longitudinal axis

      \item Most light airplanes have two ailerons, one on the trailing edge of
        each wing

      \item Connected to control wheel through cables and pulleys

      \item Move in opposite directions

      \item Discuss how ailerons change angle of attack and cause roll

      \item Adverse yaw can be counter acted with rudder use, or special
        aileron designs: differential ailerons, Frise-type ailerons, coupled
        ailerons and rudder

      \item Aileron trim: not common on light airplanes
    \end{itemize}
  \item Elevator
    \begin{itemize}
      \item Controls pitch about lateral axis

      \item Main purpose is to change the wing's angle of attack

      \item Most light airplanes have one elevator, located on the trailing
        edge of the horizontal stabilizer

      \item Some aircraft (e.g. Pipers) use a stabilator, or movable horizontal
        stabilizer

      \item Control wheel connected to the elevator by bell cranks, cables and
        pulleys

      \item Horizontal stabilizer has a negative angle of attack to provide
        downward force

      \item Elevator moves up to increase this downward push and move the nose
        up, and therefore increase the wing's angle of attack, and vice versa

      \item Discuss how elevator movement affects pitch attitude

      \item Elevator trim: almost all light aircraft are equipped with some
        form of elevator trim; moves in the opposite direction of the control
        surface, deflecting the control surface to relieve control pressure and
        maintain a constant pitch attitude
    \end{itemize}

  \item{Rudder}
    \begin{itemize}
      \item Controls the airplane about its vertical axis -- yaw

      \item Most light airplanes have one rudder, located on the trailing edge
        of the vertical stabilizer

      \item Controlled through the use of foot pedals, connected to the rudder
        by bell cranks, cables and pulleys

      \item Rudder does not turn the airplane, only yaws it

      \item Used in conjunction with the ailerons for properly turning the airplane

      \item Rudder trim: most light aircraft are equipped with some form of
        rudder trim; trim tabs move in the opposite direction of the control
        surface, deflecting the control surface to relieve control pressure and
        maintain a constant yaw attitude; some may be a ``manual rudder trim''
        or a piece of metal that is manually adjusted before flight
    \end{itemize}

  \item{Flaps}
    \begin{itemize}
      \item Increase lift and drag

      \item Flaps have three main functions: permit a slower landing speed,
        allow for a steep angle on descent without an increase in airspeed,
        shorten takeoff distance and allow for a steeper climb

      \item Plain--simplest, changes camber, increases lift, greatly increases
        drag

      \item Split--greater increase in lift vs. plain, more drag

      \item Slotted--(most common) increases lift coefficient significantly
        more than plain or split (high-energy air is ducted to the flap's upper
        surface, delaying airflow separation)

      \item Fowler flap--a variety of slotted flap; changes camber and
        increases wing area

      \item Most flaps are located on the trailing edge of the wing in-between
        the fuselage and aileron

      \item In light aircraft they are controlled manually or electrically

      \item Extending the flaps will increase lift, cause a pitch up and loss
        of airspeed

      \item Retracting the flaps will decrease lift cause a pitch down and
        increase in airspeed
    \end{itemize}

  \item Leading edge devices:
    \begin{itemize}
      \item Fixed slots--direct airflow to upper wing surface and delay airflow
        separation; stall is delayed to greater angle of attack

      \item Moveable slats--leading edge segments on tracks; may be automatic or pilot-operated

      \item Leading edge flaps--increase coefficient of lift and camber
    \end{itemize}

  \item Spoilers:
    \begin{itemize}
      \item High-drag device; reduces lift, increases drag

      \item Used for roll control on some aircraft by eliminating adverse yaw

      \item Can shorten ground roll
    \end{itemize}

  \item Power plant:
    \begin{itemize}
        \item Reciprocating engines classified by cylinder arrangement (radial,
          inline, v-type, opposed), method of cooling (liquid or air), method
          of intake (carburetor, fuel-injection, turbo-charged), etc

        \item Main components: cylinders (contain intake/exhaust valves, spark
          plugs, pistons); crankcase (contains crankshaft, connecting rods);
          accessory housing (contains magnetos)

        \item Four-stroke operating cycle: intake, compression, power, exhaust
    \end{itemize}

  \item Oil:
    \begin{itemize}
      \item Lubricates, reduces friction, cools, provides a seal, and carries
        away contaminants

      \item Wet-sump system—sump is an integral part of the engine (in a dry
        system, it's a separate tank)

      \item Filter, cooler, filler cap/dipstick, quick-drain valve (bottom of
        sump)

      \item Pressure and temperature gauges (required instruments)
    \end{itemize}

  \item Avionics:
    \begin{itemize}
      \item Communication and navigation radios

      \item VOR, ADF, GPS

      \item Transponder

      \item Autopilot (if available)

      \item Avionics cooling fan--cools and eliminates moisture (if available)

      \item Microphone/headset intercom

      \item Static dischargers (wicks)
    \end{itemize}
\end{itemize}

\subsection{Evaluation}

Lesson is complete when student can demonstrate and discuss aircraft control
surfaces, power plant and other major systems.

\subsection{References}

FAA-H-8083-25B Pilot's Handbook of Aeronautical Knowledge Chapter 6-7, POH /
AFM Chapter 7.

