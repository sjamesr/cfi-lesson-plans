\section{Weather Theory}

\subsection{Objective}

To teach the student about basic weather theory and how to anticipate possible
weather conditions.

\subsection{Elements}

\begin{itemize}
  \item Atmosphere
  \item Wind
  \item Moisture and stability
  \item Clouds
  \item Fronts
\end{itemize}

\subsection{Schedule}

Discussion (2 hours).

\subsection{Equipment}

FAA-H-8083-25 Pilot's Handbook of Aeronautical Knowledge, AC 00-6B Aviation
Weather.

\subsection{Instructor Actions}

Discuss the following:
\begin{itemize}
  \item Nature of the atmosphere (78\% nitrogen, 21\% oxygen, 1\% other)

  \item Troposphere, the first layer of the atmosphere, contains most of the
    weather, and goes up from 20,000' to 48,000' MSL at the poles

  \item Pressure

  \item The cause of weather: uneven heating by the sun

  \item Convection

  \item How air flows from high (clockwise) to low (counterclockwise, i.e.
    cyclonic) – wind

  \item How oceans and mountains affect wind

  \item Turbulence due to mountains and man-made objects

  \item Moisture, humidity and relative humidity

  \item Stability, inversions

  \item Temperature and dew point

  \item Fog

  \item Cloud formation (vapor in the air, refer to condensation nuclei), types
    of clouds (cumulus, stratus, cirrus, nimbus, etc)

  \item Relate stability to clouds (unstable = cumuliform, stable = stratiform)

  \item Thunderstorms

  \item Cloud cover (few, scattered) and ceilings (broken, overcast)

  \item Precipitation

  \item Air masses

  \item Fronts (boundaries between air masses); flying across a front will lead
    to wind shift and likely some form of weather (always know where the fronts
    are in a long-distance flight)

  \item Warm fronts:
    \begin{itemize}
      \item Prior to passage: cirriform or stratiform clouds, fog, etc, plus
        cumulonimbus in summer; light to moderate precipitation; poor
        visibility; winds from south-southeast

      \item At passage: pressure falling; stratiform clouds; drizzle; poor
        visibility but improving; rising temperature

      \item After passage: stratocumulus clouds; rain showers possible;
        visibility improving but hazy; wind from the south-southwest; slight
        rise in pressure
    \end{itemize}

  \item Cold fronts:
    \begin{itemize}
      \item Prior to passage: cirriform or towering cumulus clouds,
        cumulonimbus also possible; rain showers and haze; wind from
        south-southwest; high dew point; falling pressure

      \item At passage: towering cumulus or cumulonimbus; heavy rain, hail,
        thunderstorms possible; more severe cold fronts produce tornadoes; poor
        visibility; winds variable and gusting; temperature and dew point
        falling rapidly; pressure falling rapidly

      \item After passage: clouds dissipate with corresponding decrease in
        precipitation; good visibility; winds from the west-northwest;
        temperatures remain cooler; pressure begins to rise
    \end{itemize}

  \item Stationary fronts: two air masses holding position for days; the
    weather at the front is usually a mix of warm and cold front weather

  \item Occluded fronts (when a fast-moving cold front catches up to a
    slow-moving warm front):
    \begin{itemize}
      \item Temperatures of the colliding fronts play a large part in the
        weather of occluded fronts

      \item Conditions vary depending on the air mass ahead of the warm front
        being overtaken

      \item Cold front occlusion: the fast-moving cold front is colder than the
        cold mass ahead of the warm front; the weather at this front is usually
        a mix of warm followed by cold front weather and is relatively stable

      \item Warm front occlusion: the fast-moving cold front is less cool than
        the cold mass ahead of the warm front, resulting in the most severe
        weather including embedded thunderstorms, rain, fog, etc
    \end{itemize}
\end{itemize}

\subsection{Evaluation}

Lesson is complete when student can discuss various aspects of weather phenomena.

\subsection{References}

FAA-H-8083-25 Pilot's Handbook of Aeronautical Knowledge, AC 00-6B Aviation
Weather.

