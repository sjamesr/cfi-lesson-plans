
\section{Airport Operations}

\subsection{Objective}

To teach the student with the operations in and around the runway environment.

\subsection{Elements}

\begin{itemize}
  \item Taxi considerations (wind correction)
  \item Airports and airport information
  \item Runway signs and markings
  \item Runway lights
  \item Runway incursion avoidance
  \item Wake turbulence
  \item Lost communications
\end{itemize}

\subsection{Schedule}

Discussion (30 minutes).

\subsection{Equipment}

FAA-H-8083-3A Airplane Flying Handbook, FAA-H-8083-25 Pilot's Handbook of
Aeronautical Knowledge, Airport/Facility Directory, Aeronautical Information
Manual (AIM).

\subsection{Instructor Actions}

Discuss taxing and flight control position during taxi (turn into for
headwinds, dive away from tailwinds). Reference variations for tailwheel
airplanes. Discuss braking. Discuss radio calls.

Discuss types of airports (controlled, uncontrolled). Discuss Airport/Facility
Directory information. Discuss runway markings and signs. Discuss airport
beacons (and what it means if they are on in daytime). Discuss runway lights.

Discuss runway incursions, Land And Hold Short Operations (LAHSO), clearances.

Discuss wake turbulence avoidance. (Land beyond the heavier aircraft's
touchdown, or take off before the heavier aircraft's rotation point; wingtip
vortices the most dangerous from a clean, heavy, slow airplane).

Discuss lost communications procedures (light gun signals). (14 CFR 91.125) Be
sure to use 7600 on the transponder.

\subsection{Evaluation}

Lesson is complete when student can demonstrate and comprehend runway signage,
lighting and exhibit good judgment in wake turbulence avoidance.

\subsection{References}

FAA-H-8083-3A Airplane Flying Handbook p. 2-9, FAA-H-8083-25 Pilot's Handbook
of Aeronautical Knowledge Chapter 12, Aeronautical Information Manual 7-3.

