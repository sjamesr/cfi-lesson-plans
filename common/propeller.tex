
\chapter{Propeller System}

\section{Objective}

To familiarize the student with common propeller systems.

\section{Elements}

\begin{itemize}
  \item Propeller basics
  \item Pre-flight
  \item Constant-speed propellers
\end{itemize}

\section{Schedule}

Discussion (45 minutes).

\section{Equipment}

Aircraft, Pilot Operating Handbook (POH) or FAA-approved Airplane Flight Manual
(AFM).

\section{Instructor Actions}

Discuss the two main types of propellers: fixed pitch and constant speed.
Others include full-feathering, reversing, and ground-adjustable.

Discuss basics of propellers: materials (typically wood, composite or aluminum
alloy).

The hub is the center of the propeller, and the spinner mounts above and covers
the hub. Propeller blades are airfoils, and have camber like any airfoil.

Blade twist: the hub has less pitch than the tips. That is, the pitch on each
blade changes the further you go from the hub. This is necessary to give the
relative same thrust across the blade despite the increased speed as you move
out towards the tips.

Discuss pre-flight of propellers: if a nick exists from one side to the other
side at the propeller's edge, or if any is, in your judgment, too significant
(recommended: greater than a quarter inch or greater than two tenths an inch
deep), it must be looked at by a mechanic. Nicks in the last third of the
propeller are extremely worrisome. This is because the outside of the propeller
travels faster than the inside, leading to higher stresses in the metal, which
can cause fatigue at the nick. At 2500 RPM, a typical single-engine
propeller’s tips travel at nearly 650 knots. Nicks are shaved off, which is
why blades have a tolerance for how long they can be. If one blade is shaved
down, the other(s) must be as well to remain balanced. Nicks that are not
dealt with can grow to bigger ones. The worse-case scenario is propeller
separation during flight, shearing the blade and causing such imbalance that
the engine is then sheared from the airplane. Other pre-flight considerations:
corrosion (painted propellers prevents corrosion), leaking oil at the hub, etc.
Also try to twist constant speed propellers (hold them at their midpoint, not
their tips as this can cause damage), they should resist rotation and not
rotate more than a quarter inch. The blades should also not move forward or aft
more than a quarter inch.

Propellers are subject to periodic inspections based on tachometer and on
calendar months. Consult the logbooks to see the propeller inspections.

Discuss the mechanics of the constant speed propeller and the governor system.
Discuss the flyweight system, speeder spring, and how these mechanisms control
high-pressure oil to the propeller hub. Discuss how the propeller hub's
internal piston moves the blades' pitch against the propeller spring. Discuss
loss of engine oil or governor, and the resulting safe condition of high-RPM
low-pitch. Discuss which settings are ideal for take-off and landing, and
consult the AFM/POH for more discussion.

\section{Evaluation}

Lesson is complete when student can discuss a constant speed propeller system.

\section{References}

%% TODO(sjr): these links are broken
\begin{itemize}
  \item \url{http://www.mccauley.textron.com/prop/prop-tech/pg00intro.html}

  \item \url{http://flash.aopa.org/asf/engine_prop/}
\end{itemize}

