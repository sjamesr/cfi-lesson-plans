\section{Spin Awareness and Avoidance}

\subsection{Objective}

To teach the student the avoidance and proper recovery from spins.

\subsection{Elements}

\begin{itemize}
  \item Uncoordinated stalls
  \item Aerodynamics of a spin
  \item Recovery procedure
\end{itemize}

\subsection{Schedule}

Discussion (30 minutes).

\subsection{Equipment}

Model airplane.

\subsection{Instructor Actions}

Discuss what is a spin (an aggravated stall that results in autorotation).
Autorotation results from unequal angles of attack on the wings. The key is
aggravated (i.e. uncoordinated). Draw or show the corkscrew/helical flight path
of a spin. The difference between a spin and a steep spiral: spin--airspeed low,
wings stalled; spiral--airspeed increasing, not stalled.

Discuss the aerodynamics of a spin. Draw a wing in straight-and-level flight
and in slow flight. Use actual angles of attack. Typical light aircraft wings
stall at $18-22^{\circ}$ How can you enter a spin? Wing exceeds critical angle
of attack with yaw acting on aircraft (uncoordinated). That is, a stall when in
a slipping or skidding turn. Danger of base to final turn--cross controlled
stall leading to spin.

The high wing has the greatest lift due to the greater airspeed, and overall
less drag and lower angle of attack. The low wing has the least lift (due to
lower airspeed) and greatest parasitic drag due to its higher angle of attack.

Center of gravity affects the spin characteristics. An aft CG makes spin
recovery more difficult. The worst case is the aircraft may enter into a flat
spin if CG is too far back, making recovery impossible.

Phases of a spin:
\begin{itemize}
  \item Entry--pilot provides input for the spin
  \item Incipient--aircraft stalls, rotation starts to develop; may take 2
    turns in most aircraft, usually 5-6 seconds
  \item Developed--rotation rate, airspeed, vertical speed are constant,
    typically 500 fpm altitude loss, just above 1G load factor; airspeed at or
    below stall speed; rotation is around all 3 axes, as high as 7000 fpm
    descent
  \item Recovery--lower angle of attack below the critical value, may take from a
    quarter to several turns to recover
\end{itemize}

The recovery process is as follows (``PARE''):
\begin{itemize}
  \item \textbf{P}ower -- reduce to idle
  \item \textbf{A}ilerons -- position to neutral
  \item \textbf{R}udder -- full opposite against the rotation
  \item \textbf{E}levator -- brisk elevator control full forward to break stall
  \item \textbf{After spin rotation stops, neutralize the rudder}
  \item \textbf{Smoothly apply back-elevator pressure to raise the nose to
    level flight}
\end{itemize}

How do you know which direction you're spinning? Use the turn coordinator. The
inclinometer ball is unreliable, and the attitude indicator will likely tumble.
Also, look at top of windshield to see the horizon.

Avoid spins by maintaining coordination during practice stalls, and avoid
stalls other than practicing.

\subsection{Evaluation}

Lesson is complete when student can recite recovery procedure and can identify
hazardous situations that may lead to a spin.

\subsection{References}

FAA-H-8083-3B Airplane Flying Handbook p. 5-36.

