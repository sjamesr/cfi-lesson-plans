\section{Aerodynamics}

\subsection{Objective}

To teach basic aerodynamic principles.

\subsection{Elements}

\begin{itemize}
  \item Four forces
  \item Airfoils
  \item Drag
  \item Stability and controllability
  \item Turning tendencies
  \item Climbs, descents, and turns
  \item Load factors
  \item Ground effect
  \item Adverse yaw
  \item Wingtip vortices
\end{itemize}

\subsection{Schedule}

Discussion (1 hour).

\subsection{Equipment}

Model aircraft.

\subsection{Instructor Actions}

Discuss the following:

\begin{itemize}
  \item Aerodynamics is the branch of physics that deals with the motion of a
    solid body through fluids.

  \item Four forces: lift, weight, thrust, drag

  \item Newton's laws of motion: first: an object in motion stays in motion,
    second: force equals mass times acceleration, third: for every action there
    is an equal and opposite reaction.

  \item In steady flight, the sum of these opposing forces is equal to zero.
    (Newton's first law) Lift = weight, thrust = drag.

  \item Bernoulli's principle: an increase in the speed of the fluid occurs
    simultaneously with a decrease in pressure.

  \item An airfoil is a structure designed to obtain reaction upon its surface
    from the air through which it moves or that moves past it. (Bernoulli)
\end{itemize}

Discuss the wing definitions: leading edge, trailing edge, camber, chord line.

\begin{itemize}
  \item Aspect ratio--wingspan to mean chord line
    \begin{itemize}
      \item High aspect ratio = more lift, less drag.
    \end{itemize}

  \item Angle of incidence--angle between the chord and the longitudinal axis

  \item Angle of attack (AoA)

  \item Wing always stalls at the critical angle of attack
\end{itemize}

Drag:

\begin{itemize}
  \item Parasite: increases as the square of the airspeed
    \begin{itemize}
      \item Form drag: shape of the aircraft, i.e. streamlined object

      \item Skin friction: surface finish

      \item Interference: if two objects are placed adjacent to one another, the
        resulting turbulence produced may be 50 to 200\% greater than the parts
        tested separately
    \end{itemize}
  \item Induced-byproduct of lift (horizontal component of lift), varies
    inversely as the square of the airspeed, also caused by downwash from
    wingtip vortices
\end{itemize}

Discuss the $L/D_{max}$ drag graph. (See figure 3-5 in PHAK)

Discuss stability (see PHAK 3-10) (Use marble and bowl analogy):

\begin{itemize}
  \item Stability is the inherent quality of an airplane to correct for
    conditions that may disturb its equilibrium, and to return or to continue
    on the original flightpath
  \item Static:
    \begin{itemize}
      \item Static stability is the initial tendency of an object to return to
        its original position after being disturbed

      \item Positive, neutral, negative

      \item Push controls: positive returns to original state; neutral: remains
        at new state; negative: keeps moving beyond
    \end{itemize}
  \item Dynamic:
    \begin{itemize}
      \item Tendency after equilibrium is disturbed (stability in motion)

      \item Positive, neutral, negative

      \item Positive: dampened, pattern of movements become smaller; neutral:
        pattern continues unchanged; negative: pattern diverges
    \end{itemize}
\end{itemize}

Discuss axes of rotation: longitudinal (roll/ailerons), lateral
(pitch/elevator), vertical (yaw/rudder).

Longitudinal stability (pitch):

\begin{itemize}
  \item Most affected by pilot, especially aircraft loading

  \item Draw CL, CG, horizontal stabilizer down force

  \item Horizontal stabilizer is an upside-down wing, provides longitudinal
    stability with a down force

  \item Nose moves up (tail down), horizontal stabilizer's AoA decreases,
    reduces lift and the nose moves back down

  \item Nose moves down (tail up), horizontal stabilizer's AoA increases,
    produces more lift and the pushes the nose moves back up

  \item Unstable if CG moves too close to or behind the center of lift
\end{itemize}

Lateral stability (roll)

\begin{itemize}
  \item Most common design factor for positive stability is wing dihedral (see
    PHAK Fig. 3-17)

  \item Positive dihedral is when the wing tips are higher than the wing root

  \item Wing position--high wing aircraft are more laterally stable than low
    wings (everything held constant)

  \item High wing aircraft require less dihedral than low wing aircraft
\end{itemize}

Vertical stability (yaw)

\begin{itemize}
  \item Vertical stabilizer, and the fuselage behind the CG to a lesser extent,
    provide directional stability

  \item Vertical stabilizer acts like the tail feathers on an arrow

  \item Swept back wings, and double taper wings to a lesser extent, also
    provide directional stability

  \item Increased induced drag on the wing that is moved forward pulls it back,
    causes Dutch roll tendency
\end{itemize}

Controllability versus Maneuverability (PHAK 3-10)

\begin{itemize}
  \item Controllability--capability of an aircraft to respond to the pilot's
    control inputs, especially with regard to flightpath and attitude

  \item Maneuverability--quality of an airplane that allows it to be easily
    controlled and maneuvered and withstand stresses imposed by maneuvers

  \item An F-16 sacrifices stability for controllability, maneuverability
\end{itemize}

Turning tendencies (PHAK 3-23)

\begin{itemize}
  \item Torque--opposite reaction to the engine (Newton's third law): leads to a
    left roll

  \item Spiraling slipstream--pushes on port side of vertical stabilizer,
    causing left yaw

  \item Gyroscopic action (precession)--a pitch down will cause a left yaw 90º
    from the top of the propeller, a pitch up will cause a right yaw 90º from
    the bottom of the propeller disk

  \item P-factor--downward blade has a higher angle of attack during a high
    pitch attitude, causing greater thrust on the right side of the propeller
    disk, creating a left yaw (and vice versa)
\end{itemize}

Climbs: Initially lift is greater, then it stabilizes in steady-state.

Descents: Initially lift is reduced, then it stabilizes in steady-state.

Turns: Show a vector diagram with horizontal component of lift.

Load factor: Any force applied to an airplane to deflect its flight from a
straight line. Ratio of total load acting on the airplane to the gross weight.
Discuss maneuvering speed $V_{a}$.

Limit load factors: Normal 3.8 to -1.52, Utility 4.4 to -1.76, Aerobatic 6.0 to
-3.0. 1.5 factor of safety built in. See load factor chart (PHAK Fig. 3-36),
noting $60^{\circ}$ of bank equals $2g$. Discuss $V_{g}$ diagram (PHAK Fig.
3-38).

Ground effect (PHAK 3-7):

\begin{itemize}
  \item Within one wingspan of earth's surface, wingtip vortices are reduced

  \item Provides decrease in induced drag

  \item Wing will require a lower angle of attack in ground effect to produce
    the same lift
\end{itemize}

Adverse yaw (PHAK 4-2):

\begin{itemize}
  \item Purpose of the rudder is to counteract adverse yaw and control movement
    around the vertical axis

  \item Raised wing in a turn has a higher angle of attack, more lift, more
    induced drag

  \item Differential ailerons, Frise-type ailerons, and coupled ailerons and
    rudder reduce adverse yaw
\end{itemize}


Wingtip vortices (PHAK 3-6):

\begin{itemize}
    \item Spanwise movement of air along wing due to pressure differential

    \item Air from bottom of wing moves outward from fuselage and ``spills''
      over the wingtips, creating a vortex

    \item Air from the top of the wing flows in toward the fuselage and spills
      off the trailing edge--vortex is insignificant because fuselage limits the
      flow

    \item Vortices increase drag because of energy spent in producing turbulence

    \item High angle of attack = more violent vortices

    \item Heavier and slower aircraft = more violent vortices
\end{itemize}

Wake turbulence on take-off/landing:

\begin{itemize}
  \item Stay above glidepath

  \item Land beyond the point of landing of the preceding heavier aircraft--look
    for puffs of tire smoke

  \item Liftoff prior to the point a larger aircraft took off

  \item Light quartering tailwind keeps vortices on the runway the longest (the
    most dangerous)
\end{itemize}

\subsection{Evaluation}

Lesson is complete when student can demonstrate fundamental understanding of
aerodynamics.

\subsection{References}

FAA-H-8083-25 Pilot's Handbook of Aeronautical Knowledge (PHAK) Chapter 3,
FAA-H-8083-3A Airplane Flying Handbook Chapter 3, Aeronautical Information
Manual (AIM) 7-3-1.

