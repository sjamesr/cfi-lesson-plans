\chapter{Flight Illusions and Vision of Flight}

\section{Objective}

To familiarize the student with currency requirements, health requirements, medical requirements, etc.

\section{Elements}

\begin{itemize}
  \item VFR flight illusions
  \item Vision of flight considerations
\end{itemize}

\section{Schedule}

Discussion (30 minutes).

\section{Equipment}

14 CFR (FAR/AIM).

\section{Instructor Actions}

Discuss the following:
\begin{itemize}
  \item Illusions Leading to Landing Errors
    \begin{itemize}
      \item False horizon -- clouds or at night, lights on terrain, gives the
        sense of a horizon other than the actual horizon
      \item Autokenesis -- staring at a point of light at night, the point will
        appear to dance around
      \item Runway width illusions -- narrow-than-usual runways give the
        illusion the aircraft is higher than actual, and vice versa
      \item Runway and terrain slopes illusion -- an upsloping runway, terrain,
        or both, give the illusion the aircraft is higher than actual, and vice
        versa
      \item Featureless terrain illusion -- lack of terrain features give the
        illusion that an aircraft is higher than it is
      \item Atmospheric illusions -- rain on the windscreen can create the
        illusion of greater height, haze can give illusion of greater distance
      \item Ground lighting illusions -- lights along a straight path can be
        mistaken for a runway
    \end{itemize}
  \item Vision of Flight
    \begin{itemize}
      \item In dim light, eyes require 30 minutes to adjust
      \item Consider red cockpit lighting for optimum night vision
      \item If lightening nearby, adjust the cockpit lights to brightest as
        your eyes will be quickly adjusted to bright light
      \item In VFR: for bright days, wear sunglasses that absorb 85\% visible
        light
      \item In VFR: scanning for other aircraft: from left to right, 10º for at
        least 1 second
      \item Due to the increased use of oxygen by the human eye in dark
        settings, night vision can be impaired by the less oxygen at as low as
        5000'
    \end{itemize}
\end{itemize}

\section{Evaluation}

Ensure understanding of illusions of VFR flight and vision of flight.

\section{References}

AIM Chapter 8.

