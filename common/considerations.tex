\section{Piloting considerations}

\subsection{Objective}

To familiarize the student with currency requirements, health requirements,
medical requirements, etc.

\subsection{Elements}

\begin{itemize}
  \item Health (``I'M SAFE'')
  \item Medical requirements
  \item Aeromedical physiology
  \item Currency requirements
  \item Log books
\end{itemize}

\subsection{Schedule}

Discussion (30 minutes).

\subsection{Equipment}

14 CFR (FAR/AIM).

\subsection{Instructor Actions}

Discuss the following regulations and requirements:

\begin{itemize}
  \item Health considerations: ``I'M SAFE'': Illness (14 CFR 61.53, 91.17),
    Medication (14 CFR 91.17; the best list is at AOPA's members section or
    \url{http://www.leftseat.com/medcat1.htm}; FAA has no official list),
    Stress, Alcohol (14 CFR 91.17; none within 8 hrs, $<$ 0.04\% BAC), Fatigue,
    Emotion.

  \item Medical requirements: (14 CFR 61.23) 3rd class medical lasts 36
    calendar months ($<$ age 40) or 24 calendar months ($>$ age 40). A current
    medical is required to exercise the privileges allowed by the pilot
    certificates held.

  \item Aeromedical physiology: (chapter 8 of the AIM) hypoxia, ear block,
    sinus block, decompression sickness, hyperventilation, carbon monoxide
    poisoning and disorientation.

  \item Currency requirements: (14 CFR 61.57) 3 take-offs and landings every 90
    days for daytime, 3 full-stop take-offs and landings every 90 days for
    nighttime to fly with passengers. Also, a biennial flight review (14 CFR
    61.56) consisting of 1 hour of ground that at least covers 14 CFR 91 and 1
    hour of flight maneuvers as deemed adequate by the instructor. The pilot
    certificate (other than student certificate) lasts indefinitely (14 CFR
    61.19). Moving requires an update sent to the FAA within 30 days (14 CFR
    61.60).

  \item Log book requirements: (14 CFR 61.51) Only requirement is training and
    experience required to obtain certificates, ratings, a flight review, or
    currency requirements. Expand on how to log time and what to log.

\end{itemize}

\subsection{Evaluation}

Ensure understanding of piloting considerations, currency requirements, etc.

\subsection{References}

AIM Chapter 8, 14 CFR 61 and 91.

