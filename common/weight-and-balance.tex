\section{Weight and Balance}

\subsection{Objective}

To teach the student aircraft weight and balance considerations.

\subsection{Elements}

\begin{itemize}
  \item Weight and balance definitions
  \item Effects of greater weight
  \item Effects of CG location
  \item Discuss how to calculate weight and balance
\end{itemize}

\subsection{Schedule}

Discussion (1 hour).

\subsection{Equipment}

Aircraft, airplane Pilot Operating Handbook or FAA-approved Airplane Flight
Manual.

\subsection{Instructor Actions}

Discuss weight and balance limitations (max weights determined by structural
strength and performance). Discuss CG and the CG envelope. Define center of
pressure (lift) and its relation to angle of attack (center of lift moves
forward with higher AoA). CG is always ahead of center of pressure (otherwise
an airplane would tumble). Compare lift on the wing, CG, and lift generated
from the horizontal stabilizer.

Define weight definitions (empty weight, payload, zero fuel weight, fuel load,
useful load). Define standard weights: Fuel (Avgas) 1 gallon = 6 lbs, Oil 1
gallon = 7.5 lbs (approx. 2 lbs per quart). Define moment (tendency to rotate),
arm (distance at which a force is applied), station (arm, measured in reference
to datum on aircraft).

Discuss the effects of greater weight (higher take-off speeds, longer take-off
run, reduced rate and angle of climb, lower maximum altitude, shorter range,
reduced cruising speed, reduced maneuverability, higher stalling speed, higher
approach and landing speed, longer landing roll, excessive weight on
nose/tailwheel).

Discuss effects of CG location. Discuss how this can shift due to weight shift
during flight, fuel burn, etc. Forward CG: nose heavy, increased take-off and
landing speed, higher stall speed, good stall recovery, higher angle of attack,
less range and endurance, increased stability. Aft CG: tail heavy, decreased
take-off and landing speed, lower stall speed, poor stall recovery, smaller
angle of attack, greater range and endurance, decreased stability.

Discuss calculations using examples from the POH. Discuss the ``WAM'' equation:
$W \cdot A = M$. Weight times Arm equals Moment. Discuss how this simple setup
can be used for weight shift, weight change, as well as standard CG
calculations.  Discuss how lateral CG is not computed but can cause wing
heaviness.

\subsection{Student Actions}

Calculate weight and balance under a variety of scenarios or as directed.

\subsection{Evaluation}

Lesson is complete when student can demonstrate and calculate weight and balance.

\subsection{References}

FAA-H-8083-25 Pilot's Handbook of Aeronautical Knowledge Chapter 8.

