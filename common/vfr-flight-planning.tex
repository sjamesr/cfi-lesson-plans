\chapter{VFR Flight Planning}

\section{Objective}

To teach judgment and considerations when planning a VFR cross-country flight.

\section{Elements}

\begin{itemize}
  \item Weather considerations
  \item Sectional charts and the A/FD
  \item Airspace and visibility requirements
  \item Terrain, altitude, cruising requirements and right-of-way rules
  \item POH performance charts and density altitude
  \item Fuel requirements
  \item Pilotage and Dead Reckoning
  \item Plotting a flight and using a navigational log
  \item True courses, magnetic headings and wind
  \item E6-B
  \item Discuss advanced navigation methods (in brief)
  \item FAA Flight Plans
  \item Flight Following
  \item Lost procedures
  \item Aeronautical Decision Making
\end{itemize}

\section{Schedule}

\begin{itemize}
  \item Discussion (3 hours)
  \item Pre-flight instruction (40 minutes)
  \item VFR Cross-Country Flight (as needed)
  \item Post-flight instruction (20 minutes)
\end{itemize}

\section{Equipment}

14 CFR and Aeronautical Information Manual (FAR/AIM), Sectional charts,
Airport/Facility Directory, FAA-H-8083-25 Pilot's Handbook of Aeronautical
Knowledge, E6-B circular slide rule, navigation log, chart plotter, aircraft
pilot operating handbook (POH).

\section{Instructor Actions}

Discuss weather requirements, how to obtain weather, weather considerations.
Discuss basic weather minimums (14 CFR 91.155). (A separate lesson plan
discusses weather theory and services in detail).

Discuss sectional charts and their legend in detail. Discuss the
airport/facility directory.

Discuss airspace and visibility requirements. (A separate lesson plan discusses
airspace in detail.)

Discuss terrain considerations (and how this relates to POH and performance
charts), altitude requirements (14 CFR 91.119), VFR altitudes (14 CFR 91.159),
altimeter settings (14 CFR 91.121) and right-of-way rules (14 CFR 91.113).
Discuss when to safely descend before arriving at an airport environment.

Discuss performance charts from the POH and density altitude.

Discuss fuel requirements (14 CFR 91.151): 30 minutes extra fuel for day VFR,
45 minutes for night VFR. Discuss the definition of night (Pilot/Controller
Glossary). Relate to POH performance charts.

Discuss navigation methods: pilotage (purely visual, following landmarks), dead
reckoning (navigation using compass headings and checkpoints). (See Chapter 14
of FAA-H-8083-25 Pilot's Handbook of Aeronautical Knowledge.) Discuss latitude
and longitude.

Discuss how to use a plotter with a chart. Introduce the navigation log.

Discuss true courses, magnetic headings, variation and deviation. Discuss wind
correction.

Discuss E6-B circular slide rule, and in particular, making wind calculations.

Briefly discuss other means of navigation, including VOR and GPS (and possibly
NDB). This should be a separate lesson and should not be incorporated into the
initial cross-country flights.

Discuss the FAA Flight Plan (14 CFR 91.153, AIM 5-1-4), how to file it, how to
complete it, how to close it, and results if it isn’t closed. Discuss changes
in a Flight Plan (AIM 5-1-14). Discuss Zulu (GMT) time.

Discuss Flight Following and ATC communications in cross-country flight.

Discuss lost procedures.

Discuss Aeronautical Decision Making (ADM) (see Chapter 16 of FAA-H-8083-25
Pilot's Handbook of Aeronautical Knowledge).

Discuss ``Aviate, Navigate, Communicate''.

\section{Student Actions}

Exhibit knowledge on all aspects of flight planning. Plan then fly a dual VFR
cross-country flight.

\section{Evaluation}

Evaluate based on planning and success of an actual VFR dual cross-country
flight per the requirements listened in 14 CFR 61.109.

\section{Common Errors}

\begin{itemize}
  \item Failure to plan appropriately based on the weather
  \item Failure to create an adequate navigation plan that considers fuel,
    winds, terrain, etc
  \item Failure to have all necessary publications available during flight
  \item Failure to understand the complexities of the airspace system
\end{itemize}

\section{References}

AIM, 14 CFR 91, FAA-H-8083-25 Pilot's Handbook of Aeronautical Knowledge
Chapter 14 and 16.

