\section{Aircraft Documents and Maintenance}

\subsection{Objectives}

To familiarize the student about the required documents needed to legally
operate an aircraft as well as the maintenance required to keep the aircraft in
a legal status.

\subsection{Elements}

\begin{itemize}
  \item Certification and documentation required
  \item Maintenance and inspections
  \item Preventative maintenance
\end{itemize}

\subsection{Schedule}

Discussion (45 minutes).

\subsection{Equipment}

Aircraft, Aircraft maintenance records.

\subsection{Instructor Actions}

The pilot in command ultimately decides if the aircraft is airworthy (14 CFR
91.7).

Discuss required documents ``AROW'': \textbf{A}irworthiness certificate (14 CFR
91.203), \textbf{R}egistration certificate (14 CFR 91.203), \textbf{O}perating
manual and placards (FAA-approved Airplane Flight Manual, serial number
specific) (14 CFR 91.9), \textbf{W}eight and balance, current and specific for
the exact aircraft serial number (part of the AFM, covered under 14 CFR 91.9).

\begin{table}[h]
\centering
\begin{tabular}{ll}
A & \textbf{A}nnual Inspection (12 calendar months) (14 CFR 91.409)        \\
V & \textbf{V}OR (30 days for IFR, annual for VFR) (14 CFR 91.171)         \\
1 & Inspection at \textbf{1}00 Hours off of tachometer (14 CFR 91.409)     \\
A & \textbf{A}ltimeter/Pitot Static System (24 calendar months) 91.411)    \\
T & \textbf{T}ransponder/Mode C (24 calendar months) (14 CFR 91.413)       \\
E & \textbf{E}LT (50\% battery life/1 hr cum. use, annual) (14 CFR 91.207) \\
A & \textbf{A}irworthiness Directives (ADs, aka Recalls) (14 CFR 91.403),
\end{tabular}
\end{table}

ADs may be divided into two categories:
\begin{itemize}
  \item those of an emergency nature requiring immediate compliance prior to
    further flight, and

  \item those of a less urgent nature requiring compliance within a specified
    period of time
\end{itemize}

Preventative maintenance can be accomplished by any certificated pilot (14 CFR
43 Appendix A). Repairs must be completed by FAA-certified mechanics. If they
are major or minor repairs, as defined by 14 CFR 43 Appendix A, defines what
level of oversight and qualification is required of the mechanic or repair
facility.

A special flight permit is a Special Airworthiness Certificate issued
authorizing operation of an aircraft that does not currently meet applicable
airworthiness requirements but is safe for a specific flight (e.g. to a repair
facility).

\subsection{Evaluation}

Lesson is complete when student can demonstrate and comprehend emergency
considerations.

\subsection{References}

FAA-H-8083-25 Pilot's Handbook of Aeronautical Knowledge Chapter 7.

