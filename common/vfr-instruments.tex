\chapter{VFR Instruments}

\section{Objective}

To teach the student about VFR instruments, their function, and the
requirements.

\section{Elements}

\begin{itemize}
  \item VFR required instruments
  \item Vacuum-driven gyroscopic instruments
  \item Electric-driven gyroscopic instruments
  \item Pitot-static system
  \item Compass errors
\end{itemize}

\section{Schedule}

Discussion (1h30m).

\section{Equipment}

Aircraft.

\section{Instructor Actions}

Explain required instruments (``A-GOOSEACAT'': Anti-collision lights, Gas
gauges, Oil pressure gauge, Oil temperature gauge, Seatbelts with should
harnesses, ELT, Airspeed indicator, Compass, Altimeter, Tachometer for each
engine), plus night requirements (``APES'': Anti-collision lights, Position
lights, Electrical source, Spare fuses as required). For complex aircraft:
landing gear position indicators. For commercial flights: landing light. Other
requirements: manifold pressure gauge for each altitude engine and temperature
gauge for each liquid-cooled engine. Source: 14 CFR 91.205.

Explain Vacuum-Driven Gyroscopic Instruments:

Attitude Indicator: Explain construction, demonstrate and explain behavior and
indications.

Directional Gryo (Heading Indicator): Explain construction, demonstrate and
explain behavior, indications, tick marks (i.e. $45^{\circ}$ tick marks),
precession (check with magnetic compass every 15 minutes). Note: Heading
indicator is always the primary instrument for bank.

Explain Electric Gyroscopic Instruments:

Turn Coordinator (with Inclinometer): Explain needle and ball construction.
Demonstrate and explain needle and ball behavior under all conditions. Discuss
and explain needle and ball indications.

Explain Pitot-Static System:

Airspeed Indicator: Explain construction, demonstrate and explain behavior,
indications (V-Speeds), types of airspeed, errors (pitot tube at high pitch
attitude). In straight and level flight the airspeed indicator is the primary
power instrument. In climbs and descents at a specific airspeed, the airspeed
indicator is the primary pitch instrument.

Altimeter: Explain construction, demonstrate and explain behavior, types of
altitudes, errors (``High to Low or Hot to Cold, Look Out Below!'') In straight
and level flight the altimeter is the primary pitch instrument. Altimeter
should be within 75' of field elevation.

Vertical Speed Indicator: Explain construction, demonstrate and explain
behavior, rate information (vertical speed) versus trend information (changes
of vertical speed), 6-9 second lag. During constant rate climbs and descents,
the vertical speed indicator is the primary pitch instrument.

Explain Pitot-Static System Blockages:
\begin{itemize}
  \item Complete blockage (Pitot tube and drain, static ports): airspeed and
    altimeter will stay constant and VSI will indicate zero

  \item Pitot tube complete blockage (static port open): altimeter and VSI will
    indicate correctly but airspeed will react like an altimeter

  \item Pitot tube blocked, drain clear (static port open): altimeter and VSI
    will indicate correctly but airspeed will decrease to zero

  \item Static port blocked (only): airspeed continues to operate but will be
    erroneous; at higher altitude than when the blockage occurred, airspeed
    will show slower, and vice versa
\end{itemize}

Discuss magnetic compass errors. Explain construction, demonstrate and explain
behavior, variation (magnetic versus true north), deviation, magnetic dip
errors, Northerly Turning Error (``Lag from the North, Lead from the South'';
When determining to lag or lead, remember OSUN: ``Overshoot when turning to
South, undershoot when turning to North''), Acceleration Errors (On an east or
west heading, ANDS: Accelerate turns to the North, Decelerate turns to the
South).

\section{Evaluation}

Lesson is complete when student has a thorough knowledge of required
instruments and compass errors.

\section{References}

FAA-H-8083-15A Instrument Flying Handbook Chapter 3.

