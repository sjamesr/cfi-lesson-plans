\section{Weather Services}

\subsection{Objective}

To teach sources of weather information.

\subsection{Elements}

\begin{itemize}
  \item Observations
  \item Forecasts
  \item Weather products
  \item Briefings
\end{itemize}

\subsection{Schedule}

Discussion (2 hours).

\subsection{Equipment}

AC 00-6A Aviation Weather, Sectional or Terminal chart.

\subsection{Instructor Actions}

Discuss the following, and in the case of each weather source, go through real examples.

Preflight actions: 14 CFR 91.103 requires that the PIC shall become familiar
with all available information concerning that flight including weather
information when flying beyond the vicinity of an airport.

To contact FSS: 1-800-WX-BRIEF, Duat, Duats, Visit in person, 122.2 or discrete
frequency (see sectional chart). A FSS standard briefing gives complete and
customized description of all conditions that may affect the proposed flight,
based on route and altitude to by flown. Includes adverse conditions, synopsis,
current and forecast conditions, winds and temps aloft, NOTAM’s, anything else
requested.

Briefing formats:

\begin{table}[h]
\centering
\begin{tabular}{llll}
Time                                    & Visualize        & Compare          & Briefing     \\\hline
6+ hours before departure               & big picture      & N/A              & outlook      \\
1-4 hours before departure              & detailed picture & big picture      & standard     \\
just before departure and during flight & updated picture  & detailed picture & abbrieviated
\end{tabular}
\end{table}

Current weather products (observations):

\begin{itemize}
  \item Satellite Weather picture:
    \begin{itemize}
      \item graphically display cloud position and approx. thickness and height
      \item issued every 30 minutes and as needed
      \item valid at time of report
    \end{itemize}
  \item Radar Summary Chart:
    \begin{itemize}
      \item graphically display areas of precipitation—not clouds
      \item issued hourly and as needed
      \item valid at time of report
      \item contours indicate intensity of precipitation
    \end{itemize}
  \item Weather Depiction Chart:
    \begin{itemize}
      \item same as significant weather chart
      \item issued every 3 hours
      \item valid at time of report
    \end{itemize}
  \item Freezing Level Chart
  \item Aviation Routine Weather Report (METAR) (AIM 7-1-30):
    \begin{itemize}
      \item aviation routine weather report at the surface, issued hourly
      \item SPECI issued when certain significant changes occur
    \end{itemize}
  \item Pilot Weather Report PIREP's (UA/UUA):
    \begin{itemize}
      \item provide information on actual flight conditions as experienced by
        pilots
      \item issued upon receipt
      \item valid at time of report
      \item UA = normal, UUA = urgent
    \end{itemize}
\end{itemize}

Forecast weather products:

\begin{itemize}
  \item AIRMET (WA):
    \begin{itemize}
      \item Airman's meteorological information
      \item Moderate icing (zulu AIRMET), turbulence (tango AIRMET), IFR
        (sierra AIRMET), mountain obscuration, sustained surface winds $>$30
        knots
    \end{itemize}
  \item SIGMET (WS):
    \begin{itemize}
      \item significant meteorological information
      \item severe turbulence, icing, widespread duststorms, sandstorms,
        volcanic ash lowering visibility to $<$3 SM
      \item valid 4 hours (except 6 hours for hurricanes)
    \end{itemize}
  \item Convective SIGMET (WST):
    \begin{itemize}
      \item severe thunderstorms with surface winds greater than 50 knots, hail
        at the surface greater than or equal to 3/4 inch in diameter,
        tornadoes, embedded thunderstorms, lines of thunderstorms, or
        thunderstorms with heavy or greater precipitation
      \item valid 2 hours
    \end{itemize}
  \item Prognostic Charts:
    \begin{itemize}
      \item 24-, 36-, and 72-hour formats
      \item graphically display general weather conditions for contiguous U.S.
      \item 4 times daily
      \item left panel valid 12 hours, right panel valid 24 hours
      \item types and positions of fronts and pressure systems
      \item pressures in millibars.
      \item significant weather chart shows areas of VFR, MVFR, IFR, freezing
        level
    \end{itemize}
  \item Convective Outlook Chart
    %% TODO(sjr): replace this with Graphical Area Forecast, FA's are gone
  \item Area Forecast (FA):
    \begin{itemize}
      \item describes forecast weather conditions for several states
      \item issued 3 times daily
      \item synopsis valid 18 hours, weather and clouds 12 hours, +6 hr outlook
    \end{itemize}
  \item Terminal Aerodrome Forecast (TAF):
    \begin{itemize}
      \item describes forecast weather for an area within 5sm of airport
      \item issued 4 times daily
      \item valid 24 hours
    \end{itemize}
  \item Winds and Temperatures Aloft Forecast (FD):
    \begin{itemize}
      \item provide estimated wind direction, speed, and temperatures at
        selected stations and altitudes
      \item issued 2 times daily
      \item valid as stated—6, 12, or 24 hours
      \item winds greater than 100, subtract 50 from wind direction
    \end{itemize}
\end{itemize}

In-flight weather services:

\begin{itemize}
  \item Automatic Terminal Information System ATIS:
    \begin{itemize}
      \item airport name, time (UTC), wind direction and speed, visibility and
        obstructions, cloud coverage, temp and dew point, altimeter, remarks.
      \item different from AWOS/ASOS in that ATIS is usually only issued hourly
        and includes NOTAM's
    \end{itemize}
  \item AWOS, ASOS:
    \begin{itemize}
      \item AWOS—automated weather observation system (AIM 4-3-26, 7-1-12)
      \item ASOS—automated surface observation system
    \end{itemize}
  \item EFAS
    \begin{itemize}
      \item en-route flight advisory service (aka ``Flight Watch'')
      \item 122.0 MHz above 5000' AGL
      \item operated by the local center, e.g. ``Los Angeles Flight Watch''
    \end{itemize}
  \item TWEB:
    \begin{itemize}
      \item continuous transcribed weather broadcast.
      \item available over selected NAVAID's (T on chart)
    \end{itemize}
  \item HIWAS
    \begin{itemize}
      \item Hazardous In-flight Weather Advisories Services HIWAS
      \item recorded severe weather advisories
      \item ``H'' on NAVAID on charts
    \end{itemize}
\end{itemize}

NOTAMS (Notices to Airmen):

\begin{itemize}
  \item Notification of unforeseen changes in the national airspace system, not
    known in sufficiently in advance to publicize by other means, that may
    affect the pilot’s decision to make a flight

  \item May be divided into three categories: local (e.g. taxi way closures),
    distant (e.g. VOR out of service), FDC (issued by the Flight Data Center,
    regulatory in nature such as temporary flight restrictions, instrument
    approach procedure changes, etc)

  \item Obtain the same way a briefing can be obtained (FSS, Duats, Duat, etc)
\end{itemize}

\subsection{Evaluation}

\begin{itemize}
  \item Exhibit knowledge of the elements related to weather information by
    analyzing weather reports, charts, and forecasts from various sources with
    emphasis on:
    \begin{itemize}
        %% TODO(sjr): remove references to FA.
      \item METAR, TAF, and FA.
      \item surface analysis chart.
      \item radar summary chart.
      \item winds and temperature aloft chart.
      \item significant weather prognostic charts.
      \item convective outlook chart.
      \item AWOS, ASOS, and ATIS reports.
    \end{itemize}
  \item Make a competent ``go/no-go'' decision based on available weather information.
\end{itemize}

\subsection{References}

AC 00-6A Aviation Weather, FAA-H-8083-25 Pilot's Handbook of Aeronautical
Knowledge Chapter 12.

