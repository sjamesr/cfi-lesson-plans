\chapter{Landing Gear (C172RG)}

\section{Objective}

To teach the components and operating procedures of the landing gear system.

\section{Elements}

\begin{itemize}
  \item Components
  \item Pre-flight
  \item Normal operation
  \item Emergency operation
\end{itemize}

\section{Schedule}

Discussion (45 minutes), in-flight (20 minutes).

\section{Equipment}

Aircraft, Pilot Operating Handbook (POH) or FAA-approved Airplane Flight Manual (AFM).

\section{Instructor Actions}

Discuss the components (show hydraulic schematic, POH 7-28):
\begin{itemize}
  \item Nose gear--nitrogen/oil nose gear shock strut, positive mechanical down
    lock

  \item Nose gear doors--mechanically opened and closed by nose gear

  \item Main gear--tubular spring steel struts, positive mechanical down locks

  \item Hydraulic power pack--electrically driven, located aft of firewall
    between pilot and copilot's rudder pedals
    \begin{itemize}
      \item Pressurized between 1000-1500 psi
      \item Pressure switch causes electric pump to turn on
      \item If the pump stays on, there is a problem
      \item MIL-H-5606, red color, hydraulic fluid
    \end{itemize}

  \item Hydraulic actuators--one for each gear

  \item Landing gear lever--directs pressure

  \item Landing gear position indicator lights--required for flight

    \begin{itemize}
      \item Amber = up, green = down (some models, red gear unsafe light and
        green down light for other models)

      \item Lights are interchangeable

      \item Up and down switches for each gear, in series
    \end{itemize}

  \item Nose gear safety squat switch--open on the ground, prevents inadvertent
    gear retraction

  \item Gear-up warning system--intermittent tone through the speaker if
    manifold pressure $<$12'' Hg or flaps $\geq20^{\circ}$
    \begin{itemize}
      \item Push green light to turn off the tone
    \end{itemize}
  \item Emergency extension hand pump--double action hydraulic pump
    \begin{itemize}
      \item Can't retract the gear with pump
    \end{itemize}

  \item Circuit breakers -- ``pull off'' for gear pump, separate breaker for
    position lights
\end{itemize}

Discuss pre-flight of landing gear:

\begin{itemize}
  \item Cockpit--push to test gear indicator lights
  \item Check that gear handle is down
  \item Check reservoir at 25 hour intervals
  \item Outside--check for leaks
  \item Clear the wheel wells
  \item Make sure squat switch is open
\end{itemize}

Discuss normal operation:

\begin{itemize}
  \item $V_{LO}$ 140 (just below top of green arc), $V_{LE}$ 164 (redline)

  \item 5-7 seconds to extend or retract

  \item Keep hand on lever until operation is complete

  \item Tap brakes before retraction—tires expand due to centrifugal force and
    heat

  \item Mains swing down 2' during retraction

  \item Extend gear before entering traffic pattern

  \item Leave gear extended for continuous traffic pattern operations
\end{itemize}

Discuss manual gear extension:

\begin{itemize}
  \item Not an emergency
  \item Follow the checklist:
    \begin{itemize}
      \item Master ON
      \item Landing gear lever DOWN
      \item Breakers IN
      \item Hand pump--pump about 35 times until gear down light indicates
    \end{itemize}
\end{itemize}

\section{Student Actions}

Demonstrate and explain adequate gear pre-flight. Demonstrate proper use of
landing gear during flight. Conduct a manual gear extension in-flight.

\section{Evaluation}

Lesson is complete when student can demonstrate and discuss proper use of
landing gear in all flight scenarios.

\section{References}

FAA-H-8083-25 Pilot's Handbook of Aeronautical Knowledge p. 5-22,
FAA-H-8083-15A Airplane Flying Handbook p. 11-9, POH / AFM pp. 3-8, 4-18, 4-21,
7-11, 7-27.
